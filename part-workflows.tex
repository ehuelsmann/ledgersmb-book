

\part{Workflows}
\label{part:Workflows}

\chapter{Customers and vendors}

\section{Creating customers and vendors}

\section{Multiple customers within one company}

\section{Creating vendors from customers}

\section{}

\section{Maintaining contact information}


\chapter{Quotations from Vendors and for Customers}

\section{Creating Quotations and RFQs}

\chapter{Sales and vendor orders}

\section{Creating new orders}

\section{Creating orders from quotations}

\section{Creating orders from projects}

\section{Creating purchase orders from sales orders}

\section{Combining orders}

\section{Recurring orders}


\chapter{Sales and vendor invoices}

\section{Creating new invoices}

-- this section is about Sales and Vendor Invoices

\section{Creating invoices from orders}



\section{Invalidating invoices}

Sometimes, it's necessary to invalidate an invoice. When an invoice has been
posted, this also means derived administrations have been updated, such as
inventory for the items on the invoice.

To undo the effects of an invoice, i.e. to reduce the amount outstanding with a
customer, use the \texttt{VOID} button on the invoice screen as shown in @@@figref .
This creates a new invoice by the same number as the original, except that the new
invoice has a suffix \texttt{-VOID}.

\begin{quotation}
Unfortunately, in LedgerSMB 1.3 - the earlier versions - voiding an invoice did not
automatically close the original and voiding invoices.  To close both invoices from
the open invoice overview, use the cash receipt process as described in
\secref{sec:SinglePayments} to make a zero amount payment.
\end{quotation}

\section{Correcting invoices}
\label{sec:CorrectingInvoices}

There's only one way to persist an invoice in LedgerSMB: posting it. This means
the invoice gets added to the accounts. Because one of the primary properties of
an accounting system @@@ continue here

@@@ There's no saved form for invoices, meaning that invoices can only
be required to be corrected *after* being posted.


\chapter{Shop sales}

\section{Opening and closing the cash register}

\section{Shop sales invoices}


\chapter{Manufacturing management}

\section{Producing sales orders}

\subsection{Work orders}


\chapter{Inventory management}

\section{Shipping}

\subsection{Pick lists}
\subsection{Packing list}

\section{Receiving}

\section{Partial shipments}

\section{Transferring between warehouses}

\section{Inventory reporting}



\chapter{Accounts receivable and payable}

\section{Creating generic AR/AP items}

\section{Handling refunds, overpayments and advances}

- this bit is about credit notes and debit notes

\section{Handling returns}

--> this bit is about credit (sales) and debit (vendor) invoices

\chapter{Credit risk management}

\section{Introduction}

A company runs credit risk when it gives credit: it runs the risk of the
creditor not paying off its debts.  LedgerSMB features two ways to manage
the risks involved:

\begin{enumerate}
\item Limit management
\item Arrears management
\end{enumerate}

The former tries to limit the risk involved by making sure no customer
receives more credit than a certain limit while the latter tries to
make sure any over due payments get cashed.

\section{Limit management}

Limit management should prevent a company on one hand from delivering too much
to its customers at once and on the other from taking (and delivering) new orders
to customers with too high amount of unpaid invoices.

@@@ Where to set up limits

@@@ How to monitor limits


\section{Managing arrears}

As mentioned before, the process of managing arrears is directed toward
detecting arrears positions with customers early and taking appropriate
action.


\subsection{Monitoring AR aging}

In order to find out about over due invoices, a company should run the AR
aging report available under AR $\rightarrow$ Reports $\rightarrow$ AR Aging.
The initial screen presents parameters for the aging report to be generated.

@@@ discuss parameters

This report shows customers and their outstanding invoices categorised as:

\begin{description}
\item [Current] Invoices not over due
\item [30] Invoice amounts over due by 30 days or less
\item [60] Invoice amounts over due by 60 days or less (but more than 30)
\item [90] Invoice amounts over due by 90 days or less (but more than 60)
\end{description}


@@@ Printing / mailing aging reports to customers


\subsection{Tracking invoice history}

After an invoice becomes over due a process will be started to remind
the customer of the outstanding amount requiring payment.

In order to keep records of actions taken to chase customer payment,
the invoice screen has an ``Internal Notes'' field which can be edited
after the invoice has been posted.

In order to save any edits to that field, hit the ``Save Info'' button.

\begin{quotation}
Note that the ``Save Info'' button also saves any changes to the ``TaxForm'' column or
rather, any information that's not accounting information (posted to the books and
thereby fixed) nor information which appears on the invoice - which also should remain
unedited in order to be able to generate an exact copy at a later date.
\end{quotation}


\subsection{Special treatment of invoices}

There may be good reasons to treat some over due invoices differently. E.g. in case
payment arrangements have been made with the customer and further standard arrears
management would not be appropriate any more.

In this case, you can put an invoice ``On Hold''. The opposite of being on hold is
being active. The AR Aging report allows selection of all invoices, only active
invoices (those not on hold) or only invoices on hold.


\section{Interest on arrears}

\section{Allowance for doubtful accounts}

\section{Writing off bad debt}

\subsection{Direct write-off}

\subsection{Allowed-for write-off}


\chapter{Receipts and payment processing}


\section{Single receipts and payments}
\label{sec:SinglePayments}

\section{Batch receipts and payments}


\section{Check payments}


\section{Using overpayments}
\label{sec:UsingOverpayments}

\section{Receipt and payment reversal}

\section{Receipts and payments in foreign currencies}

\chapter{Accounting}

\section{Separation of duties: Transaction approval}
\label{sec:SeparationOfDuties}

\section{Period closing}

\section{Bank reconciliation}
\label{sec:Reconciliation}

\section{Year-end processing}

\section{Tax (VAT) reporting}

\begin{itemize}
\item 1099
\item EU VAT
\end{itemize}

\section{Entering general accounting documents}

\section{Fixed asset accounting}
\label{sec:FixedAssetAccounting}

\section{Reporting}
\subsection{Income statement}
\subsection{Balance sheet}
\subsection{Trial balance}


