% !TeX encoding = UTF-8
% !TeX spellcheck = en_US
% !TeX root = ledgersmb-book.tex

\part{General Operation}
\label{part-general-operation}

\chapter{User Interface}
\label{cha-general-operation-user-interfacei}
Description of the general operation of LedgerSMB that appears across multiple views in the application and that are not described elsewhere.

\section{Reports}
\label{sec-general-operation-user-interface-reports}

At the bottom of most report views there is a '[permalink]'. The idea of a \gls{permalink} is that you get a link which you can share with your colleagues and when inserted into the browser, returns exactly the same report (or at least a search with exactly the same parameters as you used).  To copy a  \gls{permalink} the user can right-click and select "Copy link" from the drop down menu and pass it to a colleague by pasting into an email or other messaging App.

\part{Customization}
\label{part-customization}

\chapter{Batch data import methods}
\label{cha-customization-batch-import}

LedgerSMB 1.3 doesn't have batch data imports built in out of the box,
except for bank statement imports for reconciliation purposes. This
chapter starts to explain how to use the one import that is supported
out of the box. It then goes on to highlight some often-used customizations: 
It is quite easy to customize a LedgerSMB 1.3 instance using
the LedgerSMB 1.4-built-in data import routines to create a variety
of file-upload based data imports.

\section{Custom bank statement import}
\label{sec-customization-batch-import-bank-statement}

\section{Data imports back port from 1.4}
\label{sec-customization-import-routines-from-14}

\subsection{Chart of Accounts import}
\label{subsec-customization-import-coa}

This function allows bulk import of an entire chart of accounts using a
single \gls{csv} file.

The first line of the file contains the headers of the columns. The
import routine expects the columns as presented in (table name).

\begin{description}[style=nextline]
\item [accno] Number of the account or header
\item [desc] Description for the account or header
\item [charttype] ``H'' for heading record \\
``A'' for account record
\item [category] ``I'' for Income \\
``E'' for Expense \\
``A'' for Asset \\
``L'' for Liability
\item [contra] ``0'' not a \gls{contra} account \\
``1'' is a \gls{contra} account
\item [tax] ``0'' not a tax account \\
``1'' is a tax account.
\item [link] A colon (':') separated list of account check-mark values (links) as described
    in \secref{subsec-coa-account-links}, or ``AR'', ``AP'' or ``IC'' for the specified summary accounts
\item [heading] Id of the heading to group the account (or heading) under
\item [gifi] \gls{gifi} account number
\end{description}

All lines after the first one are considered to be data lines.

\chapter{Add-ons and plug-ins}
\label{cha-customization-add-ons}


