% !TeX encoding = UTF-8
% !TeX spellcheck = en_US

\part{Overview}
\label{part-overview}
% An overview should contain:

% * why read?
%   * problems being solved
%   * goals to be achieved
%   * 
% * how
%   * erp

\chapter{What is LedgerSMB}
\label{cha-what-is-ledgersmb}

\section{Introduction}
\label{sec-ledgersmb-introduction}
LedgerSMB is an open source, web based \gls{erp}. An ERP is a system
which supports business processes of all disciplines throughout the
organization, automating as much of those processes as possible. To
illustrate this, consider the process of selling goods to a customer
in a trade company. Typically, a customer will request a quote, which
Sales will provide. When satisfied he or she will convert the quote 
to an order. Sales will register the order, leading to order-pickers
to collect one or more shipments. Upon completion, Finance sends
out an invoice and records the customer's payment. LedgerSMB supports
this process by automating the conversion from quotation to order,
order to shipment and shipment to invoice, as well as providing
pack lists and other production related documents.

LedgerSMB includes a
powerful framework which supports building your own extensions and
integrations with other applications. Through this philosophy, it
aspires to be \emph{the} (open source) integrated administration system.

The software is being developed by the LedgerSMB project with its homepage at \url{http://ledgersmb.org/}.
However, the actual project activity can be witnessed at the SourceForge
project site hosted at \url{http://sf.net/projects/ledger-smb/} and the mailing list archive
hosted at \url{http://archive.ledgersmb.org/}.  The project
employs the Freenode.net IRC network's \#ledgersmb chat channel to help out users and to discuss
development.

Its open source nature allows you to download it and use it with any
infrastructure you like. So there's no vendor lock-in: you can
always take your data and set up your system with another hardware vendor
or set up your own hardware.

\section{Application architecture}
\label{sec-ledgersmb-architecture}

@@@TODO

\section{Supported functionality}
\label{sec-ledgersmb-modules}
As most \glspl{erp} LedgerSMB's functionalities are grouped into modules.
Many modules are integrated parts of the base application.  New features
are implemented in separate modules at first to allow evaluation of the
feature set. When a feature set has become sufficiently stable, the new
module will be integrated in the base application. As of that time, the
existing feature set of the module will be frozen, meaning that the utmost
will be done to prevent changes to how the modules operate: to keep them
stable.

These separate modules - which are called \glspl{add-on} - have to be
installed separately. After installation they become seamless parts of
LedgerSMB with no visible difference from the base application. An
additional benefit of having the separation between the base application
and \glspl{add-on} is that it allows for different release schedules
and separate maturity levels.


LedgerSMB 1.3 features the following integrated modules:

\begin{itemize}
\item General ledger
\item Payment and Accounts payable
\item Invoicing and Accounts receivable
\item Fixed asset accounting
\item Time registration and invoicing
\item Point of Sale
\item Quotation and Order management
\item Manufacturing
\item Inventory (warehousing) and shipping management
\item VAT reporting (cash based)
\item Controlling
\begin{itemize}
\item Project accounting
\item Department accounting
\end{itemize}
\item Application administration
\end{itemize}

These add-ons can be installed:
\begin{itemize}
\item Budgeting
\item Enhanced AR and AP support
\item VAT reporting (accrual based)
\item Enhanced trial-balance report
\item Enhanced recurring transactions
\item Payroll (to be created - under discussion at the time of writing)
\end{itemize}

With this list of modules and add-ons LedgerSMB has succesfully been implemented
in a wide range of companies of varying types and sizes: shops, manufacturing
companies and service oriented businesses up to as big as four thousand (4.000)
Accounts payable transactions \emph{per week}. 

\section{Feature comparison with alternatives}
\label{sec-ledgersmb-feature-comparison}
@@@ TODO

Packages to compare to:

\begin{itemize}
\item GNUcash
\item OSfinancials
\item ERP5
\item OpenERP
\item xTuple
\end{itemize}

\section{Release history}
\label{sec-ledgersmb-release-history}

The project started out as a fork of SQL-Ledger - the open source \gls{erp}
developed by Dieter Simader - somewhere between SQL-Ledger versions 2.6
and 2.8.  A fork happens when a group of developers can't - for whatever
reason - continue to work as one group on a project.  At that time, the
project splits into two or more projects and the fork is in effect.

LedgerSMB split off from the SQL-Ledger project (i.e. forked) because
there was disagreement between developers about how to go forward both with
respect to handling of security vulnerability reports as well as the general
state of the code base.

After the fork, between versions 1.0 and 1.2 a lot of energy was spent on
making LedgerSMB more secure (i.e. less vulnerable).  In technical terms,
measures were taken to fend off (amongst other things):

\begin{itemize}
\item Cross site scripting attacks
\item Replay attacks
\item SQL injection attacks
\end{itemize}

Come version 1.3 the development directed toward improvement of the overall
quality of the code base as the old SQL-Ledger code was in very poor state:
looking very much like webscripts as they were written in 1998, the code had
grown largely outdated in style and was no longer maintainable by 2007.

The 1.3 effort focussed on bringing relief on that front by introducing
modern structure into the application.  With the new application structure
modern and important features were realized: separation of duties (for the
accounting part of the application) and authorizations to allow distinguishing
different roles in your company.

Unfortunately, by the beginning of 2011 the project looked mostly dead from
an outside perspective: the team had not brought forward any releases since
2007, there were no signs of development activity and the
mailing lists (a measure for community activity) were
completely silent.  SVN commits were continuing, but were being made by ever 
fewer committers and contributors.

Fortunately development activity was showing again in the first half 
year of 2011,
leading to the release of version 1.3 by September.  Between September and the
year end in total 10 small bug fixes were released, showing active commitment
of the developers to maintain the application.

New committers showed up, indicating revived community interest. Other signs of
increased interest are the higher number of bug reports and the creation of the
linux package for Debian 7, which has been included in Ubuntu 12.04 as of
October 2012.



\section{System requirements}
\label{sec-ledgersmb-system-requirements}

The INSTALL file which comes with every LedgerSMB software release should be
considered the authorative source of system requirements. In summary, the
following technical components are required:

\begin{itemize}
\item The Apache web server (others may or may not work)
\item Perl 5.10 or higher, with additional modules
\item PostgreSQL 8.3 or higher, with contribs
\item \LaTeX or XeLaTeX from the TeTeX or TexLive \TeX distributions
\end{itemize}

Other system requirements such as required RAM and number of CPUs and their speed
largely depend on the expected system activity. However, any modern VPS should
provide enough memory and storage to satisfy a reasonable number of users.

\section{License}
\label{sec-ledgersmb-license}

LedgerSMB is being made available under the terms of the
\textit{GNU Public License version 2}, or shorter \textit{GPLv2}
\footnote{\url{http://www.opensource.org/licenses/GPL-2.0}}.

The project attaches this meaning to the license:
The copyright holders grant you the right to copy and
redistribute the software.  In case you make any modifications to the sofware
you're obligated to make public those changes.  You are however, free to use
the APIs from your own software without being required to publish your own software.

The project considers the following to be APIs:
\begin{itemize}
\item Database tables
\item URLs with their input and output
\item Webservices of any kind
\item Function and object calls
\end{itemize}

The effect of this interpretation is that changes directly to the code base as
well as inheritance of classes defined in the software constitute ''making modifications''.

\chapter{Reasons to use LedgerSMB}
\label{cha-advocacy}

Jack finds several tools which suit his requirements to some extent or another.
After evaluation of his options he decides to use LedgerSMB for the following reasons:

\begin{itemize}
\item Centralized data storage
\item Actively developed
\item Development team with security focus
\item Access to the application requires only a web browser
\item Integrated sales, shipping, invoicing, purchasing and accounting
\item Open source solution, so no vendor lock in
\item The roadmap appeals to him, because it has web services payrolling on it
\item There are multiple vendors offering commercial support, including hosted options
\item The developers envision building a platform out of it: creating the building blocks
required to build a company on
\end{itemize}


\begin{itemize}
\item Own your own data
\item Freedom to change
	\begin{itemize}
	\item Support organization
	\item Developer (organization)
	\item Data storage provider
	\item Application service provider
	\end{itemize}
\end{itemize}

\section{Internal control}
\label{sec-advocacy-internal-control}

Internal control\footnote{See also \url{http://en.wikipedia.org/wiki/Internal_control}}
helps organizations to prevent and detect fraud by introducing checks and balances
to assess effectiveness and validity of transactions in the organization and thereby
in its \gls{erp} and accounting system(s).

\section{Accounting principles}
\label{sec-system-accounting-principles}

The accounting guidelines \gls{ias} and \gls{ifrs} describe requirements
to financial statements (reports) and the underlying accounting process
\footnote{See also \url{http://en.wikipedia.org/wiki/International_Financial_Reporting_Standards}}.
Said requirements include qualitative characteristics:

\begin{enumerate}
\item Relevance
\item Faithful representation
\item Comparability
\item Verifiability
\item Timeliness
\item Understandability
\end{enumerate}

While some - if not most - of these characteristics relate to the process of accounting,
the ``Verifiability'' item clearly has an impact on the underlaying accounting systems:
In order to be verifiable, there must be a clear audit trail to show the origin of the
figures. To make sure users leave behind the required audit trail, some actions can't
be performed in LedgerSMB, even though it would seem to be a logical requirement to be
able to do so - from the perspective of a non-accountant.


\section{Impact of tight integration}
\label{sec-system-impact-of-tight-integration}

While both the qualitative characteristics from \gls{ifrs} and the checks and balances
from the internal controls are pose restrictions on the accounting process,
sometimes these restrictions require support from the underlying accounting
software.

One example is the support for creating reliable audit trails
by protecting accounting data from deletion. It's important to realize the scope
of accounting data in this respect: because invoices are being registered in the
accounts receivable administration - which is summarized in the general ledger -
they are part of the data for which the audit trail needs to be recorded.

Another example is separation of duties (also known as the ``four eye principle''),
where one accountant enters financial transactions and another is responsible for
posting them. This procedure protects the company from an accountant single-handedly
faking transactions and possibly masking fraud.

The requirements for good accounting processes and internal control have impact
on the work flows supported by LedgerSMB. As a consequence some of the work flows
described in part \ref{part-workflows} may seem unwieldy; an example being the
lack of functionality to delete or correct incorrect invoices (See \secref{sec:CorrectingInvoices} for more details).



\chapter{Introduction to accounting}
\label{cha-accounting-introduction}


\section{Valuation of inventory}
\label{sec-accounting-valuation-inventory}


