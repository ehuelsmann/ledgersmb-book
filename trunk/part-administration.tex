

\part{Administration}
\label{part:Administration}

%\section{Introduction}
%This section of the book describes which tasks and processes might need to be carried out
%on a regular basis in order to keep the application in good health and in line with
%end-users requirements.  As a result this part is more related to the functional rather
%than technical parts of the application.
%
%The fact that the tasks described in this part of the book may be recurring during the
%lifetime of the application does not exclude them from being part of the setup phase
%of LedgerSMB.  In other words: you're likely to have to dig into this section when
%creating a company as well as when maintaining it.

\chapter{Overview}


(Within-application tasks)

(DBA tasks from setup.pl)

(Outside-application tasks)

\chapter{User management}

\section{User creation}

\section{User authorization}

\section{Same user in multiple companies}

\chapter{Definition of products and services}

\section{Definition of products}
Structure of products in the system.

\subsection{Definition of parts}
\label{sec:DefinitionOfParts}

\subsection{Definition of partgroups}
\subsection{Definition of assemblies}
\subsection{Definition of overhead}

\section{Definition of services}

\chapter{Chart of accounts}

\section{Accounts and headers}

The system allows ordering accounts into groups by assigning accounts to headers. Headers
can themselves be assigned to other headers resulting in trees of account groups.\footnote{Although the database structure supports this type of account hierarchy
doesn't the 1.3 user take advantage of it yet: in 1.3 accounts can be assigned a header,
but headers can't be assigned to headers themselves.}



\section{Account configuration}

Headers don't have any configuration, other than their number and description. Accounts also
have a number and description, but require additional configuration for the application to work
correctly. The settings are described in the sections that follow.

\subsection{Account options}
\label{sec:AccountOptions}
\begin{itemize}
\item Contra This checkmark identifies the account as a contra account, which means
   that the account is going to hold the opposite of an account it's associated with.
   A good example of this kind would be the depreciation account associated with a fixed
   asset account where the depreciation account contains the credit amount to be added to
   the original asset (debit) value to get the current asset value.
\item Recon This checkmark identifies the account as one which needs reconciliation as
   described in section \ref{sec:Reconciliation}.
\item Tax This checkmark identifies the account as a Tax (VAT) account. Tax accounts need
   to be further configured. See chapter \ref{cha:Taxes} for further discussion of the
   subject.
\end{itemize}

\subsection{Summary accounts}

There are currently three types of summary accounts:

\begin{enumerate}
\item AR Marking an account as a summary account for AR means that all outstanding
   receivable amounts will be posted to this account. The Accounts Receivable administration
   will contain the details of which amount is owed by which customer.
\item AP Same as the AR account, except for amounts owed to vendors.
\item Inventory This account holds the monetary value equal to the items on stock.
\end{enumerate}

\subsection{Account inclusion in drop down lists}

@@@ Add summary

\subsubsection{Receivables \& payables UI}

\begin{itemize}
\item[Income (AR\_amount)] ...
\item[Payment (AR\_paid)] This check mark adds the account to the list of accounts
   to choose from in the Receipts (AR) and Payments (AP) screens. Additionally, it
   adds the account to the part entry screen as described in section \ref{sec:DefinitionOfParts}.
\item[Tax (AR\_tax)] This check mark makes the account show up as a check mark on the
   customer (AR) or vendor (AP) entry screen. See chapter \ref{cha:Taxes} for further discussion.
\item[Overpayment (AR\_overpayment)] Adds the account to the receipts screen as discussed
   in section \ref{sec:UsingOverpayments}.
\item[Discount (AR\_discount)] Adds the account to the customer entry screen's selection
   list for accounts to post 
\end{itemize}

The payables UI works the same way as does the receivables UI. The difference is
that the technical names of the configuration identifiers are prefixed by AP\_ instead
of AR\_.

\subsubsection{Tracking Items}

The items on this line relate to stocked items, i.e. those tracked for inventory: parts and
assemblies.

\begin{enumerate}
\item[Income (IC\_sale)] Adds the account to the selection list of income accounts on the
   part and assembly definition screens.
\item[COGS (IC\_cogs)] Adds the account to the selection list of COGS @@@ accounts on the
   part, assembly and overhead definition screen.
\item[Tax (IC\_taxpart)] Adds a check mark to the part and assembly definition screen
   for the applicable account. See \ref{cha:Taxes} for more details on how taxes
   work in LedgerSMB.
\end{enumerate}

@@@ Question: Labor/Overhead accounts == inventory accounts??

\subsubsection{Non-tracking items}

The items on this line relate to untracked (non stocked) items, i.e. services.

\begin{enumerate}
\item[Income (IC\_income)] Adds the account to the income account selection list in
   the service definition screen.
\item[Expense (IC\_expense)] Adds the account to the expense account selection list in
   the service definition screen.
\item[Tax (IC\_taxservice)] Adds a check mark to the service definition screen for the
   applicable account. See \ref{cha:Taxes} for more details on how taxes work in LedgerSMB.
\end{enumerate}

\subsubsection{Fixed assets}

\begin{enumerate}
\item[Fixed asset (Fixed\_Asset)] Marks the account as holding the original asset value for the fixed
   assets module, for some classes of fixed assets.
\item[Depreciation (Asset\_Dep)] Marks the account as holding the cumulative depreciation amount
   for the fixed assets module, for some classes of fixed assets.
\item[Expense (asset\_expense)] Adds the expense account to the selection list of the fixed assets
   accounting module. See section \ref{sec:FixedAssetAccounting} for more details.
\item[Gain (asset\_gain)] Account to hold book value gain upon disposal of a fixed asset.
\item[Loss (asset\_loss)] Account to hold book value loss upon disposal of a fixed asset.
\end{enumerate}


\section{Alternative accounts (GIFI)}

Next to the regular account numbering scheme, LedgerSMB supports a second numbering scheme: GIFI numbering. The GIFI accounts are a kind of secondary numbering scheme to support legal requirements.

Some jurisdictions require a specific numbering scheme, which can be supported using GIFI. If you
use GIFI account numbers, each account is associated with a GIFI account. Multiple accounts may map
to a single GIFI account.

Many General Ledger reports exist in two variants: a variant using the normal G/L accounts and
one with the GIFI numbering scheme. In the GIFI variant, when a single GIFI has multiple accounts,
the total reported under GIFI is the sum of the mapped accounts.


\subsection{Maintaining GIFI}

GIFI accounts should be created before being assigned to a standard G/L account. GIFI accounts
can be maintained through the System $\rightarrow$ Chart of accounts $\rightarrow$ Add GIFI and List GIFI menu items. Existing accounts can be edited by selecting them from the List GIFI menu, which opens a page where individual GIFI items can have their number or
description adjusted.


\chapter{Taxes}
\label{cha:Taxes}

\section{Overview}



\section{Tax calculation plug-ins}
\label{sec:TaxRulePlugins}

\chapter{Pricing}

\section{Definition of types of business}

\section{Definition of price groups}

\chapter{contingency planning}

\section{Backup and restore}

\section{Advanced PostgreSQL: replication}

\chapter{Software updates}

\section{LedgerSMB patch release roll out}





