

\part{Administration}
\label{part:Administration}

%\section{Introduction}
%This section of the book describes which tasks and processes might need to be carried out
%on a regular basis in order to keep the application in good health and in line with
%end-users requirements.  As a result this part is more related to the functional rather
%than technical parts of the application.
%
%The fact that the tasks described in this part of the book may be recurring during the
%lifetime of the application does not exclude them from being part of the setup phase
%of LedgerSMB.  In other words: you're likely to have to dig into this section when
%creating a company as well as when maintaining it.

\chapter{Overview}
\label{cha:AdministrationOverview}

\section{Introduction}

This part of the book describes the tasks and processes that may need to be carried out
on a regular basis in order to keep the application in good health and in line with
requirements from end users.

Maintenance may require different types of system access for different types of tasks:

\begin{enumerate}
\item Within application tasks, such as user management, require an appropriately authorized
   normal application login
\item Database administration tasks, such as backups and application upgrades,
 require
   a database level login to be used with 'setup.pl'
\item Other system-level maintenance tasks, such as updating PostgreSQL or Apache, which
   require user accounts on the server hosting LedgerSMB
\end{enumerate}

% @@@ (Within-application tasks)

% @@@ (DBA tasks from setup.pl)

% @@@ (Outside-application tasks)

\chapter{User management}

\section{User creation}
\label{sec:user-creation}

Users experienced with LedgerSMB 1.2 or before or SQL-Ledger (any version) are
referred to appendix \ref{sec:DifferencesUsers} to read about the differences
with version 1.3.

In order to create users, the current user must be sufficiently authorized. The user
created at application set up time is such a user.

Go to the System $\rightarrow$ Admin Users $\rightarrow$ Add User. You'll be presented the page as shown in figure \figref{fig:create-user-step1}.

\begin{figure}[h]
\includegraphics{create-user-step1.png}
\caption{Screen for user creation - step 1}
\end{figure}
\label{fig:create-user-step1}

The value entered in the 'Username' field will cause a database user by that name
to be created. Database users are a global resource, meaning that a collision will
occur if multiple people try to define the same user in multiple companies.
\secref{sec:UserImports} describes how to use the same user across multiple companies.

Enter the password to be used for this user into the ``Password'' field. If you're
importing a user, please leave the field empty -- that will prevent the password
from being changed.

The ``Import'' field is discussed in \secref{sec:UserImports}. To create a new
user, leave the setting at ``No''.

All of the ``First Name'', ``Last Name'' and ``Employee No.'' fields are required.
However, when no employee number is specified, the system will generate one using
the sequence specified in the Defaults screen as documented in
\secref{sec:DefaultsItemNumbering}.

The ``Country'' field speaks for itself and is required as well.


\section{User authorization}


After filling out all the fields as described in the previous section and
clicking ``Save user'', you'll be presented
a second screen in the user creation process: the user authorization screen.
See \figref{fig:create-user-step2} for a screenshot of the top of that screen.

\begin{figure}[h]
\includegraphics{create-user-step2.png}
\caption{Screen for user creation - step 2}
\end{figure}
\label{fig:create-user-step2}

The process of assigning user authorizations is the process by which the granted
access to specific parts of the application. One can imagine that - in a moderately
sized company - sales should not be editing accounting data and accountants should
not be editing sales data. Yet, in order to cooperate, both parties need to be
given access to the same application. This is where authorizations come in.

In aforementioned screen, which equals the ``Edit user'' screen, you have to assign the
newly created user his application rights. By default, the user doesn't have any
rights. Checking all check marks makes the user an application ``super user'', i.e.
gives the user all available application rights.

For a description of the roles a user can be assigned and their effects, the
reader is referred to \appref{cha:RolesListing}.


\section{Maintaining users}

\subsection{Editing user information and authorizations}

When the role of a user in the company changes, it may be necessary to assign
that user new roles and possibly revoke some other roles. This can be done through
user search: System $\rightarrow$ Admin Users $\rightarrow$ Search Users $\rightarrow$ Search $\rightarrow$ {[}edit] which brings you to the same screen as presented in
\figref{fig:create-user-step2}.

Similarly, there may be reasons to change the user information, such as a last name
(e.g. upon marriage).

\subsection{Deleting users}

From the ``User search'' result screen, users can be ``deleted'' from the company:
they have their access to the current company revoked.

@@@ Does the delete button delete the user from the cluster as well as from the application?


\section{User imports}
\label{sec:UserImports}

If a database user already exists, e.g. because this user was created to be used
with another LedgerSMB company, it can't be created a second time. In order to be
able to use that user with the current company, it needs to be ``imported'' instead.

The difference between creating a new user and importing one is that the ``Import''
radio button should be set to ``Yes'' and that you should not fill out a password.
If you do, the password of that user will be overwritten for all companies.

All other fields are still applicable: the data entered for other companies isn't
copied to the current company.


\section{setup.pl users}

Users created through the process documented above don't have enough rights to
execute actions through the setup.pl database administration tool. Note that this
is on purpose. You will need access to the server to create such users, or request
one from your application service provider (ASP) if you use a hosted solution.

During the set up process such a user is normally being created. This user can later
be used to manage the database from the database administration tool setup.pl.


\chapter{Definition of products and services}
\label{cha:ProductsDefinition}


\section{Definition of products}
Structure of products in the system.

\subsection{Definition of parts}
\label{sec:DefinitionOfParts}

\subsection{Definition of partgroups}
\subsection{Definition of assemblies}
\subsection{Definition of overhead}

\section{Definition of services}

\chapter{Chart of accounts}

\section{Accounts and headers}

The system allows ordering accounts into groups by assigning accounts to headers. Headers
can themselves be assigned to other headers resulting in trees of account groups.\footnote{Although the database structure supports this type of account hierarchy
doesn't the 1.3 user take advantage of it yet: in 1.3 accounts can be assigned a header,
but headers can't be assigned to headers themselves.}



\section{Account configuration}

Headers don't have any configuration, other than their number and description. Accounts also
have a number and description, but require additional configuration for the application to work
correctly. The settings are described in the sections that follow.

\subsection{Account options}
\label{sec:AccountOptions}
\begin{itemize}
\item Contra This checkmark identifies the account as a contra account, which means
   that the account is going to hold the opposite of an account it's associated with.
   A good example of this kind would be the depreciation account associated with a fixed
   asset account where the depreciation account contains the credit amount to be added to
   the original asset (debit) value to get the current asset value.
\item Recon This checkmark identifies the account as one which needs reconciliation as
   described in \secref{sec:Reconciliation}.
\item Tax This checkmark identifies the account as a Tax (VAT) account. Tax accounts need
   to be further configured. See \charef{cha:Taxes} for further discussion of the
   subject.
   \label{item:AccountOptionsTax}
\end{itemize}

\subsection{Summary accounts}

There are currently three types of summary accounts:

\begin{enumerate}
\item AR Marking an account as a summary account for AR means that all outstanding
   receivable amounts will be posted to this account. The Accounts Receivable administration
   will contain the details of which amount is owed by which customer.
\item AP Same as the AR account, except for amounts owed to vendors.
\item Inventory This account holds the monetary value equal to the items on stock.
\end{enumerate}

\subsection{Account inclusion in drop down lists}

@@@ Add summary

\subsubsection{Receivables \& payables UI}

\begin{itemize}
\item[Income (AR\_amount)] This check mark adds the account to the list of accounts
   in the transaction and invoice screens which are used to post income on.
\item[Payment (AR\_paid)] This check mark adds the account to the list of accounts
   to choose from in the Receipts (AR) and Payments (AP) screens. Additionally, it
   adds the account to the part entry screen as described in \secref{sec:DefinitionOfParts}.
\item[Tax (AR\_tax)] This check mark makes the account show up as a check mark on the
   customer (AR) or vendor (AP) entry screen. See \charef{cha:Taxes} for further discussion.
\item[Overpayment (AR\_overpayment)] Adds the account to the receipts screen as discussed
   in \secref{sec:UsingOverpayments}.
\item[Discount (AR\_discount)] Adds the account to the customer entry screen's selection
   list for accounts to post 
\end{itemize}

The payables UI works the same way as the receivables UI. The difference is
that the technical names of the configuration identifiers are prefixed by AP\_ instead
of AR\_.

\subsubsection{Tracking Items}

The items on this line relate to stocked items, i.e. those tracked for inventory: parts and
assemblies.

\begin{enumerate}
\item[Income (IC\_sale)] Adds the account to the selection list of income accounts on the
   part and assembly definition screens.
\item[COGS (IC\_cogs)] Adds the account to the selection list of COGS @@@ accounts on the
   part, assembly and overhead definition screen.
\item[Tax (IC\_taxpart)] Adds a check mark to the part and assembly definition screen
   for the applicable account. See \charef{cha:Taxes} for more details on how taxes
   work in LedgerSMB.
\end{enumerate}

@@@ Question: Labor/Overhead accounts == inventory accounts??

\subsubsection{Non-tracking items}

The items on this line relate to untracked (non stocked) items, i.e. services.

\begin{enumerate}
\item[Income (IC\_income)] Adds the account to the income account selection list in
   the service definition screen.
\item[Expense (IC\_expense)] Adds the account to the expense account selection list in
   the service definition screen.
\item[Tax (IC\_taxservice)] Adds a check mark to the service definition screen for the
   applicable account. See \charef{cha:Taxes} for more details on how taxes work in LedgerSMB.
\end{enumerate}

\subsubsection{Fixed assets}

\begin{enumerate}
\item[Fixed asset (Fixed\_Asset)] Marks the account as holding the original asset value for the fixed
   assets module, for some classes of fixed assets.
\item[Depreciation (Asset\_Dep)] Marks the account as holding the cumulative depreciation amount
   for the fixed assets module, for some classes of fixed assets.
\item[Expense (asset\_expense)] Adds the expense account to the selection list of the fixed assets
   accounting module. See \secref{sec:FixedAssetAccounting} for more details.
\item[Gain (asset\_gain)] Account to hold book value gain upon disposal of a fixed asset.
\item[Loss (asset\_loss)] Account to hold book value loss upon disposal of a fixed asset.
\end{enumerate}


\section{Alternative accounts (GIFI)}

Next to the regular account numbering scheme, LedgerSMB supports a second numbering scheme: GIFI numbering. The GIFI accounts are a kind of secondary numbering scheme to support legal requirements.

Some jurisdictions require a specific numbering scheme, which can be supported using GIFI. If you
use GIFI account numbers, each account is associated with a GIFI account. Multiple accounts may map
to a single GIFI account.

Many General Ledger reports exist in two variants: a variant using the normal G/L accounts and
one with the GIFI numbering scheme. In the GIFI variant, when a single GIFI has multiple accounts,
the total reported under GIFI is the sum of the mapped accounts.


\subsection{Maintaining GIFI}

GIFI accounts should be created before being assigned to a standard G/L account. GIFI accounts
can be maintained through the System $\rightarrow$ Chart of accounts $\rightarrow$ Add GIFI and List GIFI menu items. Existing accounts can be edited by selecting them from the List GIFI menu, which opens a page where individual GIFI items can have their number or
description adjusted.


\chapter{Taxes}
\label{cha:Taxes}

\section{Overview}

When an account has been marked as a Tax account (see \secref{sec:AccountOptions}, item
\ref{item:AccountOptionsTax}) several things happen:

\begin{itemize}
\item The account will be shown in the customer and vendor account screens with
   a check mark to mark it ``relevant for the customer''
\item The account will be shown in the part and service configuration screens
   with a check mark to mark it ``relevant for the part/service item''
\item The account will be shown in the tax configuration screen in order to set
   a tax percentage on the account
\end{itemize}

By marking an account relevant for a customer, taxes will be calculated when
creating a sales invoice for the given customer which includes parts which also
have the specific account marked as relevant.

\section{Tax account configuration}

@@@ What does the ``order'' field do again??

\section{Tax calculation plug-ins}
\label{sec:TaxRulePlugins}

The tax calculation system has been designed to handle the most complex tax system
thinkable. Because the tax calculation rules for most set ups aren't really all that
complex, LedgerSMB comes with a single tax calculation plug in out of the box: the
``Simple'' tax calculation rule.

For more complex needs, more complex routines can be developed and plugged into
LedgerSMB side-by-side with the simple rule.

\chapter{Pricing}

\section{Introduction}

In the application there are four reasons for an invoice to include a discount:

\begin{enumerate}
\item Because a discount is entered by the sales person creating the quotation,
  sales order or sales invoice
\item Because the customer's payment terms include a discount for paying early
\item Because the customer belongs to a type of business which is entitled to discounts
\label{item:PricingBusinessType}
\item Because the customer has been assigned a price matrix leading to discounts
\label{item:PricingPriceMatrix}
\end{enumerate}

This section deals with items \ref{item:PricingBusinessType} and \ref{item:PricingPriceMatrix} only.

@@@ types of business are 'old school'; price groups have been introduced to replace types of business with a more fine-grained structure.

\section{Definition of types of business}

Types of business are really straight forward: they feature a description
which allows them to be identified in the customer account screen and a discount
percentage which is applied across the board. I.e. all invoices to the customer
will have that discount applied.

\section{Definition of price groups / price matrix}

\chapter{Contingency planning}

\section{Backup and restore}

\subsection{Backup using setup.pl}

\subsection{Backup using PostgreSQL administration tools}

\subsection{Restore}

\section{Advanced PostgreSQL: replication}

\chapter{Software updates}

\section{LedgerSMB patch release roll out}

\chapter{Optional Features}

As business requirements change sometimes it may be necessary to add or remove
some of the optional features of LedgerSMB.  This chapter describes how these
optional features work, how to troubleshoot them of things go wrong, and how to
enable or disable them.

\section{PDF and Postscript Documents}

LedgerSMB can create PDF and Postscript documents for orders, invoices, and
more.  This is an optional feature and requires some additional software not
included with LedgerSMB, including a \LaTeX\ distribution and extensions to 
Perl's TemplateToolkit framework.

The PDF and PS invoices are generated using a program called \LaTeX\
which handles the layout and typesetting.  The actual \LaTeX\ files are
creating using Template Toolkit with extensions for \LaTeX.  These
extensions are in the Template::Latex package available from CPAN.
The software then generates a \LaTeX\ file which is then processed to
create a PDF or PS.

Typically the first thing to do is to install a \LaTeX\ distribution
like TexLive (distributed with many Linux distributions and available
for OSX and Windows).  This provides \LaTeX\ and many of the modules
needed.   In general I recommend that if your distro has a
texlive-extras package that you install this too.

After this is installed, you must then install Template::Latex.  This
can be done by typing on the command line:

\# cpan Template::Latex

This will also install a number of dependencies including
LaTeX::Driver, which will need to know where your LaTeX binaries are.
It is usually pretty good at finding them.

If things go wrong and you can't get it to work, the following
commands may provide useful diagnostic information when requesting
help:

From the LedgerSMB application directory:
\$ perl -MLedgerSMB::Template::Latex -e print

From the doc/manual directory in the LedgerSMB application directory:
\# pdflatex LedgerSMB-manual

\section{Attaching Uploaded Files to PDF Invoices}

\section{OpenDocument Spreadsheet Output}

\section{Microsoft Excep Output}


