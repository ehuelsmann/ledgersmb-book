

\part{Workflows}
\label{part:Workflows}

\chapter{Customers and vendors}

\section{Creating customers and vendors}

\section{Multiple customers within one company}

\section{Creating vendors from customers}

% \section{}

\section{Maintaining contact information}


\chapter{Quotations from Vendors and for Customers}

\section{Creating Quotations and RFQs}

\section{Attaching files to quotations}
\label{sec:FileAttachments}


\chapter{Sales and vendor orders}
\label{cha:OrderManagement}

\section{Creating new orders}

\section{Creating orders from quotations}

\section{Creating orders from projects}

\section{Creating purchase orders from sales orders}

\section{Combining orders}

\section{Recurring orders}
\label{sec:RecurringOrders}

% @@@ ### Not getting these to work in the test system. Need to check with Chris.

\chapter{Sales and vendor invoices}

Invoices can come from multiple sources. When the quotation and order
management functionalities in LedgerSMB aren't used, they will usually
be entered manually. This work flow is covered in Section
\ref{sec:ManuallyCreatingInvoices}.
When order management \emph{is} being used they mostly originate from orders
which is covered in Section \ref{sec:InvoicesFromOrders}.



\section{Creating invoices from orders}
\label{sec:InvoicesFromOrders}

% This section first because the previous chapter is about orders,
% which seems like a natural flow for the book


\section{Creating new invoices}
\label{sec:ManuallyCreatingInvoices}


%% this section is about Sales and Vendor Invoices

When a business decides not to use the order management as per the previous
chapter it may find itself in need to manually enter invoices. But even
if it does use order management, it may be necessary to enter an invoice
directly.

When creating a transaction to record that the company owes another
entity (a vendor invoice) or that it has outstanding receivables,
LedgerSMB offers two options:

\begin{enumerate}
\item Invoices
\item Transactions
\end{enumerate}

Transactions have very limited functionality: they allow a user to enter
a debt owed or owned into the AR and AP subsystems. They also require the
user to think how the other side of the transaction should be registered;
i.e. which cost account the AP transaction should be posted against, or
which income account the AR transaction should be posted against. If there
are sales taxes applicable, the user is required to manually calculate and
enter them.

Invoices offer a much more clever set of functionalities. First of all, it
allows the user to create a document to be sent to the vendor or customer.
Second, invoices take advantage of parts and services
to automate calculation of sales taxes. Third, invoices update inventory
for items held in stock (parts, assemblies). Transactions offer none of this.

\subsection{Invoices}

As mentioned in the previous paragraph, invoices can perform automatic
sales tax calculations, maintain inventory and post income (or expense)
to the correct GL accounts.

To be able to do so, they need the items on the invoice to be correctly
configured. See \charef{cha:ProductsDefinition} how to set up products
and services.

% @@@ Invoice entry screen screenshot(s)

\subsection{Transactions}

Transactions serve an important purpose not handled by invoices: payroll
calculations are often too difficult to fit in the simple ``amount times price''
model offered by invoices. In order to still be able to track which ``vendor''
was paid which amount such payment obligations can be recorded in the AP subsystem
with a transaction.

Likewise it's often more hassle than it's worth to create the parts and services
required to correctly calculate the utility bill. In such cases the transaction
(possibly with a linked document as supporting evidence) offers good per-vendor
traceable history records.

% @@@ Transaction creation screen shot(s)

\section{Recurring invoices}



\section{Invalidating invoices}

Sometimes, it's necessary to invalidate an invoice. When an invoice has been
posted, this also means derived administrations have been updated, such as
inventory for the items on the invoice.

To undo the effects of an invoice, i.e. to reduce the amount outstanding with a
customer, use the \texttt{VOID} button on the invoice screen as shown in @@@figref .
This creates a new invoice by the same number as the original, except that the new
invoice has a suffix \texttt{-VOID}.

\begin{quotation}
Unfortunately, in LedgerSMB 1.3 - the earlier versions - voiding an invoice did not
automatically close the original and voiding invoices.  To close both invoices from
the open invoice overview, use the cash receipt process as described in
\secref{sec:SinglePayments} to make a zero amount payment.
\end{quotation}

\section{Correcting and deleting invoices}
\label{sec:CorrectingInvoices}

There's only one way to persist an invoice in LedgerSMB: posting it. This means
the invoice becomes part of the accounting information. One of the primary
properties of an accounting system is to record full audit trails and help enforce
internal controls as detailed in \secref{sec:AccountingSystemRequirements}. Because
of that fact there's no way to delete or edit invoices after they have been posted
\footnote{LedgerSMB currently does not support saving an invoice without posting
it. This functionality is on the roadmap for addition when the AR/AP functionality
is being rewritten - currently 1.5 or 1.6.}.

The only way to ``undo'' an invoice is by voiding it. This is important for several
reasons:

\begin{enumerate}
\item Invoices can't be deleted (because they're accounting data)
\item Invoices pose a claim on the assets of a customer
\label{item:InvoicesAsClaims}
\item @@@ others?
\end{enumerate}

Specifically item \ref{item:InvoicesAsClaims} is important: when you sent the invoice
to your customer, you effectively sent them a claim. When you decide to refrain from
pursuing that claim, you should notify them of that fact so they have the documentation
to update their accounting system to reflect that fact: they need your documentation
to void their vendor invoice, instead of paying it.

For the same reason it's ill-advised (and no longer supported) to edit invoices:
when a customer has multiple invoices, each stating a different amount, all
using the same invoice number; how is that customer supposed to document (verifiably)
that the claim has been settled satisfactorily by paying the one he did?

See the Remarks section at the end of this chapter for details on how to handle
the draft invoice requirement.

\section{Handling invoice disputes}

%When an invoice has been sent to a customer it could happen that the customer disagrees
%with the invoice due to incorrect amounts, discounts, products, etc.
%
%Depending on whether the customer has already paid the invoice there are several
%options to handle this situation:
%
%@@@ AR Invoices have nothing to do with it! They parallel AR Transactions!!
%
%\begin{description}
%\item [Credit invoices] The customer has not paid the invoice yet
%\item [Credit notes] The customer has paid the invoice and allows offsetting against
%   other (possibly future) invoices
%\item[AR vouchers] The customer has paid the invoice and demands repayment
%\end{description}
%
%Note that the naming in the listing above applies to AR items but could equally apply
%to AP items by replacing the word ``Credit'' by ``Debit'' and ``AR'' by ``AP''.
%
%
%
%\subsection{Credit and debit invoices}
%
%@@@ produce an accounting document to exchange with the customer/vendor
%@@@ restock credited parts
%@@@ reverse taxes
%
%\subsection{Credit and debit notes}
%
%@@@ generic transaction, not related to orders, inventory or anything else
%@@@ amount taken out of an income account, which is to be selected
%@@@ i.e. based on the account on which the original income was posted.
%
%\subsection{AR and AP Vouchers}
%
%@@@ like a credit note or debit note, meant to initiate
%@@@ a repayment to the customer
%
%@@@ exists as a batch workflow only -- to support separation of duties


\section{Remarks}

\begin{description}
\item [Why can't I send a draft invoice to a customer and edit it
   to match their expectations?] 
You can't edit invoices any more in LedgerSMB 1.3 because it breaks the audit trail
in financial accounting. But in fact there's functionality available which is meant
exactly for this purpose. It's called ``Sales order'' and its details are in
\charef{cha:OrderManagement}. Sales orders can be converted - upon customer approval -
into an invoice with a click of a button.
\end{description}


\chapter{Shop sales}

\section{Opening and closing the cash register}

\section{Shop sales invoices}


\chapter{Manufacturing management}

\section{Producing sales orders}

\subsection{Work orders}


\chapter{Inventory management}

\section{Shipping}

\subsection{Pick lists}
\subsection{Packing list}

\section{Receiving}

\section{Partial shipments}

\section{Transferring between warehouses}

\section{Inventory reporting}



\chapter{Accounts receivable and payable}

\section{Creating generic AR/AP items}

\section{Handling refunds, overpayments and advances}

- this bit is about credit notes and debit notes

\section{Handling returns}

--> this bit is about credit (sales) and debit (vendor) invoices

\chapter{Credit risk management}

\section{Introduction}

A company runs credit risk when it gives credit: it runs the risk of the
creditor not paying off its debts.  LedgerSMB features two ways to manage
the risks involved:

\begin{enumerate}
\item Limit management
\item Arrears management
\end{enumerate}

The former tries to limit the risk involved by making sure no customer
receives more credit than a certain limit while the latter tries to
make sure any over due payments get cashed.

\section{Limit management}

Limit management should prevent a company on one hand from delivering too much
to its customers at once and on the other from taking (and delivering) new orders
to customers with too high amount of unpaid invoices.

@@@ Where to set up limits

@@@ How to monitor limits


\section{Managing arrears}

As mentioned before, the process of managing arrears is directed toward
detecting arrears positions with customers early and taking appropriate
action.


\subsection{Monitoring AR aging}

In order to find out about over due invoices, a company should run the AR
aging report available under AR $\rightarrow$ Reports $\rightarrow$ AR Aging.
The initial screen presents parameters for the aging report to be generated.

@@@ discuss parameters

This report shows customers and their outstanding invoices categorised as:

\begin{description}
\item [Current] Invoices not over due
\item [30] Invoice amounts over due by 30 days or less
\item [60] Invoice amounts over due by 60 days or less (but more than 30)
\item [90] Invoice amounts over due by 90 days or less (but more than 60)
\end{description}


@@@ Printing / mailing aging reports to customers


\subsection{Tracking invoice history}

After an invoice becomes over due a process will be started to remind
the customer of the outstanding amount requiring payment.

In order to keep records of actions taken to chase customer payment,
the invoice screen has an ``Internal Notes'' field which can be edited
after the invoice has been posted.

In order to save any edits to that field, hit the ``Save Info'' button.

\begin{quotation}
Note that the ``Save Info'' button also saves any changes to the ``TaxForm'' column or
rather, any information that's not accounting information (posted to the books and
thereby fixed) nor information which appears on the invoice - which also should remain
unedited in order to be able to generate an exact copy at a later date.
\end{quotation}


\subsection{Special treatment of invoices}

There may be good reasons to treat some over due invoices differently. E.g. in case
payment arrangements have been made with the customer and further standard arrears
management would not be appropriate any more.

In this case, you can put an invoice ``On Hold''. The opposite of being on hold is
being active. The AR Aging report allows selection of all invoices, only active
invoices (those not on hold) or only invoices on hold.


\section{Interest on arrears}

\section{Allowance for doubtful accounts}



\section{Writing off bad debt}

\subsection{Direct write-off}

\subsection{Allowed-for write-off}


\chapter{Receipts and payment processing}


\section{Single receipts and payments}
\label{sec:SinglePayments}

% The X column in the single payment interface deletes the checked invoice
% from the payment list on the next screen update.
% ### Shouldn't this be replaced by some nice JS/Ajax code which hides the row instead?

% the input box in the single-receipt/payment interface right after the Cash|Check|Deposit|Other
% plays an important role in the bank statement reconciliation. The drop-down selected
% selects a numbering range for real world documents which should be uniquely identifying
% documents. Reconciliation aggregates 

\section{Batch receipts and payments}


\section{Check payments}

% Cash -> Vouchers -> Payments, create batch, print checks from the AP transactions.

\section{Using overpayments}
\label{sec:UsingOverpayments}

\section{Receipt and payment reversal}

% Cash/Vouchers/Reverse Payment
% Batches/Approval

\section{Receipts and payments in foreign currencies}

\chapter{Accounting}

\section{Separation of duties: Transaction approval}
\label{sec:SeparationOfDuties}

\section{Bank reconciliation}
\label{sec:Reconciliation}



\section{Period closing}

Period closing is a concept used by accountants to ensure that audited and
known-correct accounting data stays correct by ``freezing'' it: by closing
an accounting period no modifications can be made to the accounting data
before a certain date.

LedgerSMB supports this concept through the System $\rightarrow$ Audit Control
menu, where you'll find the ``Close Books up to'' item. By filling out a date
in the input field and hitting Continue posting to dates before the entered
date will be disallowed.

% @@@ screenshot

\begin{quotation}
Note that due to a design limitation in LedgerSMB 1.3 - to be lifted with the
general AR/AP redesign - invoices in foreign currencies can't be reversed on
other dates than their original posting date. That is: they can, but their
reversal will result in P\&L and balance sheet effects which presumably isn't
desirable. Since period closing disables posting before a certain date this
functionality may have negative side effects in some set ups.

Also note that this pertains exclusively to invoices and transactions in foreign
currencies and has no effect in case of invoices and transactions in the default
currency.
\end{quotation}

\section{Year-end processing}
\label{sec:YearEndProcessing}

Year end closing is a concept which prepares the accounting books for the next
accounting year. Note that this is unrelated to the calendar year but to the
accounting year of the company instead. To muddy the waters even more: there's
no inherent requirement for this process to be run at least once a year. If the
first book year of the company spans more than a year, then this procedure will
be run more than a year after starting up the company.

This procedure freezes the accounting data in the year to be closed as described
in the previous section. Additionally it clears out the profit and loss accounts:
setting all the account
balances to zero by posting their balance to the retained earnings account. Some
businesses prefer to create a retained earnings account for each book year they
close. LedgerSMB supports that use-case by allowing the user to select which
retained earnings account the balance should be posted to.

Some companies want may to include additional transactions related to dividend
payment regarding the current year: reduce the equity by the amount paid as
dividends in transactions marked as ``year-end transaction''. Support
for this use case isn't available in LedgerSMB 1.3.

% @@@ screenshot

\section{Entering general accounting documents}

Even though the application handles many general ledger postings as consequences
from work flows elsewhere in the system - thus not requiring separate postings -
sometimes the need may occur to create manual postings not resulting from
AR or AP transactions or till and inventory adjustments.

One example of a case like that is the calculation, and posting of
corporate taxes presumably at the end of each accounting period but at least
at the end of the book year.

% @@@ screenshot

\section{Fixed asset accounting}
\label{sec:FixedAssetAccounting}


\section{Tax (VAT) reporting}

\begin{itemize}
\item 1099
\item EU VAT
\end{itemize}


\section{Reporting}
\subsection{Income statement}
\subsection{Balance sheet}
\subsection{Trial balance}


