% !TeX encoding = UTF-8
% !TeX spellcheck = en_US

\documentclass[10pt,a4paper]{book}

\usepackage{url, palatino, color}
\usepackage[colorlinks, linkcolor=black, urlcolor=black]{hyperref}
\usepackage{graphicx}
\usepackage{longtable}
\graphicspath{{images/}}

% glossaries must be loaded *after* hyperref
\usepackage[acronym]{glossaries}
\makeglossaries

\ifdefined\LaTeXML
  \newcommand{\figref}[1]{Figure~\ref{#1}}
  \newcommand{\tabref}[1]{Table~\ref{#1}}
  \newcommand{\secref}[1]{Section~\ref{#1}}
  \newcommand{\charef}[1]{Chapter~\ref{#1}}
  \newcommand{\appref}[1]{Appendix~\ref{#1}}
\else
  \newcommand{\figref}[1]{Figure~\ref{#1} on page~\pageref{#1}}
  \newcommand{\tabref}[1]{Table~\ref{#1} on page~\pageref{#1}}
  \newcommand{\secref}[1]{Section~\ref{#1} on page~\pageref{#1}}
  \newcommand{\charef}[1]{Chapter~\ref{#1} starting at page~\pageref{#1}}
  \newcommand{\appref}[1]{Appendix~\ref{#1} starting at page~\pageref{#1}}
\fi

\newcommand{\menupath}[1]{{\tt #1}}
\newcommand{\ma}{\ensuremath{\rightarrow}}


\begin{document}

\newacronym{ap}{AP}{Accounts payable}
\newacronym{ar}{AR}{Accounts receivable}
\newacronym{coa}{CoA}{chart of accounts}
\newacronym{erp}{ERP}{Enterprise Resource Planning system}
\newacronym{oh}{OH}{On hand}
\newacronym{plu}{PLU}{Product lookup unit}
\newacronym{rfq}{RFQ}{Request for quotation}
\newacronym{rop}{ROP}{Re-order point}
\newacronym{sku}{SKU}{Stocking unit}
% !TeX spellcheck = en_US

\newglossaryentry{credit limit}{
  name={credit limit},
  description={Maximum amount of open invoices and orders to be allowed a credit
     entity account. Used to limit credit risk.}}
\newglossaryentry{add-on}{
  name=add-on,
  description={Extension to the base LedgerSMB system to add new functionality},
  plural=add-ons}



\author{Erik H\"ulsmann, Chris Travers}
\title{\begin{Large}
Running your business with
\end{Large} \\
LedgerSMB 1.3 \\
 ~ \\
\texttt{DRAFT / WORK IN PROGRESS} }


\maketitle


\tableofcontents

\listoffigures

\listoftables

\printglossaries

\chapter*{Preface}
\label{preface}

\section*{Intended audience}
\label{sec-intended-audience}

The book has been split up in several parts intended for starting entrepreneurs,
(potential) users, application integration developers and
administrators, it offers something for everybody who works with or considers working with
LedgerSMB.

\section*{Organization of the book}
\label{sec-book-organization}

The first
part ('Overview') is intended for everybody who tries to get an impression of LedgerSMB, with
chapters on technology, features, licensing and a bit of project history.

The second part
('Getting started') leads new users and especially new entrepreneurs step by step through the
application. This part also contains additional explanations on good business and accounting
process.

The third part ('Configuration') describes system wide configuration settings such as
the LedgerSMB configuration file and those required for dependencies like PostgreSQL. Additionally
it describes the company specific settings within the application, such as audit control settings
and required special accounts. This part is required material for anybody who wants to set up
a LedgerSMB instance.

The fourth part ('Administration') discusses the different topics
regarding application administration such as user management, product definition, taxes and etc.
Anybody responsible for maintaining application instances in good health should read this part.

The fifth part discusses work flows; LedgerSMB best practices, so to say. This part is highly advised
reading for everybody who is a LedgerSMB user and should be considered required reading for 'business
architects': those in the company who decide about process design and execution.

The sixth part discusses how to customize the application. This part is intended for developers of
custom extensions (such as company specific integrations), add-ons and plug-ins.

The last part, the appendices, contain information on various subjects, such as a listing and description
of the authorization roles in the application as well as information on migration to or upgrading of
LedgerSMB versions.


\section*{Comments and questions}
\label{sec-comments}

The sources of this book are being developed in the ledgersmb-book Google Code project at
\url{https://code.google.com/p/ledgersmb-book/}. Comments and enhancement ideas can be filed
in the ticket tracker of the project. Additionally, discussion about the content of the book
can be organized on the IRC freenode.net channel \#ledgersmb or the ledger-smb-devel for which
you can subscribe at \url{https://lists.sourceforge.net/lists/listinfo/ledger-smb-devel}.



\section*{Acknowledgements}
\label{sec-acknowledgements}

H{\aa}vard S{\o}rli for his efforts to help out develop the book outline.

% !TeX encoding = UTF-8
% !TeX spellcheck = en_US
% !TeX root = ledgersmb-book.tex

\part{Overview}
\label{part-overview}
% An overview should contain:

% * why read?
%   * problems being solved
%   * goals to be achieved
%   *
% * how
%   * erp

\chapter{What is LedgerSMB}
\label{cha-what-is-ledgersmb}

\section{Introduction}
\label{sec-ledgersmb-introduction}
LedgerSMB is an open source, web based \gls{erp}. An \gls{ERP} is a system
which supports business processes of all disciplines throughout the
organization, automating as much of those processes as possible. To
illustrate this, consider the process of selling goods to a customer
in a trade company. Typically, a customer will request a quote, which
Sales will provide. When satisfied he or she will convert the quote
to an order. Sales will register the order, leading to order-pickers
to collect one or more shipments. Upon completion, Finance sends
out an invoice and records the customer's payment. LedgerSMB supports
this process by automating the conversion from quotation to order,
order to shipment and shipment to invoice, as well as providing
pack lists and other production related documents.

LedgerSMB includes a
powerful framework which supports building your own extensions and
integrations with other applications. Through this philosophy, it
aspires to be \emph{the} (open source) integrated administration system.

The software is being developed by the LedgerSMB project with its
homepage at \url{https://ledgersmb.org/}.
However, the actual project activity can be witnessed at the Github
project site hosted at \url{https://github.com/ledgersmb/LedgerSMB} and
on the mailing lists or chat channel 
listed at \url{https://ledgersmb.org/topics/support}

Its open source nature allows you to download LedgerSMB and use it with any
infrastructure you like. So there's no vendor lock-in: you can
always take your data and set up your system with another hardware vendor
or set up your own hardware.

\section{Supported functionality}
\label{sec-ledgersmb-modules}
As most \glspl{erp} LedgerSMB's functionalities are grouped into modules.
Many modules are integrated parts of the base application.  New features
can be implemented in separate modules at first to allow evaluation of the
feature set.  When a it has become sufficiently stable, the new module may
be integrated in the base application. As of that time, the existing features
will be frozen, meaning that the utmost will be done to prevent it from being
changed.

These separate modules - which are called \glspl{add-on} - have to be
installed separately.  After installation they become seamless parts of
LedgerSMB with no visible difference from the base application. An
additional benefit of having \glspl{add-on} is that they can follow different
release schedules and separate maturity levels than the main application.


LedgerSMB \ledgerSMBversion features the following integrated modules:

\begin{itemize}
        \item General ledger
        \item Payment and Accounts payable
        \item Invoicing and Accounts receivable
        \item Fixed asset accounting
        \item Time registration and invoicing
        \item Point of Sale
        \item Quotation and Order management
        \item Manufacturing
        \item Inventory (warehousing) and shipping management
        \item VAT reporting
        \item Controlling
        \begin{itemize}
                \item Project accounting
                \item Department accounting
        \end{itemize}
        \item Application administration
\end{itemize}

There are no known \glspl{add-on} available at this time.

With this list of modules and \glspl{add-on} LedgerSMB has been successfully
implemented in a wide range of companies of varying types and sizes: shops,
manufacturing companies and service oriented businesses up to as big as @@TODO million transactions, growing by 50,000 transactions (spread across \acrshort{AR}, \acrshort{AP} and \acrshort{GL}) \emph{per month}.

\section{Feature comparison with alternatives}
\label{sec-ledgersmb-feature-comparison}
@@@ TODO

Packages to compare to:

\begin{itemize}
        \item GNUcash
        \item OSfinancials
        \item ERP5
        \item Odoo
        \item xTuple
\end{itemize}



\section{System requirements}
\label{sec-ledgersmb-system-requirements}

The \href{https://github.com/ledgersmb/LedgerSMB/blob/1.10/README.md#system-requirements}{README file}
which comes with every LedgerSMB software release should be
considered the canonical source for system requirements. In summary, the
following technical components are required:

\begin{itemize}
        \item A web server (known working are Nginx, Apache, lighttpd and Varnish)
        \item Modern Perl 5, with \href{https://github.com/ledgersmb/LedgerSMB/blob/1.10/cpanfile}{additional modules}
        \item PostgreSQL 13 or higher
        \item (optional) \LaTeX{} or \XeLaTeX{} from the Te\TeX{} or \TeX{}Live \TeX{} distributions
\end{itemize}

More information about system requirements for clients and servers is available at
\href{https://ledgersmb.org/content/system-requirements}{https://ledgersmb.org/content/system-requirements}.

System requirements such as required RAM and number of CPUs and their speed
largely depend on the expected system activity. However, any modern \acrshort{VPS} should provide enough memory and storage to satisfy a reasonable number of users.

\section{Application architecture}
\label{sec-ledgersmb-architecture}

Due to its heritage from \href{https://sql-ledger.com/}{SQL Ledger} and the on-going process of rewriting
the inherited code, the architecture differs between parts
of the application: the old parts and the ones which have already been
rewritten.

Overall, the application consists of five layers:

\begin{itemize}
\item The web browser (as the user interface)
\item The web server (as a network traffic handler)
\item The Perl (web)server process (as an application - \gls{PSGI})
\item The database (as an application - \gls{PL/SQL})
\item The database (as storage)
\end{itemize}

In version 1.10 the user interface is a \gls{SPA} using a mixture of new (e.g. 
\href{https://vuejs.org/}{Vue 3},
\href{https://webpack.js.org/}{webpack 5} and \href{https://developer.mozilla.org/en-US/docs/Web/Web_Components}{Web Components}) and old (e.g. \href{https://dojotoolkit.org/}{Dojo Toolkit}) technology.

The web server is an optional component that's highly recommended for TLS termination
and protecting the back end Perl HTTP server from malicious traffic\footnote{This setup is an industry-wide accepted best practice for deploying web applications; web servers like Apache or Nginx have been and continue to be extensively security-reviewed. No other implementations can achieve the same level of scrutiny.}. It may also be used
for serving static page components such as images, style sheets and JavaScript component as an efficiency measure.

Recent releases add separation between business logic and process state with the
latter being stored in the database and managed at the Perl layer.  Other than that, it's the Perl layer's responsibility to forward web requests to the database and
presenting the resulting data in response to the user interface.
The original goal to reduce the Perl layer to be a ``glue'' layer between the web
server and the database has been abandoned with  the introduction of this additional
(business process) layer.

At the ``database as an application'' layer, \gls{PL/SQL} functions implement business
functionality, such as creating \gls{COGS} accounting entries.

The ``database as storage'' layer is responsible for storing data with consistency and
integrity; to that extent (and much more than in \acrshort{SL}) constraints and triggers
have been implemented.  An additional role for the storage layer is to enforce data
access rules; i.e. to protect data from being accessed by unauthorized users.

The main difference between old and new coding paradigms is at the Perl application
layer: older code generates HTML fragments while newer code delivers data through
web services.  As part of the on-going code restructuring, there's a major effort
to create a \gls{REST} web service based APIs.  The intent here is to facilitate
integration with other applications used by businesses (both in the business itself
or as provided by customers and vendors).






\section{License}
\label{sec-ledgersmb-license}

LedgerSMB is available under the terms of the
\textit{GNU Public License version 2}, in short: \textit{GPLv2}
\footnote{\url{https://opensource.org/license/gpl-2-0/}}.

The project attaches the following meaning to this license:
The copyright holders grant you the right to copy and
redistribute the software.  In case you make any modifications to the software
you're obligated to publish the changes \textit{if you distribute the software}.  You are always free to provide third party access to
the \acrshort{API} from your modified software without being required to disclose your changes.

The project considers the \acrshort{API} to include:
\begin{itemize}
\item Database tables
\item URLs with their input and output
\item Webservices of any kind
\item Function and object calls
\end{itemize}

The effect of this interpretation is that changes directly to the code base as well as inheritance of classes defined in the software constitute ``making modifications''.

\chapter{Reasons to use LedgerSMB}
\label{cha-advocacy}

Jack finds several tools which suit his requirements to some extent or another.
After evaluation of his options he decides to use LedgerSMB for the following reasons:

\begin{itemize}
\item Centralized data storage
\item Actively developed
\item Development team with security focus
\item Access to the application requires only a web browser
\item Integrated sales, shipping, invoicing, purchasing and accounting
\item Open source solution, so no vendor lock in
\item The roadmap appeals to him, because it has web services payrolling on it
\item There are multiple vendors offering commercial support, including hosted options
\item The developers envision building a platform out of it: creating the building blocks
required to build a company on
\end{itemize}


\begin{itemize}
\item Own your own data
\item Freedom to change
        \begin{itemize}
        \item Support organization
        \item Developer (organization)
        \item Data storage provider
        \item Application service provider
        \end{itemize}
\end{itemize}

\section{Internal control}
\label{sec-advocacy-internal-control}

Internal control\footnote{See also \url{https://en.wikipedia.org/wiki/Internal_control}}
helps organizations to prevent and detect fraud by introducing checks and balances
to assess effectiveness and validity of transactions in the organization and thereby
in its \gls{erp} and accounting system(s).

\section{Accounting principles}
\label{sec-system-accounting-principles}

The accounting guidelines \gls{ias} and \gls{ifrs} describe requirements
to financial statements (reports) and the underlying accounting process
\footnote{See also \url{https://en.wikipedia.org/wiki/International_Financial_Reporting_Standards}}.
Said requirements include qualitative characteristics:

\begin{enumerate}
\item Relevance
\item Faithful representation
\item Comparability
\item Verifiability
\item Timeliness
\item Understandability
\end{enumerate}

While some - if not most - of these characteristics relate to the process of accounting,
the ``Verifiability'' item clearly has an impact on the underlaying accounting systems:
In order to be verifiable, there must be a clear audit trail to show the origin of the
figures. To make sure users leave behind the required audit trail, some actions can't
be performed in LedgerSMB, even though it would seem to be a logical requirement to be
able to do so - from the perspective of a non-accountant.


\section{Impact of tight integration}
\label{sec-system-impact-of-tight-integration}

While both the qualitative characteristics from \gls{ifrs} and the checks and balances
from the internal controls are pose restrictions on the accounting process,
sometimes these restrictions require support from the underlying accounting
software.

One example is the support for creating reliable audit trails
by protecting accounting data from deletion. It's important to realize the scope
of accounting data in this respect: because invoices are being registered in the
accounts receivable administration - which is summarized in the general ledger -
they are part of the data for which the audit trail needs to be recorded.

Another example is separation of duties (also known as the ``four eye principle''),
where one accountant enters financial transactions and another is responsible for
posting them. This procedure protects the company from an accountant single-handedly
faking transactions and possibly masking fraud.

The requirements for good accounting processes and internal control have impact
on the work flows supported by LedgerSMB. As a consequence some of the work flows
described in part \ref{part-workflows} may seem unwieldy; an example being the
lack of functionality to delete or correct incorrect invoices (See \secref{sec-workflows-invoicing-correction-or-deletion} for more details).



\chapter{Introduction to accounting}
\label{cha-accounting-introduction}

The purpose of an accounting system is to keep track of the company's financial status.  The following reports are used to
present it:

\begin{description}[style=nextline]
        \item[Balance Sheet] Provides a snapshot of the financial position of a company, listing its possessions (assets), debts (liabilities) and residual value (equity).
        \item[Income Statement] Also known as the ``Profit \& Loss Statement'', summarizes the income generated over a specific period and the expenses associated with it, resulting in ``Net Result''.
        \item[Cashflow Statement\footnote{Not currently implemented in LedgerSMB}] Summarizes the incoming and outgoing cash flows over a specific period, resulting in ``Free Cash Flow''.
          %%@@@TODO The footnote on the following line is a duplicate; using \footref didn't work with latexml
        \item[Statement of Owner's Equity\footnote{Not currently implemented in LedgerSMB}] Summarizes the changes in equity over a specific period.
        \item[Trial Balance] An intermediate report used for preparing the Balance Sheet and Income Statement, tallies transactions over the reporting period; used to assert that all accounts in the General Journal are balanced.
\end{description}

The accounting system is composed of these parts:
\begin{description}[style=nextline]
        \item [Journals] Contain the company's transactions with extensive additional data; they are the first point of entry for transactions, ordered by order of entry.
        \item [Ledgers] Contain aggregated data from the journals, ordered by date.
        \item [Chart of Accounts] The categories by which financial data are classified.
\end{description}

All accounting data ends up summarized in the \acrshort{GL}. Transactions may be entered in a different (sub)ledger before ending up in the \acrshort{GL}; e.g. the Sales Ledger. With the advent of computerized accounting, the use of journals is in decline: transactions are classified while directly being entered into ledgers.

In addition to the parts mentioned above the following terms are used throughout this document:
\begin{description}[style=nextline]
        \item [Assets] Money and anything that can be converted into money without reducing
        the net equity of the business. Assets include money owed, money held, goods
        available for sale, property, etc.
        \item [Liabilities] Debts owned by the business such as bank loans and unpaid bills.
        \item [Equity] What would be left for the owner if all the assets were converted to
        money and all the liabilities paid off.
        \item [Revenue] Income from business activities.
        \item [Expense] Money paid to operate the business.
        \item [Cost of Goods] Money that was spent to acquire material and services to build a product being sold.
        \item [Operating Expenses] Expenses that are consumed to administer the business.
        \item [Accounts Receivable] An asset on the books as a claim for a future payment
        from a customer.
        \item [Accounts Payable] A liability on the books for a future payment to a supplier
        or vendor.
\end{description}

\section{Double-entry accounting}
\label{sec-accounting-double-entry}

The basic concept in double-entry accounting is that every transaction is an exchange. For example, when a business
sells goods to a customer, it issues an invoice.
The exchange is to provide the goods and receive the right for (future) payment. When the customer later pays the invoice, the right for payment is exchanged for cash.

Each exchange is to have the same value on both sides of the exchange, making sure no transaction is incomplete. This is where
the requirement comes from that double-entry transactions need
to be balanced.  It can occur that a customer's payment is a bit short of the owed amount.  Should the company decide not to pursue payment, the transaction would end up being unbalanced. After all the exchange is not equal-value between the providing and receiving sides. Double-entry accounting accommodates this scenario by explicitly recording the unpaid amount as an expense.

Because every transaction is balanced, so is the balance sheet. Through this approach, double-entry accounting provides strong
control over correctness of the numbers: as soon as a transaction
is unbalanced (i.e. contains an error), the balance sheet becomes unbalanced -- a relatively simple check.

\section{Cash versus accrual basis}
\label{sec-accounting-cash-vs-accrual}

Financial statements, such as the Income Statement and Balance Sheet can be prepared using either a cash or accrual basis.

In \gls{cash basis} accounting, the income is deemed earned when the business physically
receives the customer payment, and the expenses are deemed incurred when the business
physically pays them. Cash basis accounting does not require the use of purchase orders,
invoices, or long term liabilities.

In \gls{accrual basis} accounting, income is considered earned when a valid asset is
received for services or product provided. The asset is the claim on the customer,
by way of the invoice, to collect an amount at a later date. This asset is called
Accounts Receivable. An expenses is considered earned when a liability is created,
by way of a purchase order, to the supplier or vendor. The liability is the commitment
to pay the supplier or vendor at a later date. This liability is called Accounts Payable.
Accrual basis accounting requires the use of invoices and purchase orders.

There can be a number of problems with \gls{cash basis} accounting:
\begin{itemize}
        \item No visibility in the accounting system regarding your cash commitments leading
        you to think you have more or less money to spend than you actually have.
        \item No visibility in the accounting system about unpaid customer debts.
        \item Your taxing jurisdiction may limit businesses that can use cash basis accounting.
        For example in the United States, if you sell products or services on credit,
        have gross receipts higher that allowed, or need inventory to account for income.
        \item  Because cash basis accounting isn't part of any accounting standards, there are varying expectations of what a cash balance or income statement looks like.
\end{itemize}

\section{Valuation of inventory}
\label{sec-accounting-valuation-inventory}

@@@ TODO Need why is this important paragraph?

The cost of inventory, in theory, includes all costs incurred to acquire the goods and make 
them ready for sale. This theoretical cost may include shipping costs, discounts, insurance, 
receiving costs, handling costs, storage costs, etc.

In practice, the cost used is often limited to the total invoice price for the goods.
This formula may or may not include shipping and handling.
Other costs are often ignored if they are immaterial to the overall cost of the inventory, if there
is no easy way to allocate the costs to the inventory, or they are relatively constant period to period.

Inventory valuation must consider in the following situations:
\begin{itemize}
        \item When the valuation of inventory varies from purchase to purchase. The costing method
        determines the valuation.  This situation is automatically handled in LedgerSMB using one of the
        inventory valuation methods discussed below.
        \item When the valuation of inventory is less that what was paid for it. The inventory must
        be written down. This is usually a manual calculation that is the result of damage, obsolescence, 
        or decline in the vendor or suppliers selling price.
        \item When estimates are required, such as when inventory is stolen, destroyed, or when a physical inventory
        cannot be performed. In this case, a reasonable and consistent manual method must be used to 
        estimate the value. 
\end{itemize}

There are several general automated inventory valuation methods including the following:
\begin{itemize}
        \item Specific Identification -- Each specific inventory item has its cost tracked individually. 
        This method is usually used for large and expensive items that can be tracked by 
        serial number or identification tag. 
        Typically, the specific identification inventory valuation method results an inventory valuation
        close to value of using the \gls{FIFO} method.
        \item \gls{FIFO} -- This method is similar to selling the oldest product first merchandising policy.
        In the \gls{FIFO} method the lastest costs are included in inventory cost and the older costs
        are charged back to \gls{COGS}.

        \gls{FIFO} is the preferred method for maintaining accurate historical costs and is less 
        susceptible to income manipulation by the timing of new purchases. Typically \gls{FIFO}
        results in the highest inventory value of these methods.
        
        % https://github.com/ledgersmb/LedgerSMB/blob/master/sql/modules/COGS.sql
        In LedgerSMB the inventory cost is tracked on a \gls{FIFO} basis. When a part is purchased, 
        its cost is added to the inventory asset account. When the part is sold, the cost of the item
        is moved to the cost of goods sold account.\\
        @@@TODO Is this paragraph redundant?\\
        @@@TODO  What sold cost is used; oldest, newest, average?\\
        @@@TODO How is the inventory asset account debited?
        \item \gls{LIFO} --  Under \gls{LIFO} the costs of the last goods purchased are charged
        against revenues as the \gls{COGS} and the inventory is composed of the costs
        of the oldest goods acquired.
        
        \gls{LIFO} is preferred when prices are rising as it typically results in reducing net income and
        thereby reducing taxes. However, \gls{LIFO} allows manipulation of income by simply changing the time
        at which additional purchases are made and does not represent accurate historical costs. 
        Typically \gls{LIFO} results in the lowest inventory value.
        
        \item Weighted Average -- Under the weighted average method, the total number of units 
        purchased plus those on-hand at the beginning of the period is divided by the total costs
        of purchases plus the cost of the beginning inventory.
        
        Typically the weighted average inventory valuation method results in 
        an inventory value between \gls{FIFO}, which is the highest,
        and \gls{LIFO} which is the lowest.
\end{itemize}

 



\part{Getting started}
\label{part:GettingStarted}

\chapter{Creating a company administration}

\chapter{The first login}

\chapter{Building up stock}

\chapter{Ramping up to the first sale}

% sending out a quote followed by a sales order

\chapter{Shipping sales}

\chapter{Invoicing}

\chapter{Collecting sales invoice payments}

\section{Customer payments}

\section{Customer payment mismatch}

% choosing between pardonning and registering underpayment

% large ones, as in partial payments or largish under/over payments

% pardonning small mismatches


\chapter{Paying vendor invoices}

% handling vendors who match amounts to exact invoices

% handling vendors with running balances

% handling bounced checks: voiding checks to undo payments of vendor invoices
%   relating to bounced checks

\chapter{Monitoring arrears}

% handling interest on arrears

\chapter{Handling sales taxes}

% invoices with taxes included

% invoices with explicit tax amounts

% 

\chapter{Branching out: services}

% including creation / assignment to different accounts


\chapter{Recording service hours}

\chapter{Customer approval on service hours}

\chapter{Invoicing services}

\chapter{Branching out II: service subscriptions}



% !TeX encoding = UTF-8
% !TeX spellcheck = en_US
% !TeX root = ledgersmb-book.tex

\part{Configuration}
\label{part-configuration}


\chapter{Overview}
\label{cha-configuration-overview}

\section{Introduction}
\label{sec-config-overview-introduction}
This section of the book describes how to set up LedgerSMB and its components.
Configuration \index{configuration} is assumed to be mostly one-off and rather technical in nature.  To find
out which tasks might need to be performed in order to keep the application in good
health the reader is referred to the 'Administration Introduction', \secref{sec-administration-introduction}.

\chapter{Global configuration}
\label{cha-global-configuration}

\section{Apache}
\label{sec-global-config-apache}

Section about installing on \index{apache} Apache 2+

items to be discussed:

\begin{description}[style=nextline]
\item [Forwarding of authentication] @@@TODO
\item [PSGI configuration] @@@TODO
\item [performance] cgiD configuration: don't (yet) [but will be supported once all legacy code is gone] @@@TODO
\item [security] suEXEC environment @@@TODO
\end{description}

\subsection{Differences between Apache 1.3 and 2+}
\label{subsec-global-config-apache-13-vs-2}

Explain how to use lsmb with 1.3 instead of 2+.

\section{PostgreSQL}
\label{sec-global-config-postgresql}

\begin{description}[style=nextline]
\item [pg\_hba.conf] authentication @@@TODO \index{pg\_hba.conf}
\item [security] local vs IP connections @@@TODO 
\index{Postgres} \index{Postgres Security}
\end{description}


\section{LedgerSMB version numbers}
\label{sec-global-config-ledgersmb-version-numbers}

LedgerSMB version numbers \index{version number} are in the form of Major.Minor.Patch-Optional Tag. So for example, the current 1.9 development release is \texttt{1.9.28-dev} and the production release is \texttt{1.9.27}

\begin{description}[style=nextline]
\item [Major] An increment in the major release number \index{release number} can mean significant architectural, \gls{API}, functional, or usage changes. It is expected that an upgrade to a new Major version is  going to be planned and tested by the user organization before upgrading.
\item [Minor] An increment in the minor release number may indicate changes to \glspl{API} \index{API}, interfaces, user instructions, etc.  But these changes are not expected to have major impact and should require minor planning and testing. Often small impact enhancements are back-patched to previous Minor versions as patches.
\item [Patch] Increments in the Patch number represent bug fixes, security improvements, or enhancements that do not break existing functionality, usage, or \glspl{API}.  These changes are expected to be applied to an installation as soon as possible.
\item [Optional Tag] Common values include dev, beta, or alpha. Typically, these are only used internally by the LedgerSMB developers.
\end{description}

\section{LedgerSMB Configuration}
\label{sec-global-config-ledgersmb}

LedgerSMB configuration using \texttt{ledgersmb.conf} \index{ledgersmb.conf} is deprecated as of 1 Jan 2023. New functionality may only available when using \texttt{ledgersmb.yaml} \index{ledgersmb.yaml} configuration file.

For the time being there is a conversion step that converts the old 'ledgersmb.conf' to `ledgersmb.yaml`, but the old conf file does not support new functionality.

\subsection{ledgersmb.yaml}
\label{subsec-global-config-ledgersmb-yaml}

For an example of the default, non debug \texttt{ledgersmb.yaml} \index{ledgersmb.yaml} see  \url{https://github.com/ledgersmb/LedgerSMB/blob/master/doc/conf/ledgersmb.yaml}

\subsubsection{\texttt{cookie}}
 @@@TODO

\subsubsection{\texttt{db}}
 @@@TODO

\subsubsection{\texttt{default\_locale}}
@@@TODO

\subsubsection{\texttt{environment\_variables}}
 @@@TODO

\subsubsection{\texttt{extra\_middleware}}
@@@TODO

\subsubsection{\texttt{logging}}
@@@TODO

\subsubsection{\texttt{login\_settings}}
@@@TODO

\subsubsection{\texttt{mail}}

Email \index{mail} can be configured by selecting which of the three available transports to use. The default \texttt{ledgersmb.yaml} file contains examples for the first two.

\begin{description}
    
    \item{\texttt{Email::Sender::Transport::Sendmail}} – Emails are sent using the local server's \texttt{sendmail} \index{sendmail} binary. The configuration parameters are:
    \begin{description}
        \item{\texttt{transport:\$class}} – \texttt{Email::Sender::Transport::Sendmail}
        \item{\texttt{transport:path}} – optionally provide a path to the directory that contains the 'sendmail' binary.
    \end{description}
    
    \item{\texttt{LedgerSMB::Mailer::TransportSMTP}} - Emails \index{email} are sent using a remote SMTP \index{SMTP} server.  The configuration parameters are:
    \begin{description}
        \item{\texttt{transport:\$class}} – \texttt{LedgerSMB::Mailer::TransportSMTP}
        \item {\texttt{transport:host}} – The required host name of the smtp \index{ SMTP} server.
        \item {\texttt{transport:port}} – The required port number of the smtp server. Note this might vary depending on whether TLS or SSL is used.
        \item {\texttt{sasl\_username:\$class}} – The required smtp server authentication method. The values can be `Authen::SASL` or `Authen::SASL::SCRAM`.
        \item {\texttt{sasl\_username:mechanism}} –  The available mechanism are defined at \url{https://metacpan.org/dist/Authen-SASL} or \url{ https://metacpan.org/dist/Authen-SASL-SCRAM} depending on the selected `\$class`.
        \item {\texttt{sasl\_username:callback:user}} – The required SMTP user name.   'the-user' in the default file is a place holder and must be replaced.
        \item {\texttt{sasl\_username:callback:pass}} – The required SMTP password.  'SECURITY-FIRST' in the default file is a place holder and must be replaced.
    \end{description}
    
    \item{\texttt{Email::Sender::Transport::DevNull}} -  Emails are sent to \texttt{/dev/null}, in other words emails are not sent anyplace. This prevents errors in the user interface, but throws away any mail.  There is only one configuration parameter.  This is not a recommended production configuration. It is usually used for testing.
    \begin{description}
        \item{\texttt{transport:\$class}} – \texttt{Email::Sender::Transport::DevNull}
    \end{description}

\end{description}

\subsubsection{\texttt{miscellaneous}}
@@@TODO

\subsubsection{\texttt{output\_formatter}}
@@@TODO  I really don't have a good way to format lists explanations.  See below at \texttt{paths:config:workflows} for an example. Guidance welcome.

\subsubsection{\texttt{paths}}

This section configures the various paths used by LedgerSMB. The configuration parameters are:

\begin{description}
    \item {\texttt{config:locale}} – Path to locale \index{locale path} files. Defaults to \texttt{locale/po}.
    \item {\texttt{config:sql}} – Path to the SQL schema \index{SQL schema path} definition files. 
    \item {\texttt{config:sql\_data}} – Path to the reference and initial SQL database load \index{database load path} files. For example, the names of the countries.
    \item {\texttt{config:templates}} – Path to the templates \index{template path} base directory. Typically set to \texttt{templates}.
    \item {\texttt{config:UI}} – Path to the UI HTML \index{HTML path} files. Defaults to \texttt{./UI/}
    \item {\texttt{config:UI\_cache}} – Path to the location of the UI template cache \index{template cache path}. These are cached after they have been parsed and translated. This improves performance.
    \item {\texttt{config:workflows}} – A list of the Directories where workflow files \index{workflow path} are stored. Contains the default and custom workflows.  Custom workflows are used to override behavior of the default workflows by providing actions, conditions, etc. with the same name and type or by providing workflows of the same type with additional states and actions. Default workflows defaults to \texttt{workflows}. Custom workflows defaults to \texttt{custom\_workflows}. Default workflow path must precede all the custom workflow paths. @@@TODO is the last statement correct?  Can there be more than 2 paths?
\end{description}

\subsubsection{\texttt{printers}}
This section contains a list of printers \index{printing} and their definition.

The default  \texttt{ledgersmb.yaml} file shows two printers \index{printer configuration} named 'Laser' and 'Epson' with the printers defined using the linux \texttt{lpr} command and its arguments. 

This default definition will provide for the selection of the printers named  'Laser' and 'Epson' in the LedgerSMB user interface.

For more information search the internet for 'linux lpr command' or use \texttt{man lpr} at the linux command line.

\subsubsection{\texttt{reconciliation\_importer}}
 @@@TODO

\subsubsection{\texttt{setup\_settings}}
@@@TODO

\subsubsection{\texttt{ui}}
@@@TODO

\subsubsection{\texttt{workflows}}
@@@TODO

\chapter{Per company configuration}
\label{cha-company-config}

\section{Matching your business processes}
\label{sec-company-config-matching-your-business}

By default, LedgerSMB operates such that all optional functionality is available and the user decides to use it or not based on which menu items they select, what fields they enter data into, and what buttons they click.

Removing application roles \index{application roles} (see \appref{app-role-listing}) can limit the visibility of  menu items, data fields, and buttons. This will simplify the users view of the system, reduce training, and better configure LedgerSMB to your business requirements.

Outside of application roles, the only other enforced business process configuration is whether the same person can both create and post transactions. It is called 'separation of duties' \index{separation of duties} and is defined in \secref{subsubsec-company-config-defaults-separation-of-duties}.

LedgerSMB is configured to adjust inventory \index{inventory} when a Sales or Purchase Invoice is posted. This closely matches retail business processes where the customer walks out of the establishment with the product and an invoice \index{invoice} (or receipt).  This functionality can also be used for wholesale shipping applications because LedgerSMB can also produce picking \index{picking} and \index{shipping} shipping documents, just remember that the inventory transaction happens when an invoice is posted.

LedgerSMB uses \gls{FIFO} \index{FIFO} for \gls{COGS} \index{COGS} and inventory calculations.  There are provisions for alternatives, but the code is not yet complete. See \secref{sec-accounting-valuation-inventory} for a detailed explanation of  \gls{FIFO} calculations.

In addition to the above, LedgerSMB has configurable Workflows \index{workflows}. These are not user configurable but your technical support staff should be able to create and edit them. See \charef{part-workflows}.

Contact the \href{https://ledgersmb.org}{development team} for other business process customizations.  In most cases the need for more flexible matching of LedgerSMB to your business processes is understood, but the project has not not yet had any customer requests or development volunteers.

\section{Administrative user}
\label{sec-company-config-admin-user}

\section{Chart of accounts}
\label{sec-company-config-coa}

@@@ Should refer to the 'administration' section???

\subsection{Special accounts}
\label{subsec-company-config-coa-special-accounts}

\begin{itemize}
\item AR/AP summary accounts
\item 5 other special purpose accounts, see ``Defaults'' screen discussion
\item sales tax accounts
\end{itemize}


\section{System menu settings}
\label{sec-company-config-system-menu}

This section enumerates the ``System'' menu's immediate children. In some cases the
functionality is too complex and is referred to a chapter of its own.

\subsection{Audit control}
\label{subsec-company-config-audit-control}

\subsubsection{Enforce transaction reversal for all dates}
\label{subsubsec-company-config-audit-control-reversals}


This is a Yes/No value which affects the actions which can be performed on posted financial transactions.
\begin{itemize}
\item No means transactions can be altered or deleted, even after posting them. Note that
if a transaction has been posted before the latest closing date, it can never be altered,
not even when this value is in effect.
\item Yes means transactions can't be altered after posting. This setting is highly preferred and considered the only correct approach to accounting as it assures visible
audit trails and thereby supports fraud detection.
\end{itemize}

\subsubsection{Close books up to}
\label{subsubsec-company-config-audit-control-close-books}


@@@ This item isn't a system setting; shouldn't it move to ``Transaction approval''?? That way system settings (config) and processes are separated.

@@@ My preference is to remove the setting entirely and rely on year-end 
workflow.  We might add an account checkpoint interface as well at some point
--Chris T

It's advisable to regularly close the books after review. This prevents user error changing
reviewed numbers: after closing the books, it's no longer possible to post in the closed
period.

There are also performance benefits to closing the books, because LedgerSMB uses the
fact that the figures are known-stable as a performance optimization when calculating
account balances.

\subsubsection{Activate audit trail}
\label{subsubsec-company-config-audit-control-audit-trail}

This is a Yes/No value which - when Yes - causes the system to install triggers to register
user actions (creation/adjustments/reversals/etc...) executed on financial transactions.


@@@ Once activated, where can we see it the audit trail??

@@@ This setting should go.  In 1.3 the audit trails are always enforced via
triggers so this setting does nothing.  --CT

\subsection{Taxes}
\label{subsec-company-config-taxes}


This page lists all accounts which have the ``Tax'' account option enabled as discussed in \secref{sec-coa-account-options}.

Each account is listed at least once, but can be listed many times, if it has had different
settings applied over different time periods. E.g. if one of the current VAT rates is 19\%,
today but it used to be 17.5\% until last month, there will be 2 rows for the applicable
VAT account. See \charef{cha-taxes} for further discussion of how taxes work in
LedgerSMB and the choices involved when being required to handle changes in tax rates.

Each row lists the following fields:

\begin{description}[style=nextline]
\item [Rate (\%)] The tax rate to be applied when calculating VAT to be posted on this account.
\item [Number] Account number
\item [Valid To] The ending date of the settings in this row. This can apply to the rate as well as the ordering or the tax rules (but usually applies to the rate).
\item [Ordering] This has to do with cumulative taxes.  For example if two taxes
exist and one has an ordering of 0 and one of 1, then the second tax will be
calculated on a basis that includes the first.  One place where this used to be
used was in Quebec, where GST was taxable under PST.
\item [Tax rules] LedgerSMB features a flexible structure to facilitate complex tax
calculations (see \secref{sec-tax-rule-plugins}). By default the ``Simple'' module
is the only one installed.
\end{description}

\subsection{Defaults}
\label{subsec-company-config-defaults}

\subsubsection{Business number}
\label{subsubsec-company-config-defaults-business-number}
   This is used to store an arbitrary identification number for the business.  It
could be used to store a business license number or anything similar.
   
\subsubsection{Weight unit}
\label{subsubsec-company-config-defaults-weight-unit}
   The unit of measurement for weights. @@@ why don't we have a unit of measurement for distance as well??? And maybe a unit of measurement for content?
   
\subsubsection{Separation of duties}
\label{subsubsec-company-config-defaults-separation-of-duties}

% For better or worse this is the spot for the canonical definition of separation of duties.

Separation of duties \index{separation of duties} is a method to help reduce fraud where one employee can't modify the
accounting ledger without another employee's approval.

Select "Yes" if you want to activate separation of duties or "No" if you don't
want to activate it.

In order for separation of duties to be enforced, user roles have to be set differently for each user. This is done by removing the \texttt{draft\_post} role from the users that cannot post and making sure that the users that can post have the role enabled.  See \secref{sec-user-management-authorization} for more details about changing and setting User Roles.

\subsubsection{Default accounts}
\label{subsubsec-company-config-defaults-accounts}

This setting will be used to preselect an account in
the listings of the three categories listed below:
\begin{itemize}
\item Inventory
\item Income
\item Expense
\end{itemize}


\subsubsection{Foreign exchange gain and loss accounts}
\label{subsubsec-company-config-defaults-fx-accounts}

When working with foreign currencies,
the system needs two special purpose accounts. One to post the gains onto which are
caused by foreign currencies increasing in value; the other to post the losses onto
which are caused by foreign currencies decreasing in value.


\subsubsection{Default country}
\label{subsubsec-company-config-defaults-country}

This setting indicates which country needs to be pre-selected
   in country selection lists.


\subsubsection{Default language}
\label{subsubsec-company-config-defaults-language}

The language to be used when no other language has been selected. Several parts of the
application require language selection, such as customer, vendor and employee entry screens.

\subsubsection{Templates directory}
\label{subsubsec-company-config-defaults-templates}

This setting indicates which set of templates - stored in the
   \texttt{templates/} directory - should be used. In a standard installation, the drop down
   lists two items:
\begin{description}[style=nextline]
   \item [demo] which contains templates based on \LaTeX, which is more commonly installed but has issues dealing with accented characters
   \end{description}


\subsubsection{List of currencies \& default currency}
\label{subsubsec-company-config-defaults-currencies}

Enter a list of all currencies you want
to use in your company, identified by their 3-letter codes separated by a colon; i.e.
``USD:EUR:CHF''. To ensure correct operation of the application, at least one currency
(the company default currency) must be listed. In case of multiple currencies the first
is used as the company default currency.

\subsubsection{Company data (name /address)}
\label{subsubsec-company-config-defaults-name-address}

The fields ``Company Name'', ``Company Address'',
``Company Phone'' and ``Company Fax'' will be used on printed/e-mailed invoices.

\subsubsection{Password duration}
\label{subsubsec-company-config-defaults-password-duration}

This is an integer value field measuring the validity period in days for passwords set through
the user's \texttt{Preferences} screen. If this field is empty, passwords set through that method
won't expire.

A user will receive password expiration reminders upon logging starting a week before password
expiry. When not acted upon, starting two days before expiry an hourly popup will appear
requesting the user to change the password.

The application behaves this way because users with expired passwords won't be able to log in:
their password will need to be reset by a user admin.

\begin{quote}
Note that passwords set by admins for other users expire within 24 hours after setting them.
This value is hard coded and can't be overruled. This is a security measure taken to make
sure as few unused accounts as possible exist: Existence of such accounts could open up security
holes.
\end{quote}


\subsubsection{Default E-mail addresses}
\label{subsubsec-company-config-defaults-email}

These addresses will be used to send e-mails \index{email} from the system.
Note that the ``Default Email From'' address should be configured in order to make sure
e-mail doesn't look like it's coming from your webserver. The format to be used is \texttt{``Name'' <e-mail address>} where the e-mail address should be inserted between the
``$\langle$'' and ``$\rangle$''.

\subsubsection{Max per dropdown}
\label{subsubsec-company-config-defaults-max-dropdown}

Some elements in the screens may present a drop down. However, drop downs are
relatively unwieldy to work with when used to present a large number of values
to choose from.

This configuration option sets an upper limit on the number of records to be
presented as drop down.  When the number is exceeded, no drop down is used.  Instead,
a multi-step selection procedure will be used.

\subsubsection{Item numbering}
\label{subsubsec-company-config-defaults-item-numbers}

Many items in the system have sequence numbers: invoices, parts, etc.
These \index{sequence numbers} can be just a number (i.e. 1 or 37) or
they can also be both prefixed and suffixed. For example, INV0001 for invoices and EMP001 for employees or YOU-0001TOO, in which case the next item will be YOU-0002TOO. 

You can only issue every number in the sequence once, but you can issue Y21-001 and Y22-001 by changing the sequence number format at the beginning of the year.

The numbers shown in the input boxes will be used to generate the next number in the
numbering sequence.

\begin{description}[style=nextline]
\item [GL Reference number] The default reference number for the next GL transaction.
\item [Sales invoice/ AR Transaction number] This number is used to generate an invoice
number when none is being filled out by the user.
\item [Sales order number ] Same as Sales invoice number, except that it's used for sales orders @@@ layout issue: the label is too big to fit on the page
\item [Vendor invoice/ AP Transaction number] Same as Sales invoice, except that the number
is used for accounts payable transactions. @@@ layout issue: the label is too big to fit on the page 
\item [Sales quotation number] Same as sales order number, except that it's used for quotations.
\item [RFQ number] Request for quotation number is like the sales quotation number, except
that it is used to track which vendors have been asked for quotes.
\item [Part number] All parts, services and assemblies are identified by a unique number.
When an item is created and no number is entered by the user, a number is generated
from this sequence.
\item [Job/project number] Used when creating new projects.
\item [Employee number ] Same as the sales invoice number, used by new employee entry.
\item [Customer number] @@@ is this the control code number? or is this
meta\_number?? -- Meta-number (CT) 
\item [Vendor number] @@@ same question as customer number
\end{description}

\subsubsection{Check prefix}
\label{subsubsec-company-config-defaults-check-prefix}

 The prefix to use when printing checks. There's no check sequence number. That sequence number is requested from the check printing interface, because
checks can be created outside the application as well, meaning the numbers can
get out of sync.

\subsection{Year end}
\label{subsec-company-config-year-end}

@@@ Rename ``Yearend'' in menu interface to ``Year end''.


@@@ IMO this section doesn't belong here, because it's a process, not config, but does it belong in this menu then? IMO it doesn't...


\subsection{Admin users}
\label{subsec-company-config-admin-users}

@@@ Same as Year end; doesn't belong here...

\subsection{Chart of accounts}
\label{subsec-company-config-coa}

@@@ Chart of accounts isn't exactly a ``process'', but it doesn't feel like being pure
config either. At any rate it's a fact that the CoA discussion is a full chapter in and
of itself - so discussion here isn't necessary anymore.

\subsection{Warehouses}
\label{subsec-company-config-warehouses}

Warehouses are stocking locations. They don't have any properties (in the system)
other than that they have a name. Warehouses can be added, modified and deleted from
the \menupath{System \ma Warehouses} menu item.

\subsection{Departments}
\label{subsec-company-config-departments}

Departments can be used to divide a company in smaller pieces. LedgerSMB distinguishes two
types of departments:

\begin{description}[style=nextline]
\item [Profit centers] which can be associated with any type of transaction, including AR transactions.
\item [Cost centers] which can be associated with all types of transactions, except AR transactions.
\end{description}

Departments can be created (added), modified or deleted through the \menupath{System \ma Departments} menu item.

\subsection{Type of business}
\label{subsec-company-config-business-types}

Types of business are used in sales operations where customers can be assigned a type
of business. Based on the type of business assignment, quotations, sales orders and
invoices will automatically apply discount rates. For each type of business you enter a description and a discount rate to be applied.

\subsection{Languages}
\label{subsec-company-config-languages}

The language table is the table users can select languages from, both to present
the UI of the application as well as the setting for customers to be used to generate
documents.

This listing should correspond to the actual translations of the application being
available in the program installation directory.

Languages can be added, modified or deleted through the \menupath{System \ma Language} menu item.

\subsection{Standard Industry Code (SIC)}
\label{subsec-company-config-sic}

SI codes feature these three fields:

\begin{description}
\item [Code]
\item [Heading]
\item [Description]
\end{description}

When creating a company you can assign that it an SIC code, irrespective of its role (i.e. customer,
vendor, lead or anything else). An example of an SI code system is the
US's NAICS\footnote{\url{https://www.census.gov/naics/}} code.
Other countries have their own coding systems such
as ANZSIC\footnote{\url{https://www.abs.gov.au/statistics/classifications/australian-and-new-zealand-standard-industrial-classification-anzsic/latest-release}} for Australia and New Zealand
and NACE\footnote{\url{https://ec.europa.eu/competition/mergers/cases/index/nace\_all.html}} for Europe

The SIC field currently doesn't support a specific function in the application and is there
merely for informational purposes. However in the future its role could be extended to include
impact on reports, taxes or other functionalities where type of industry could matter.

\subsection {Templates}
\label{subsec-company-config-templates}

Templates are available to control the output format of many LedgerSMB outputs including Balance Sheet, Sales Orders, etc.

There are 3 types of templates: \LaTeX, HTML, and CSV.
Templates are accessed by navigating to \menupath{System \ma Templates}.
You should see the view shown in \figref{fig:system-templates}.

The template you want to change is selected in the "Template" drop-down. The template format you want to change is selected in the "Format" drop-down.

\begin{figure}[h]
        \centering
        \includegraphics[width=\linewidth]{system-templates.png}
        \caption{System templates screen}
        \label{fig:system-templates}
\end{figure}

\subsubsection{\LaTeX{} templates}
\label{subsec-company-config-latex-templates}

To change a \LaTeX{} template navigate to \menupath{System \ma Templates}.
Select, for example, Template "Invoice" and Format "tex", you should see the view shown in \figref{fig:system-templates-edit-invoice}

\begin{figure}[h]
        \centering
        \includegraphics[width=\linewidth]{system-templates-edit-invoice.png}
        \caption{Templates edit invoice screen}
        \label{fig:system-templates-edit-invoice}
\end{figure}

To add a file to the latex template, first upload the image file to the database.  
This can be accomplished by navigating to \menupath{System \ma Files}.

To include this graphic file in your \LaTeX{} document, it needs to be retrieved from
the database and temporarily stored in a location accessible to the PDF
generator. Once the file is in the database, then the function \texttt{dbfile\_path} handles that.

For example, If the graphic file is named "FL\_Logo\_icon\_250x250.png", then add something like the following to the \LaTeX{} template using the \texttt{Edit} button.

\begin{verbatim}
\parbox[b]{.1\textwidth}{%
    \includegraphics[scale=0.7]{%
        <?lsmb dbfile_path("FL_Logo_icon_250x250.png")?>}
}
\end{verbatim}

After editing the template must be uploaded to the database using the \texttt{Upload} button.

\subsubsection{HTML templates}
\label{subsec-company-config-html-templates}

@@@TODO Add HTML specific template info.

\subsubsection{CSV templates}
\label{subsec-company-config-csv-templates}

@@@TODO Add CSV specific template information here.




\part{Administration}
\label{part:Administration}

%\section{Introduction}
%This section of the book describes which tasks and processes might need to be carried out
%on a regular basis in order to keep the application in good health and in line with
%end-users requirements.  As a result this part is more related to the functional rather
%than technical parts of the application.
%
%The fact that the tasks described in this part of the book may be recurring during the
%lifetime of the application does not exclude them from being part of the setup phase
%of LedgerSMB.  In other words: you're likely to have to dig into this section when
%creating a company as well as when maintaining it.

\chapter{Overview}
\label{cha:AdministrationOverview}

\section{Introduction}

This part of the book describes the tasks and processes that may need to be carried out
on a regular basis in order to keep the application in good health and in line with
requirements from end users.

Maintenance may require different types of system access for different types of tasks:

\begin{enumerate}
\item Within application tasks, such as user management, require an appropriately authorized
   normal application login
\item Database administration tasks, such as backups and application upgrades,
 require
   a database level login to be used with 'setup.pl'
\item Other system-level maintenance tasks, such as updating PostgreSQL or Apache, which
   require user accounts on the server hosting LedgerSMB
\end{enumerate}

% @@@ (Within-application tasks)

% @@@ (DBA tasks from setup.pl)

% @@@ (Outside-application tasks)

\chapter{User management}



\section{User creation}

Users experienced with LedgerSMB 1.2 or before or SQL-Ledger (any version) are
referred to appendix \ref{sec:DifferencesUsers} to read about the differences
with version 1.3.

In order to create users, the current user must be sufficiently authorized. The user
created at application set up time is such a user.

Go to the System $\rightarrow$ Admin Users $\rightarrow$ Add User. You'll be presented the page as shown in figure \ref{fig:create-user-step1}.

\begin{figure}[h]
\includegraphics{create-user-step1.png}
\caption{Screen for user creation - step 1}
\end{figure}
\label{fig:create-user-step1}

The value entered in the 'Username' field will cause a database user by that name
to be created. Database users are a global resource, meaning that a collision will
occur if multiple people try to define the same user in multiple companies. Section
\ref{sec:UserImports} describes how to use the same user across multiple companies.

Enter the password to be used for this user into the ``Password'' field. If you're
importing a user, please leave the field empty -- that will prevent the password
from being changed.

The ``Import'' field is discussed in section \ref{sec:UserImports}. To create a new
user, leave the setting at ``No''.

All of the ``First Name'', ``Last Name'' and ``Employee No.'' fields are required.
However, when no employee number is specified, the system will generate one using
the sequence specified in the Defaults screen as documented in section
\ref{sec:DefaultsItemNumbering}.

The ``Country'' field speaks for itself and is required as well.


\section{User authorization}


After filling out all the fields as described in the previous section and
clicking ``Save user'', you'll be presented
a second screen in the user creation process: the user authorization screen.
See figure \ref{fig:create-user-step2} for a screenshot of the top of that screen.

\begin{figure}[h]
\includegraphics{create-user-step2.png}
\caption{Screen for user creation - step 2}
\end{figure}
\label{fig:create-user-step2}

The process of assigning user authorizations is the process by which the granted
access to specific parts of the application. One can imagine that - in a moderately
sized company - sales should not be editing accounting data and accountants should
not be editing sales data. Yet, in order to cooperate, both parties need to be
given access to the same application. This is where authorizations come in.

In aforementioned screen, which equals the ``Edit user'' screen, you have to assign the
newly created user his application rights. By default, the user doesn't have any
rights. Checking all check marks makes the user an application ``super user'', i.e.
gives the user all available application rights.

For a description of the roles a user can be assigned and their effects, the
reader is referred to appendix \ref{cha:RolesListing}.


\section{Maintaining users}

\subsection{Editing user information and authorizations}

When the role of a user in the company changes, it may be necessary to assign
that user new roles and possibly revoke some other roles. This can be done through
user search: System $\rightarrow$ Admin Users $\rightarrow$ Search Users $\rightarrow$ Search $\rightarrow$ {[}edit] which brings you to the same screen as presented in
figure \ref{fig:create-user-step2}.

Similarly, there may be reasons to change the user information, such as a last name
(e.g. upon marriage).

\subsection{Deleting users}

From the ``User search'' result screen, users can be ``deleted'' from the company:
they have their access to the current company revoked.

@@@ Does the delete button delete the user from the cluster as well as from the application?


\section{User imports}
\label{sec:UserImports}

If a database user already exists, e.g. because this user was created to be used
with another LedgerSMB company, it can't be created a second time. In order to be
able to use that user with the current company, it needs to be ``imported'' instead.

The difference between creating a new user and importing one is that the ``Import''
radio button should be set to ``Yes'' and that you should not fill out a password.
If you do, the password of that user will be overwritten for all companies.

All other fields are still applicable: the data entered for other companies isn't
copied to the current company.


\section{Database superuser creation (for setup.pl)}


\chapter{Definition of products and services}

\section{Definition of products}
Structure of products in the system.

\subsection{Definition of parts}
\label{sec:DefinitionOfParts}

\subsection{Definition of partgroups}
\subsection{Definition of assemblies}
\subsection{Definition of overhead}

\section{Definition of services}

\chapter{Chart of accounts}

\section{Accounts and headers}

The system allows ordering accounts into groups by assigning accounts to headers. Headers
can themselves be assigned to other headers resulting in trees of account groups.\footnote{Although the database structure supports this type of account hierarchy
doesn't the 1.3 user take advantage of it yet: in 1.3 accounts can be assigned a header,
but headers can't be assigned to headers themselves.}



\section{Account configuration}

Headers don't have any configuration, other than their number and description. Accounts also
have a number and description, but require additional configuration for the application to work
correctly. The settings are described in the sections that follow.

\subsection{Account options}
\label{sec:AccountOptions}
\begin{itemize}
\item Contra This checkmark identifies the account as a contra account, which means
   that the account is going to hold the opposite of an account it's associated with.
   A good example of this kind would be the depreciation account associated with a fixed
   asset account where the depreciation account contains the credit amount to be added to
   the original asset (debit) value to get the current asset value.
\item Recon This checkmark identifies the account as one which needs reconciliation as
   described in section \ref{sec:Reconciliation}.
\item Tax This checkmark identifies the account as a Tax (VAT) account. Tax accounts need
   to be further configured. See chapter \ref{cha:Taxes} for further discussion of the
   subject.
\end{itemize}

\subsection{Summary accounts}

There are currently three types of summary accounts:

\begin{enumerate}
\item AR Marking an account as a summary account for AR means that all outstanding
   receivable amounts will be posted to this account. The Accounts Receivable administration
   will contain the details of which amount is owed by which customer.
\item AP Same as the AR account, except for amounts owed to vendors.
\item Inventory This account holds the monetary value equal to the items on stock.
\end{enumerate}

\subsection{Account inclusion in drop down lists}

@@@ Add summary

\subsubsection{Receivables \& payables UI}

\begin{itemize}
\item[Income (AR\_amount)] ...
\item[Payment (AR\_paid)] This check mark adds the account to the list of accounts
   to choose from in the Receipts (AR) and Payments (AP) screens. Additionally, it
   adds the account to the part entry screen as described in section \ref{sec:DefinitionOfParts}.
\item[Tax (AR\_tax)] This check mark makes the account show up as a check mark on the
   customer (AR) or vendor (AP) entry screen. See chapter \ref{cha:Taxes} for further discussion.
\item[Overpayment (AR\_overpayment)] Adds the account to the receipts screen as discussed
   in section \ref{sec:UsingOverpayments}.
\item[Discount (AR\_discount)] Adds the account to the customer entry screen's selection
   list for accounts to post 
\end{itemize}

The payables UI works the same way as does the receivables UI. The difference is
that the technical names of the configuration identifiers are prefixed by AP\_ instead
of AR\_.

\subsubsection{Tracking Items}

The items on this line relate to stocked items, i.e. those tracked for inventory: parts and
assemblies.

\begin{enumerate}
\item[Income (IC\_sale)] Adds the account to the selection list of income accounts on the
   part and assembly definition screens.
\item[COGS (IC\_cogs)] Adds the account to the selection list of COGS @@@ accounts on the
   part, assembly and overhead definition screen.
\item[Tax (IC\_taxpart)] Adds a check mark to the part and assembly definition screen
   for the applicable account. See \ref{cha:Taxes} for more details on how taxes
   work in LedgerSMB.
\end{enumerate}

@@@ Question: Labor/Overhead accounts == inventory accounts??

\subsubsection{Non-tracking items}

The items on this line relate to untracked (non stocked) items, i.e. services.

\begin{enumerate}
\item[Income (IC\_income)] Adds the account to the income account selection list in
   the service definition screen.
\item[Expense (IC\_expense)] Adds the account to the expense account selection list in
   the service definition screen.
\item[Tax (IC\_taxservice)] Adds a check mark to the service definition screen for the
   applicable account. See \ref{cha:Taxes} for more details on how taxes work in LedgerSMB.
\end{enumerate}

\subsubsection{Fixed assets}

\begin{enumerate}
\item[Fixed asset (Fixed\_Asset)] Marks the account as holding the original asset value for the fixed
   assets module, for some classes of fixed assets.
\item[Depreciation (Asset\_Dep)] Marks the account as holding the cumulative depreciation amount
   for the fixed assets module, for some classes of fixed assets.
\item[Expense (asset\_expense)] Adds the expense account to the selection list of the fixed assets
   accounting module. See section \ref{sec:FixedAssetAccounting} for more details.
\item[Gain (asset\_gain)] Account to hold book value gain upon disposal of a fixed asset.
\item[Loss (asset\_loss)] Account to hold book value loss upon disposal of a fixed asset.
\end{enumerate}


\section{Alternative accounts (GIFI)}

Next to the regular account numbering scheme, LedgerSMB supports a second numbering scheme: GIFI numbering. The GIFI accounts are a kind of secondary numbering scheme to support legal requirements.

Some jurisdictions require a specific numbering scheme, which can be supported using GIFI. If you
use GIFI account numbers, each account is associated with a GIFI account. Multiple accounts may map
to a single GIFI account.

Many General Ledger reports exist in two variants: a variant using the normal G/L accounts and
one with the GIFI numbering scheme. In the GIFI variant, when a single GIFI has multiple accounts,
the total reported under GIFI is the sum of the mapped accounts.


\subsection{Maintaining GIFI}

GIFI accounts should be created before being assigned to a standard G/L account. GIFI accounts
can be maintained through the System $\rightarrow$ Chart of accounts $\rightarrow$ Add GIFI and List GIFI menu items. Existing accounts can be edited by selecting them from the List GIFI menu, which opens a page where individual GIFI items can have their number or
description adjusted.


\chapter{Taxes}
\label{cha:Taxes}

\section{Overview}



\section{Tax calculation plug-ins}
\label{sec:TaxRulePlugins}

\chapter{Pricing}

\section{Definition of types of business}

\section{Definition of price groups}

\chapter{Contingency planning}

\section{Backup and restore}

\subsection{Using setup.pl}

\subsection{Using PostgreSQL administration tools}

\section{Advanced PostgreSQL: replication}

\chapter{Software updates}

\section{LedgerSMB patch release roll out}








\part{Workflows}
\label{part:Workflows}

\chapter{Customers and vendors}

\section{Creating customers and vendors}

\section{Multiple customers within one company}

\section{Creating vendors from customers}

% \section{}

\section{Maintaining contact information}


\chapter{Quotations from Vendors and for Customers}

\section{Creating Quotations and RFQs}

\section{Attaching files to quotations}
\label{sec:FileAttachments}


\chapter{Sales and vendor orders}
\label{cha:OrderManagement}

\section{Creating new orders}

\section{Creating orders from quotations}

\section{Creating orders from projects}

\section{Creating purchase orders from sales orders}

\section{Combining orders}

\section{Recurring orders}
\label{sec:RecurringOrders}

% @@@ ### Not getting these to work in the test system. Need to check with Chris.

\chapter{Sales and vendor invoices}

Invoices can come from multiple sources. When the quotation and order
management functionalities in LedgerSMB aren't used, they will usually
be entered manually. This work flow is covered in Section
\ref{sec:ManuallyCreatingInvoices}.
When order management \emph{is} being used they mostly originate from orders
which is covered in Section \ref{sec:InvoicesFromOrders}.



\section{Creating invoices from orders}
\label{sec:InvoicesFromOrders}

% This section first because the previous chapter is about orders,
% which seems like a natural flow for the book


\section{Creating new invoices}
\label{sec:ManuallyCreatingInvoices}


%% this section is about Sales and Vendor Invoices

When a business decides not to use the order management as per the previous
chapter it may find itself in need to manually enter invoices. But even
if it does use order management, it may be necessary to enter an invoice
directly.

When creating a transaction to record that the company owes another
entity (a vendor invoice) or that it has outstanding receivables,
LedgerSMB offers two options:

\begin{enumerate}
\item Invoices
\item Transactions
\end{enumerate}

Transactions have very limited functionality: they allow a user to enter
a debt owed or owned into the AR and AP subsystems. They also require the
user to think how the other side of the transaction should be registered;
i.e. which cost account the AP transaction should be posted against, or
which income account the AR transaction should be posted against. If there
are sales taxes applicable, the user is required to manually calculate and
enter them.

Invoices offer a much more clever set of functionalities. First of all, it
allows the user to create a document to be sent to the vendor or customer.
Second, invoices take advantage of parts and services
to automate calculation of sales taxes. Third, invoices update inventory
for items held in stock (parts, assemblies). Transactions offer none of this.

\subsection{Invoices}

As mentioned in the previous paragraph, invoices can perform automatic
sales tax calculations, maintain inventory and post income (or expense)
to the correct GL accounts.

To be able to do so, they need the items on the invoice to be correctly
configured. See \charef{cha:ProductsDefinition} how to set up products
and services.

% @@@ Invoice entry screen screenshot(s)

\subsection{Transactions}

Transactions serve an important purpose not handled by invoices: payroll
calculations are often too difficult to fit in the simple ``amount times price''
model offered by invoices. In order to still be able to track which ``vendor''
was paid which amount such payment obligations can be recorded in the AP subsystem
with a transaction.

Likewise it's often more hassle than it's worth to create the parts and services
required to correctly calculate the utility bill. In such cases the transaction
(possibly with a linked document as supporting evidence) offers good per-vendor
traceable history records.

% @@@ Transaction creation screen shot(s)

\section{Recurring invoices}



\section{Invalidating invoices}

Sometimes, it's necessary to invalidate an invoice. When an invoice has been
posted, this also means derived administrations have been updated, such as
inventory for the items on the invoice.

To undo the effects of an invoice, i.e. to reduce the amount outstanding with a
customer, use the \texttt{VOID} button on the invoice screen as shown in @@@figref .
This creates a new invoice by the same number as the original, except that the new
invoice has a suffix \texttt{-VOID}.

\begin{quotation}
Unfortunately, in LedgerSMB 1.3 - the earlier versions - voiding an invoice did not
automatically close the original and voiding invoices.  To close both invoices from
the open invoice overview, use the cash receipt process as described in
\secref{sec:SinglePayments} to make a zero amount payment.
\end{quotation}

\section{Correcting and deleting invoices}
\label{sec:CorrectingInvoices}

There's only one way to persist an invoice in LedgerSMB: posting it. This means
the invoice becomes part of the accounting information. One of the primary
properties of an accounting system is to record full audit trails and help enforce
internal controls as detailed in \secref{sec:AccountingSystemRequirements}. Because
of that fact there's no way to delete or edit invoices after they have been posted
\footnote{LedgerSMB currently does not support saving an invoice without posting
it. This functionality is on the roadmap for addition when the AR/AP functionality
is being rewritten - currently 1.5 or 1.6.}.

The only way to ``undo'' an invoice is by voiding it. This is important for several
reasons:

\begin{enumerate}
\item Invoices can't be deleted (because they're accounting data)
\item Invoices pose a claim on the assets of a customer
\label{item:InvoicesAsClaims}
\item @@@ others?
\end{enumerate}

Specifically item \ref{item:InvoicesAsClaims} is important: when you sent the invoice
to your customer, you effectively sent them a claim. When you decide to refrain from
pursuing that claim, you should notify them of that fact so they have the documentation
to update their accounting system to reflect that fact: they need your documentation
to void their vendor invoice, instead of paying it.

For the same reason it's ill-advised (and no longer supported) to edit invoices:
when a customer has multiple invoices, each stating a different amount, all
using the same invoice number; how is that customer supposed to document (verifiably)
that the claim has been settled satisfactorily by paying the one he did?

See the Remarks section at the end of this chapter for details on how to handle
the draft invoice requirement.

\section{Handling invoice disputes}

%When an invoice has been sent to a customer it could happen that the customer disagrees
%with the invoice due to incorrect amounts, discounts, products, etc.
%
%Depending on whether the customer has already paid the invoice there are several
%options to handle this situation:
%
%@@@ AR Invoices have nothing to do with it! They parallel AR Transactions!!
%
%\begin{description}
%\item [Credit invoices] The customer has not paid the invoice yet
%\item [Credit notes] The customer has paid the invoice and allows offsetting against
%   other (possibly future) invoices
%\item[AR vouchers] The customer has paid the invoice and demands repayment
%\end{description}
%
%Note that the naming in the listing above applies to AR items but could equally apply
%to AP items by replacing the word ``Credit'' by ``Debit'' and ``AR'' by ``AP''.
%
%
%
%\subsection{Credit and debit invoices}
%
%@@@ produce an accounting document to exchange with the customer/vendor
%@@@ restock credited parts
%@@@ reverse taxes
%
%\subsection{Credit and debit notes}
%
%@@@ generic transaction, not related to orders, inventory or anything else
%@@@ amount taken out of an income account, which is to be selected
%@@@ i.e. based on the account on which the original income was posted.
%
%\subsection{AR and AP Vouchers}
%
%@@@ like a credit note or debit note, meant to initiate
%@@@ a repayment to the customer
%
%@@@ exists as a batch workflow only -- to support separation of duties


\section{Remarks}

\begin{description}
\item [Why can't I send a draft invoice to a customer and edit it
   to match their expectations?] 
You can't edit invoices any more in LedgerSMB 1.3 because it breaks the audit trail
in financial accounting. But in fact there's functionality available which is meant
exactly for this purpose. It's called ``Sales order'' and its details are in
\charef{cha:OrderManagement}. Sales orders can be converted - upon customer approval -
into an invoice with a click of a button.
\end{description}


\chapter{Shop sales}

\section{Opening and closing the cash register}

\section{Shop sales invoices}


\chapter{Manufacturing management}

\section{Producing sales orders}

\subsection{Work orders}


\chapter{Inventory management}

\section{Shipping}

\subsection{Pick lists}
\subsection{Packing list}

\section{Receiving}

\section{Partial shipments}

\section{Transferring between warehouses}

\section{Inventory reporting}



\chapter{Accounts receivable and payable}

\section{Creating generic AR/AP items}

\section{Handling refunds, overpayments and advances}

- this bit is about credit notes and debit notes

\section{Handling returns}

--> this bit is about credit (sales) and debit (vendor) invoices

\chapter{Credit risk management}

\section{Introduction}

A company runs credit risk when it gives credit: it runs the risk of the
creditor not paying off its debts.  LedgerSMB features two ways to manage
the risks involved:

\begin{enumerate}
\item Limit management
\item Arrears management
\end{enumerate}

The former tries to limit the risk involved by making sure no customer
receives more credit than a certain limit while the latter tries to
make sure any over due payments get cashed.

\section{Limit management}

Limit management should prevent a company on one hand from delivering too much
to its customers at once and on the other from taking (and delivering) new orders
to customers with too high amount of unpaid invoices.

@@@ Where to set up limits

@@@ How to monitor limits


\section{Managing arrears}

As mentioned before, the process of managing arrears is directed toward
detecting arrears positions with customers early and taking appropriate
action.


\subsection{Monitoring AR aging}

In order to find out about over due invoices, a company should run the AR
aging report available under AR $\rightarrow$ Reports $\rightarrow$ AR Aging.
The initial screen presents parameters for the aging report to be generated.

@@@ discuss parameters

This report shows customers and their outstanding invoices categorised as:

\begin{description}
\item [Current] Invoices not over due
\item [30] Invoice amounts over due by 30 days or less
\item [60] Invoice amounts over due by 60 days or less (but more than 30)
\item [90] Invoice amounts over due by 90 days or less (but more than 60)
\end{description}


@@@ Printing / mailing aging reports to customers


\subsection{Tracking invoice history}

After an invoice becomes over due a process will be started to remind
the customer of the outstanding amount requiring payment.

In order to keep records of actions taken to chase customer payment,
the invoice screen has an ``Internal Notes'' field which can be edited
after the invoice has been posted.

In order to save any edits to that field, hit the ``Save Info'' button.

\begin{quotation}
Note that the ``Save Info'' button also saves any changes to the ``TaxForm'' column or
rather, any information that's not accounting information (posted to the books and
thereby fixed) nor information which appears on the invoice - which also should remain
unedited in order to be able to generate an exact copy at a later date.
\end{quotation}


\subsection{Special treatment of invoices}

There may be good reasons to treat some over due invoices differently. E.g. in case
payment arrangements have been made with the customer and further standard arrears
management would not be appropriate any more.

In this case, you can put an invoice ``On Hold''. The opposite of being on hold is
being active. The AR Aging report allows selection of all invoices, only active
invoices (those not on hold) or only invoices on hold.


\section{Interest on arrears}

\section{Allowance for doubtful accounts}



\section{Writing off bad debt}

\subsection{Direct write-off}

\subsection{Allowed-for write-off}


\chapter{Receipts and payment processing}


\section{Single receipts and payments}
\label{sec:SinglePayments}

% The X column in the single payment interface deletes the checked invoice
% from the payment list on the next screen update.
% ### Shouldn't this be replaced by some nice JS/Ajax code which hides the row instead?

% the input box in the single-receipt/payment interface right after the Cash|Check|Deposit|Other
% plays an important role in the bank statement reconciliation. The drop-down selected
% selects a numbering range for real world documents which should be uniquely identifying
% documents. Reconciliation aggregates 

\section{Batch receipts and payments}


\section{Check payments}

% Cash -> Vouchers -> Payments, create batch, print checks from the AP transactions.

\section{Using overpayments}
\label{sec:UsingOverpayments}

\section{Receipt and payment reversal}

% Cash/Vouchers/Reverse Payment
% Batches/Approval

\section{Receipts and payments in foreign currencies}

\chapter{Accounting}

\section{Separation of duties: Transaction approval}
\label{sec:SeparationOfDuties}

\section{Bank reconciliation}
\label{sec:Reconciliation}



\section{Period closing}

Period closing is a concept used by accountants to ensure that audited and
known-correct accounting data stays correct by ``freezing'' it: by closing
an accounting period no modifications can be made to the accounting data
before a certain date.

LedgerSMB supports this concept through the System $\rightarrow$ Audit Control
menu, where you'll find the ``Close Books up to'' item. By filling out a date
in the input field and hitting Continue posting to dates before the entered
date will be disallowed.

% @@@ screenshot

\begin{quotation}
Note that due to a design limitation in LedgerSMB 1.3 - to be lifted with the
general AR/AP redesign - invoices in foreign currencies can't be reversed on
other dates than their original posting date. That is: they can, but their
reversal will result in P\&L and balance sheet effects which presumably isn't
desirable. Since period closing disables posting before a certain date this
functionality may have negative side effects in some set ups.

Also note that this pertains exclusively to invoices and transactions in foreign
currencies and has no effect in case of invoices and transactions in the default
currency.
\end{quotation}

\section{Year-end processing}
\label{sec:YearEndProcessing}

Year end closing is a concept which prepares the accounting books for the next
accounting year. Note that this is unrelated to the calendar year but to the
accounting year of the company instead. To muddy the waters even more: there's
no inherent requirement for this process to be run at least once a year. If the
first book year of the company spans more than a year, then this procedure will
be run more than a year after starting up the company.

This procedure freezes the accounting data in the year to be closed as described
in the previous section. Additionally it clears out the profit and loss accounts:
setting all the account
balances to zero by posting their balance to the retained earnings account. Some
businesses prefer to create a retained earnings account for each book year they
close. LedgerSMB supports that use-case by allowing the user to select which
retained earnings account the balance should be posted to.

Some companies want may to include additional transactions related to dividend
payment regarding the current year: reduce the equity by the amount paid as
dividends in transactions marked as ``year-end transaction''. Support
for this use case isn't available in LedgerSMB 1.3.

% @@@ screenshot

\section{Entering general accounting documents}

Even though the application handles many general ledger postings as consequences
from work flows elsewhere in the system - thus not requiring separate postings -
sometimes the need may occur to create manual postings not resulting from
AR or AP transactions or till and inventory adjustments.

One example of a case like that is the calculation, and posting of
corporate taxes presumably at the end of each accounting period but at least
at the end of the book year.

% @@@ screenshot

\section{Fixed asset accounting}
\label{sec:FixedAssetAccounting}


\section{Tax (VAT) reporting}

\begin{itemize}
\item 1099
\item EU VAT
\end{itemize}


\section{Reporting}
\subsection{Income statement}
\subsection{Balance sheet}
\subsection{Trial balance}



% !TeX encoding = UTF-8
% !TeX spellcheck = en_US



\part{Customization}
\label{part:Customization}

\chapter{Batch data import methods}

LedgerSMB 1.3 doesn't have batch data imports built in out of the box,
except for bank statement imports for reconciliation purposes. This
chapter starts to explain how to use the one import that is supported
out of the box. It then goes on to highlight some often-used customizations: 
It is quite easy to customize a LedgerSMB 1.3 instance using
the LedgerSMB 1.4-built-in data import routines to create a variety
of file-upload based data imports.

\section{Custom bank statement import}


\section{Data imports from 1.4}

\subsection{Chart of Accounts import}

This function allows bulk import of an entire chart of accounts using a
single \gls{csv} file.

The first line of the file contains the headers of the columns. The
import routine expects the columns as presented in (table name).

\begin{description}
\item [accno] Number of the account or header
\item [desc] Description for the account or header
\item [charttype] ``H'' for heading record, ``A'' for account record
\item [category] ``I'' for Income, ``E'' for Expense, ``A'' for Asset, ``L'' for Liability
\item [contra] ``0'' if not a contra-account, ``1'' if it is.
\item [tax] ``0'' if it is not a tax-account, ``1'' if it is.
\item [link] A colon (':') separated list of acccount checkmark values (links) as described
    in \secref{subsec-coa-account-links}, or ``AR'', ``AP'' or ``IC'' for the specified summary accounts
\item [heading] Id of the heading to group the account (or heading) under
\item [gifi] GIFI account number
\end{description}

All lines after the first one are considered to be data lines.

\chapter{Add-ons and plug-ins}





\part{Appendices}
\appendix

\chapter{Differences between version 1.2 and 1.3}

\section{Users}
\label{sec:DifferencesUsers}

The way users are defined and used differs greatly between LedgerSMB 1.3 and
older versions. In version 1.3 user access to the database is enforced by the
database itself. This means that users logging in to the LedgerSMB web application
are in reality logging into the PostgreSQL database. In older versions, the web
app would verify the user's credentials (using a common database connection used
for all users).

The difference between these approaches is that security is no longer (solely)
maintained by the web application - with all inherent risks. Instead, the database
now plays an important role as well. The effect is that the LedgerSMB team now
leverages the experience of the PostgreSQL community - a highly respected community
well known for its security focus - to make sure your data stays secure.

This structure also enables LedgerSMB 1.3 to offer separation of duties and
authorizations throughout the application without being required to do a full
rewrite of the application.

It's this shift in paradigm that makes it impossible to meaningfully migrate
users from older LedgerSMB and SQL-Ledger versions to LedgerSMB 1.3.


\chapter{Migration}

\section{Introduction}

There are plenty of reasons to want to migrate to LedgerSMB 1.3:

\begin{enumerate}
\item Separation of duties
\item Security better integrated into the application
\item Better, more strict data model
\label{item:StricterDataModel}
\item Some important sources of user error eliminated
\item Better workflows for cash reconciliation
\item @@@ others?
\end{enumerate}

Yet, while item \ref{item:StricterDataModel} is a good reason to want to switch, it's
also a reason why migration from older versions to 1.3 can be harder than earlier
migrations: when the data
in your older version is not consistent, it won't fit into the new data model and
will need to be fixed first.

Especially if your database has a very long line of history, being migrated trough
lots of SQL-Ledger and LedgerSMB versions, you may want to consider asking for help
from a professional party. It could save you a lot of time.

However, don't be discouraged and have a go yourself first. Just be sure to run
your upgrade on a backup database: the migration process is non-destructive, but
in case accounting data is involved: better safe than sorry!

Also it is worth noting that a number of automatic checks are performed on your
data prior to migration, and to the extent possible, you are given an
opportunity to fix those issues identified.  Because these checks are
pre-migration checks, they are written to your old data and will persist after
backing out of a migration to 1.3.

\section{From older LedgerSMB versions}

\section{From SQL-Ledger 2.8}

\section{From older SQL-Ledger versions}

\section{From other accounting packages}

While accounting and ERP solution have wildly differing structures to record their
data, this sections uses data with a relatively simple structure as a show case of
how this problem may be dealt with.

\begin{quote}
Note that the encoding you use to transfer to the database depends on the settings
used to create the PostgreSQL database with.  A migration is a good moment to think
about encodings an solve older encoding issues.  Now would be a good moment to
anticipate the requirement for accented characters and non-western alphabets: set up
a UTF-8 encoded database and recode your data accordingly.
\end{quote}

\subsection{Migrating customers}

The source system for this section uses a structure where every company has one
contact person, one address, one phone number and e-mail.

In order to understand how to migrate this data structure to LedgerSMB, it's
important to understand that:

\begin{enumerate}
\item The company from the source maps to the \emph{Company} and \emph{Entity} entities
\item The contact person maps to the \emph{Entity Credit Account} entity
\item The address maps to the \emph{Location} entity - and requires a location class: Sales, Billing or Shipping
\item The phone number, fax number and e-mail map to \emph{Contact items}
\end{enumerate}

The reason behind the separation between the \emph{Company} and \emph{Entity} entities
is that every customer is an \emph{Entity}, but not all entities are companies, since
some entities are \emph{Person}s - natural persons.

@@@ How to

\subsection{Migrating parts, services, ...}
\subsection{Migrating open items}
\subsection{Migrating your ledger}

\subsection{Migrating between PostgreSQL versions}

If you're migrating between PostgreSQL versions, there are a few things to take
into account.

% pg_dump: use the NEW pg_dump utility against the old database
% pg_upgrade: <TODO>
% pg_<migratecluster>: <TODO>; in case of unclean extension upgrade,
%              possible need to issue ``CREATE EXTENSION tablefunc FROM UNPACKAGED''

\subsubsection{Notes on migration from 8.3 (or earlier) to 8.4 (or later)}

% Performance benefit due to built in support for recursive queries only available
% after the next setup.pl run
% Also, after next setup.pl run, one should uninstall the tablefunc extension

% 9.1+ DROP EXTENSION tablefunc;
% 8.4, 9.0: use uninstall_tablefunc.sql from the contrib directory

LedgerSMB 1.3 uses some extension modules for versions 8.3 and before for functionality
that has been built into 8.4 and later. To make use of the (faster) built in version
of that functionality, the following restore procedure should be used.

\begin{enumerate}
\item Migrate the database to the new function as described in section @@@ TODO
\item If you're using 9.1 and up, issue the command ``CREATE EXTENSION tablefunc FROM UNPACKAGED''
   from a psql prompt when connected to the company database
\item Run 'setup.pl' from your browser to upgrade the database's routines; this command will
   install routines optimized for your version of PostgreSQL
\item Run the command
	\begin{description}
	\item [8.4 or 9.0] \$ psql ... -f uninstall\_tablefunc.sql
	\item [9.1 and up] ``DROP EXTENSION tablefunc;'' from a psql session connected
		to the company database
	\end{description}
	to clean up functions and procedures in the database which are no longer used
\end{enumerate}


\chapter{Listing of application roles}
\label{cha:RolesListing}

Application roles specify the right to execute one or more tasks in the application.
LedgerSMB enforces these roles by allowing a user to select (list, read) data from or to
insert (create), update (edit) or delete (delete) data in the tables holding the data
related to the execution of these tasks.

\begin{description}
\item [account\_all] Allows the user to both create new and edit existing GL accounts.
\item [account\_create] Allows the user to create (but not edit) new GL accounts.
\item [account\_edit] Allows the user to edit (but not create) GL accounts.
\item [ap\_all]
\item [ap\_all\_transactions]
\item [ap\_all\_vouchers]
\item [ap\_invoice\_create]
\item [ap\_invoice\_create\_voucher]
\item [ap\_transaction\_all] \footnote{Available as of 1.3.21, missing before}
\item [ap\_transaction\_create]
\item [ap\_transaction\_create\_voucher]
\item [ap\_transaction\_list]
\item [ar\_all]
\item [ar\_invoice\_create] Allows the user to create and update sales invoices. If the
   user needs to be able to enter invoices in foreign currencies, the
   \emph{exchangerate\_edit} role must be assigned as well.
\item [ar\_transaction\_all]
% just added it -- Chris T
\item [ar\_transaction\_create]
\item [ar\_transaction\_create\_voucher]
\item [ar\_transaction\_list]
\item [assembly\_stock]
\item [assets\_administer]
\item [assets\_approve]
\item [assets\_depreciate]
\item [assets\_enter]
\item [auditor]
\item [audit\_trail\_maintenance]
\item [backup] \emph{Superseeded}. This role has been replaced by backup functionality
   in setup.pl
\item [batch\_create] Allows the user to create new batches.
\item [batch\_list]
\item [batch\_post] Allows the user to post batches; this authorization includes the
   right to search for batches (and therefore includes \emph{batch\_list})
\item [business\_type\_all]
\item [business\_type\_create]
\item [business\_type\_edit]
\item [cash\_all]
\item [close\_till]
\item [contact\_all\_rights]
\item [contact\_create]
\item [contact\_edit]
\item [contact\_read]
\item [department\_all]
\item [department\_create]
\item [department\_edit]
\item [draft\_edit]
\item [employees\_manage]
\item [file\_attach\_order]
\item [file\_attach\_part]
\item [file\_attach\_tx]
\item [file\_read]
\item [financial\_reports]
\item [gifi\_create]
\item [gifi\_edit]
\item [gl\_all]
\item [gl\_reports]
\item [gl\_transaction\_create]
\item [gl\_voucher\_create]
\item [inventory\_all]
\item [inventory\_receive]
\item [inventory\_reports]
\item [inventory\_ship]
\item [inventory\_transfer]
\item [language\_create]
\item [language\_edit]
\item [list\_all\_open]
\item [manual\_translation\_all]
\item [orders\_generate]
\item [orders\_manage]
\item [orders\_purchase\_consolidate]
\item [orders\_sales\_consolidate]
\item [orders\_sales\_to\_purchase]
\item [part\_create]
\item [part\_edit]
\item [partsgroup\_translation\_create]
\item [part\_translation\_create]
\item [payment\_process]
\item [pos\_all]
\item [pos\_cashier]
\item [pos\_enter]
\item [pricegroup\_create]
\item [pricegroup\_edit]
\item [print\_jobs]
\item [print\_jobs\_list]
\item [project\_create]
\item [project\_edit]
\item [project\_order\_generate]
\item [project\_timecard\_add]
\item [project\_timecard\_list]
\item [project\_translation\_create]
\item [purchase\_order\_create]
\item [purchase\_order\_edit]
\item [purchase\_order\_list]
\item [receipt\_process]
\item [reconciliation\_all]
\item [reconciliation\_approve]
\item [reconciliation\_enter]
\item [recurring]
\item [rfq\_create]
\item [rfq\_list]
\item [sales\_order\_create]
\item [sales\_order\_edit]
\item [sales\_order\_list]
\item [sales\_quotation\_create]
\item [sales\_quotation\_list]
\item [sic\_all]
\item [sic\_create]
\item [sic\_edit]
\item [system\_admin]
\item [system\_settings\_change]
\item [system\_settings\_list]
\item [taxes\_set]
\item [tax\_form\_save]
\item [template\_edit]
\item [users\_manage]
\item [voucher\_delete]
\item [warehouse\_create]
\item [warehouse\_edit]
\item [yearend\_run]
\end{description}

\chapter{Open source explained}

\section{An open source application}

\section{An open source book}


\chapter{Copyright and license}

Copyright (c) 2011, 2012 Erik H\"ulsmann.


This work is licensed under the Creative Commons Attribution License.
To view a copy of this license, visit \url{http://creativecommons.org/licenses/by/3.0/}
or send a letter to Creative Commons, 559 Nathan Abbott Way,
Stanford, California 94305, USA.

A summary of the license is given below, followed by the full legal text.

\section{License summary}

\begin{verbatim}

You are free:
  * to share -- to copy, distribute and transmit the work
  * to remix -- to adapt the work


Under the following condition:
  You must attribute he work in the manner specified by the author or licensor
  (but in a way that suggests that they endorse you or your use of the work).


With the understanding that:
  Waiver -- Any of the above conditions can be waived if you get permission
            from the copyright holder.
  
  Other rights -- In no way are any of the following rights affected by the license:
     * Your fair dealing or fair use rights, or other applicable 
         copyright exceptions and limitations;
     * The author's moral rights;
     * Rights other persons may have either in the work itself or
         in how the work is used, such as publicity or privacy rights.



\end{verbatim}



\section{Legal full text}

\begin{verbatim}
License

THE WORK (AS DEFINED BELOW) IS PROVIDED UNDER THE TERMS OF THIS 
CREATIVE COMMONS PUBLIC LICENSE ("CCPL" OR "LICENSE"). THE WORK
IS PROTECTED BY COPYRIGHT AND/OR OTHER APPLICABLE LAW. ANY USE OF
THE WORK OTHER THAN AS AUTHORIZED UNDER THIS LICENSE OR COPYRIGHT
LAW IS PROHIBITED.

BY EXERCISING ANY RIGHTS TO THE WORK PROVIDED HERE, YOU ACCEPT AND
AGREE TO BE BOUND BY THE TERMS OF THIS LICENSE. TO THE EXTENT THIS
LICENSE MAY BE CONSIDERED TO BE A CONTRACT, THE LICENSOR GRANTS
YOU THE RIGHTS CONTAINED HERE IN CONSIDERATION OF YOUR ACCEPTANCE
OF SUCH TERMS AND CONDITIONS.

1. Definitions

"Adaptation" means a work based upon the Work, or upon the Work
and other pre-existing works, such as a translation, adaptation,
derivative work, arrangement of music or other alterations of a
literary or artistic work, or phonogram or performance and
includes cinematographic adaptations or any other form in which
the Work may be recast, transformed, or adapted including in any
form recognizably derived from the original, except that a work
that constitutes a Collection will not be considered an Adaptation
for the purpose of this License.
For the avoidance of doubt, where the Work is a musical work,
performance or phonogram, the synchronization of the Work in
timed-relation with a moving image ("synching") will be considered
an Adaptation for the purpose of this License.

"Collection" means a collection of literary or artistic works, such
as encyclopedias and anthologies, or performances, phonograms or
broadcasts, or other works or subject matter other than works
listed in Section 1(f) below, which, by reason of the selection
and arrangement of their contents, constitute intellectual
creations, in which the Work is included in its entirety in
unmodified form along with one or more other contributions, each
constituting separate and independent works in themselves, which
together are assembled into a collective whole. A work that
constitutes a Collection will not be considered an Adaptation
(as defined above) for the purposes of this License.

"Distribute" means to make available to the public the original
and copies of the Work or Adaptation, as appropriate, through sale
or other transfer of ownership.

"Licensor" means the individual, individuals, entity or entities
that offer(s) the Work under the terms of this License.

"Original Author" means, in the case of a literary or artistic
work, the individual, individuals, entity or entities who created
the Work or if no individual or entity can be identified, the
publisher; and in addition (i) in the case of a performance the
actors, singers, musicians, dancers, and other persons who act,
sing, deliver, declaim, play in, interpret or otherwise perform
literary or artistic works or expressions of folklore; (ii) in the
case of a phonogram the producer being the person or legal entity
who first fixes the sounds of a performance or other sounds; and,
(iii) in the case of broadcasts, the organization that transmits
the broadcast.

"Work" means the literary and/or artistic work offered under the
terms of this License including without limitation any production
in the literary, scientific and artistic domain, whatever may be
the mode or form of its expression including digital form, such as
a book, pamphlet and other writing; a lecture, address, sermon or
other work of the same nature; a dramatic or dramatico-musical work;
a choreographic work or entertainment in dumb show; a musical
composition with or without words; a cinematographic work to
which are assimilated works expressed by a process analogous to
cinematography; a work of drawing, painting, architecture,
sculpture, engraving or lithography; a photographic work to which
are assimilated works expressed by a process analogous to
photography; a work of applied art; an illustration, map, plan,
sketch or three-dimensional work relative to geography, topography,
architecture or science; a performance; a broadcast; a phonogram;
a compilation of data to the extent it is protected as a
copyrightable work; or a work performed by a variety or circus
performer to the extent it is not otherwise considered a literary
or artistic work.

"You" means an individual or entity exercising rights under this
License who has not previously violated the terms of this License
with respect to the Work, or who has received express permission
from the Licensor to exercise rights under this License despite
a previous violation.

"Publicly Perform" means to perform public recitations of the
Work and to communicate to the public those public recitations,
by any means or process, including by wire or wireless means or
public digital performances; to make available to the public
Works in such a way that members of the public may access these
Works from a place and at a place individually chosen by them;
to perform the Work to the public by any means or process and
the communication to the public of the performances of the Work,
including by public digital performance; to broadcast and
rebroadcast the Work by any means including signs, sounds or
images.

"Reproduce" means to make copies of the Work by any means
including without limitation by sound or visual recordings and
the right of fixation and reproducing fixations of the Work,
including storage of a protected performance or phonogram in
digital form or other electronic medium.

2. Fair Dealing Rights. Nothing in this License is intended to
reduce, limit, or restrict any uses free from copyright or rights
arising from limitations or exceptions that are provided for in
connection with the copyright protection under copyright law or
other applicable laws.

3. License Grant. Subject to the terms and conditions of this License,
Licensor hereby grants You a worldwide, royalty-free, non-exclusive,
perpetual (for the duration of the applicable copyright) license to
exercise the rights in the Work as stated below:

to Reproduce the Work, to incorporate the Work into one or more
Collections, and to Reproduce the Work as incorporated in the
Collections;
to create and Reproduce Adaptations provided that any such
Adaptation, including any translation in any medium, takes
reasonable steps to clearly label, demarcate or otherwise identify
that changes were made to the original Work. For example, a
translation could be marked "The original work was translated from
English to Spanish," or a modification could indicate "The
original work has been modified.";
to Distribute and Publicly Perform the Work including as
incorporated in Collections; and,
to Distribute and Publicly Perform Adaptations.
For the avoidance of doubt:

Non-waivable Compulsory License Schemes. In those jurisdictions
in which the right to collect royalties through any statutory
or compulsory licensing scheme cannot be waived, the Licensor
reserves the exclusive right to collect such royalties for any
exercise by You of the rights granted under this License;
Waivable Compulsory License Schemes. In those jurisdictions in
which the right to collect royalties through any statutory or
compulsory licensing scheme can be waived, the Licensor waives
the exclusive right to collect such royalties for any exercise
by You of the rights granted under this License; and,
Voluntary License Schemes. The Licensor waives the right to
collect royalties, whether individually or, in the event that
the Licensor is a member of a collecting society that
administers voluntary licensing schemes, via that society, from
any exercise by You of the rights granted under this License.
The above rights may be exercised in all media and formats
whether now known or hereafter devised. The above rights include
the right to make such modifications as are technically
necessary to exercise the rights in other media and formats.
Subject to Section 8(f), all rights not expressly granted by
Licensor are hereby reserved.

4. Restrictions. The license granted in Section 3 above is
expressly made subject to and limited by the following
restrictions:

You may Distribute or Publicly Perform the Work only under
the terms of this License. You must include a copy of, or the
Uniform Resource Identifier (URI) for, this License with every
copy of the Work You Distribute or Publicly Perform. You may
not offer or impose any terms on the Work that restrict the
terms of this License or the ability of the recipient of the
Work to exercise the rights granted to that recipient under
the terms of the License. You may not sublicense the Work. You
must keep intact all notices that refer to this License and to
the disclaimer of warranties with every copy of the Work You
Distribute or Publicly Perform. When You Distribute or Publicly
Perform the Work, You may not impose any effective technological
measures on the Work that restrict the ability of a recipient
of the Work from You to exercise the rights granted to that
recipient under the terms of the License. This Section 4(a)
applies to the Work as incorporated in a Collection, but this
does not require the Collection apart from the Work itself to
be made subject to the terms of this License. If You create
a Collection, upon notice from any Licensor You must, to the
extent practicable, remove from the Collection any credit as
required by Section 4(b), as requested. If You create an
Adaptation, upon notice from any Licensor You must, to the
extent practicable, remove from the Adaptation any credit as
required by Section 4(b), as requested.
If You Distribute, or Publicly Perform the Work or any
Adaptations or Collections, You must, unless a request has
been made pursuant to Section 4(a), keep intact all copyright
notices for the Work and provide, reasonable to the medium or
means You are utilizing: (i) the name of the Original Author
(or pseudonym, if applicable) if supplied, and/or if the
Original Author and/or Licensor designate another party or
parties (e.g., a sponsor institute, publishing entity,
journal) for attribution ("Attribution Parties") in Licensor's
copyright notice, terms of service or by other reasonable
means, the name of such party or parties; (ii) the title of
the Work if supplied; (iii) to the extent reasonably
practicable, the URI, if any, that Licensor specifies to be
associated with the Work, unless such URI does not refer to the
copyright notice or licensing information for the Work;
and (iv) , consistent with Section 3(b), in the case of an
Adaptation, a credit identifying the use of the Work in the
Adaptation (e.g., "French translation of the Work by Original
Author," or "Screenplay based on original Work by Original
Author"). The credit required by this Section 4 (b) may be
implemented in any reasonable manner; provided, however, that
in the case of a Adaptation or Collection, at a minimum such
credit will appear, if a credit for all contributing authors
of the Adaptation or Collection appears, then as part of these
credits and in a manner at least as prominent as the credits
for the other contributing authors. For the avoidance of doubt,
You may only use the credit required by this Section for the
purpose of attribution in the manner set out above and, by
exercising Your rights under this License, You may not
implicitly or explicitly assert or imply any connection
with, sponsorship or endorsement by the Original Author,
Licensor and/or Attribution Parties, as appropriate, of
You or Your use of the Work, without the separate, express
prior written permission of the Original Author, Licensor
and/or Attribution Parties.
Except as otherwise agreed in writing by the Licensor or as
may be otherwise permitted by applicable law, if You Reproduce,
Distribute or Publicly Perform the Work either by itself or
as part of any Adaptations or Collections, You must not distort,
mutilate, modify or take other derogatory action in relation to
the Work which would be prejudicial to the Original Author's
honor or reputation. Licensor agrees that in those
jurisdictions (e.g. Japan), in which any exercise of the right
granted in Section 3(b) of this License (the right to make
Adaptations) would be deemed to be a distortion, mutilation,
modification or other derogatory action prejudicial to the
Original Author's honor and reputation, the Licensor will
waive or not assert, as appropriate, this Section, to the
fullest extent permitted by the applicable national law, to
enable You to reasonably exercise Your right under
Section 3(b) of this License (right to make Adaptations)
but not otherwise.

5. Representations, Warranties and Disclaimer

UNLESS OTHERWISE MUTUALLY AGREED TO BY THE PARTIES IN WRITING,
LICENSOR OFFERS THE WORK AS-IS AND MAKES NO REPRESENTATIONS
OR WARRANTIES OF ANY KIND CONCERNING THE WORK, EXPRESS,
IMPLIED, STATUTORY OR OTHERWISE, INCLUDING, WITHOUT
LIMITATION, WARRANTIES OF TITLE, MERCHANTIBILITY, FITNESS FOR
A PARTICULAR PURPOSE, NONINFRINGEMENT, OR THE ABSENCE OF
LATENT OR OTHER DEFECTS, ACCURACY, OR THE PRESENCE OF ABSENCE
OF ERRORS, WHETHER OR NOT DISCOVERABLE. SOME JURISDICTIONS DO
NOT ALLOW THE EXCLUSION OF IMPLIED WARRANTIES, SO SUCH
EXCLUSION MAY NOT APPLY TO YOU.

6. Limitation on Liability.

EXCEPT TO THE EXTENT REQUIRED BY APPLICABLE LAW, IN NO EVENT
WILL LICENSOR BE LIABLE TO YOU ON ANY LEGAL THEORY FOR ANY
SPECIAL, INCIDENTAL, CONSEQUENTIAL, PUNITIVE OR EXEMPLARY
DAMAGES ARISING OUT OF THIS LICENSE OR THE USE OF THE WORK,
EVEN IF LICENSOR HAS BEEN ADVISED OF THE POSSIBILITY OF SUCH
DAMAGES.

7. Termination

This License and the rights granted hereunder will terminate
automatically upon any breach by You of the terms of this
License. Individuals or entities who have received Adaptations
or Collections from You under this License, however, will not
have their licenses terminated provided such individuals
or entities remain in full compliance with those licenses.
Sections 1, 2, 5, 6, 7, and 8 will survive any termination
of this License. Subject to the above terms and conditions,
the license granted here is perpetual (for the duration of
the applicable copyright in the Work). Notwithstanding the
above, Licensor reserves the right to release the Work under
different license terms or to stop distributing the Work at
any time; provided, however that any such election will not
serve to withdraw this License (or any other license that has
been, or is required to be, granted under the terms of this
License), and this License will continue in full force and
effect unless terminated as stated above.

8. Miscellaneous

Each time You Distribute or Publicly Perform the Work or a
Collection, the Licensor offers to the recipient a license
to the Work on the same terms and conditions as the license
granted to You under this License. Each time You Distribute
or Publicly Perform an Adaptation, Licensor offers to the
recipient a license to the original Work on the same terms and
conditions as the license granted to You under this License.
If any provision of this License is invalid or unenforceable
under applicable law, it shall not affect the validity or
enforceability of the remainder of the terms of this License,
and without further action by the parties to this agreement,
such provision shall be reformed to the minimum extent
necessary to make such provision valid and enforceable.
No term or provision of this License shall be deemed waived
and no breach consented to unless such waiver or consent shall
be in writing and signed by the party to be charged with
such waiver or consent.
This License constitutes the entire agreement between the
parties with respect to the Work licensed here. There are
no understandings, agreements or representations with respect
to the Work not specified here. Licensor shall not be bound
by any additional provisions that may appear in any
communication from You. This License may not be modified
without the mutual written agreement of the Licensor and You.
The rights granted under, and the subject matter referenced,
in this License were drafted utilizing the terminology of the
Berne Convention for the Protection of Literary and Artistic
Works (as amended on September 28, 1979), the Rome Convention
of 1961, the WIPO Copyright Treaty of 1996, the WIPO
Performances and Phonograms Treaty of 1996 and the Universal
Copyright Convention (as revised on July 24, 1971). These
rights and subject matter take effect in the relevant
jurisdiction in which the License terms are sought to be
enforced according to the corresponding provisions of the
implementation of those treaty provisions in the applicable
national law. If the standard suite of rights granted under
applicable copyright law includes additional rights not
granted under this License, such additional rights are deemed
to be included in the License; this License is not intended
to restrict the license of any rights under applicable law.

\end{verbatim}






\end{document}

