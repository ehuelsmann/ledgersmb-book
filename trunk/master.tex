
\documentclass[10pt,A4]{book}
\usepackage{url, palatino, color}
\usepackage[colorlinks, linkcolor=black, urlcolor=black]{hyperref}
\usepackage{graphicx}
\graphicspath{{images/}}


\begin{document}



\author{Erik H\"ulsmann}
\title{Running your business with LedgerSMB - 1.3}


\maketitle


\tableofcontents


% \includegraphics[width=\linewidth]{Openingschreen}

\chapter*{Preface}

\part{Overview}

\chapter{What is LedgerSMB}

\section{Introduction}
LedgerSMB is an open source, web based ERP system. This means it aspires to be
\textit{the} integrated administration system for your business.
The software is being developed by the LedgerSMB project.

The homepage of the project is hosted at \textit{http://ledgersmb.org/}.
However, the actual project activity can be witnessed at the SourceForge
project site hosted at \textit{http://sf.net/projects/ledger-smb/}.  The project
employs the Freenode.net IRC network's \#ledgersmb chat channel to discuss development
and to help out users.

Its open source nature allows you to download it and use it with any
infrastructure you like. So there's no room for vendor lock-in: you can
always take your data and set up your system with another hardware vendor
or set up your own hardware.

\section{Overview of modules}
\label{sec:ModuleOverview}
As with most ERPs LedgerSMB is built up from multiple modules.  Many modules are
available as part of the base application.  Newer modules aren't being integrated
into the base application immediately, but are kept separate for some time to allow
wide spread evaluation of the feature set.  This separation allows these modules
to have their own release schedule.  These separate modules are called \textit{add-on}s.

LedgerSMB 1.3 features the following integrated modules:

\begin{itemize}
\item General ledger
\item Payment and Accounts payable
\item Invoicing and Accounts receivable
\item Fixed asset accounting
\item Time registration and invoicing
\item Point of Sale
\item Quotation and Order management
\item Manufacturing
\item Inventory (warehousing) and shipping management
\item VAT reporting (cash based)
\item Controlling
\begin{itemize}
\item Project accounting
\item Department accounting
\item Budgeting (as an add-on)
\end{itemize}
\item Application administration
\end{itemize}

And add-ons:
\begin{itemize}
\item Enhanced AR and AP support
\item VAT reporting (accrual based)
\item Enhanced trial-balance report
\item Enhanced recurring transactions
\item Payroll (to be created - under discussion at the time of writing)
\end{itemize}

With this list of modules and add-ons LedgerSMB has succesfully been implemented
in shops, manufacturing companies and service oriented companies such as internet
service providers and self-employment situations.

\section{Feature comparison with alternatives}

Packages to compare to:

\begin{itemize}
\item GNUcash
\item OSfinancials
\item ERP5
\item OpenERP
\item xTuple
\end{itemize}

\section{Release history}

The project started out as a fork of SQL-Ledger - the open source ERP
developed by Dieter Simader - somewhere between SQL-Ledger versions 2.6
and 2.8.  A fork happens when a group of developers can't - for whatever
reason - continue to work as one group on a project.  At that time, the
project splits into two or more projects and the fork is in effect.

LedgerSMB split off from the SQL-Ledger project (i.e. forked) because
there was disagreement between developers about how to go forward both with
respect to handling of security vulnerability reports as well as the general
state of the code base.

After the fork, between versions 1.0 and 1.2 a lot of energy was spent on
making LedgerSMB more secure (i.e. less vulnerable).  In technical terms,
measures were taken to fend off (amongst other things):

\begin{itemize}
\item Cross site scripting attacks
\item Replay attacks
\item SQL injection attacks
\end{itemize}

Come version 1.3 the development directed toward improvement of the overall
quality of the code base as the old SQL-Ledger code was in very poor state:
looking very much like webscripts as they were written in 1998, the code had
grown largely outdated in style and was no longer maintainable by 2007.

The 1.3 effort focussed on bringing relief on that front by introducing
modern structure into the application.  With the new application structure
modern and important features were realized: separation of duties (for the
accounting part of the application) and authorizations to allow distinguishing
different roles in your company.

Unfortunately, by the beginning of 2011 the project looked mostly dead from
an outside perspective: the team had not brought forward any releases since
\#\#\#(since when?)\#\#\#, there were no signs of development activity and the
mailing lists (activity of which is a measure for community activity) were
completely silent.

Fortunately development activity was restarted in the first half year of 2011,
leading to the release of version 1.3 by September.  Between September and the
year end in total 10 small bug fixes were released, showing active commitment
of the developers to maintain the application.  New committers showed up, indicating
revived community interest.



\section{System requirements}

The INSTALL file which comes with every LedgerSMB software release should be
considered the authorative source of system requirements.  For the reader's
convenience, the system requirements for the 1.3.9 release have been included
below:

\begin{itemize}
\item Webserver; one of:
\begin{itemize}
\item Apache HTTPD ???(2.0?) or higher
\item LigHTTPd ???
\end{itemize}
\item Perl 5.10 or higher, with these modules:
\begin{itemize}
\item ??
\end{itemize}
\item PostgreSQL 8.4 or higher, with these contribs:
\begin{itemize}
\item tablefunc
\item 
\end{itemize}
\item \LaTeX or TeTeX (for PDF output)
\end{itemize}

Other system requirements such as required RAM and number of CPUs and their speed
largely depend on the expected system activity.

\section{License}

LedgerSMB is being made available under the terms of the
\textit{GNU Public License version 2}, or shorter \textit{GPLv2}.

The project attaches this meaning to the license:
The copyright holders grant you the right to copy and
redistribute the software.  In case you make any modifications to the sofware
you're obligated to make public those changes.  You are however, free to use
the APIs from your own software without being required to publish your own software.

The project considers the following to be APIs:
\begin{itemize}
\item Database tables
\item URLs with their input and output
\item Webservices of any kind
\item Function and object calls
\end{itemize}

The effect of this interpretation is that changes directly to the code base as
well as inheritance of classes defined in the software constitute 'making modifications'.


\chapter{Overview}

\section{Introduction}
This section of the book describes functionalities of the modules listed in section \ref{sec:ModuleOverview}.  The information provided should be enough to get a clear
understanding of what the application can do for you as well as being a sound basis
for parts \ref{part:Configuration} and \ref{part:Administration}.

For in-depth instruction for actual day to day use of LedgerSMB the reader is referred
to part \ref{part:Workflows}.

\section{Technical overview}


% Users: database vs company users
% Companies: companies as databases


The main technical building blocks of LedgerSMB are

\begin{description}
\item[Apache HTTPD] In a standard setup, the role of the http server is very limited. It
may be used to perform some security checks such as
\begin{itemize}
\item restrict the range of IP addresses from which connections are allowed
\item run the LedgerSMB application in an
suEXEC\footnote{http://en.wikipedia.org/wiki/suEXEC} restricted environment
\end{itemize}
\item[PostgreSQL] The database component has multiple functions:
\begin{itemize}
\item Authentication (checking user credentials)
\item Authorization (limiting user access in accordance with roles assigned)
\item Data storage and retrieval
\item Data consistency enforcement
\end{itemize}
This is achieved by creating databases with a large number of LedgerSMB specific
functions and roles.
\item[Intermediate layer] This layer contains the code to mediate between the web
and the database.  For one thing, it forwards authentication requests from the web
to the database.  Additionally, it interprets web requests, reading, updating and
writing to the database, generating the appropriate responses to be shown in the
user's browser.
\end{description}

Due to the architecture described above, where the authentication and authorization
are handed off to PostgreSQL, the development team is able to prevent writing their
own - error prone - authentication code, leveraging widely used and tested code instead.

\section{Layout of the distribution}


\begin{tabular}{|c|p{9cm}|}
\hline Directory &  Description \\
\hline
./ & The root project directory holding documentation files such as
  README, INSTALL and UPDATING  \\ 
LedgerSMB/ & Supporting Perl library code \\
bin/ & Inherited SQL-Ledger application code \\
scripts/ & Post-fork rewritten application code \\
UI/ & Post-fork HTML and CSV application templates with supporting Javascript code \\
templates/ & User templates, used for invoices, quotations, etc. \\
locale/ & Files holding translations for localized application output \\
sql/ & Files required for initialization of the database: chart of accounts,
 table structure, stored procedures, etc \\
\hline 
\end{tabular} 


\section{Functional overview}

One of the main things to remember when working in an ERP is that nearly everything is
linked to the ledger in one way or another.  This relation is much stronger in ERP
systems than others, where accounting information is much more often an after-thought,
if a subject at all.  Since an ERP includes accounting, being able to post to the
ledger is part of the design from the beginning.

% others?




\section{Technical components}

The intermediate layer itself consists of several parts and - as far as code written
after 2007 is concerned - is modelled after the Model-View-Controller design paradigm.
Most of the model has been implemented in the database and its stored procedures, meaning
that the intermediate layer itself implements the view and the controller.

The view - meaning the output as sent in response to the web requests - is generated
using templates.  These templates use the highly respected and often-used Perl templating
library Template Toolkit\footnote{http://template-toolkit.org}.

Code written before 2007 is to be replaced in due course and will follow the same
pattern.


\part{Getting started}
\label{part:GettingStarted}

\chapter{Creating a company administration}

\chapter{The first login}

\chapter{Building up stock}

\chapter{Ramping up to the first sale}

% sending out a quote followed by a sales order

\chapter{Shipping sales}

\chapter{Invoicing}

\chapter{Collecting sales invoice payments}

\section{Customer payments}

\section{Customer payment mismatch}

% choosing between pardonning and registering underpayment

% large ones, as in partial payments or largish under/over payments

% pardonning small mismatches


\chapter{Paying vendor invoices}

% handling vendors who match amounts to exact invoices

% handling vendors with running balances

% handling bounced checks: voiding checks to undo payments of vendor invoices
%   relating to bounced checks

\chapter{Monitoring arrears}

% handling interest on arrears

\chapter{Handling sales taxes}

% invoices with taxes included

% invoices with explicit tax amounts

% 

\chapter{Branching out: services}

% including creation / assignment to different accounts


\chapter{Recording service hours}

\chapter{Customer approval on service hours}

\chapter{Invoicing services}

\chapter{Branching out II: service subscriptions}


\part{Configuration}
\label{part:Configuration}


\chapter{Overview}

\section{Introduction}
This section of the book describes how to set up LedgerSMB and its components.
Configuration is assumed to be mostly one-off and rather technical in nature.  To find
out which tasks might need to be performed in order to keep the application in good
health the reader is referred to the section ``Administration''. 

\chapter{Global configuration}

\section{Apache}

Section about installing on Apache 2+

items to be discussed:

Forwarding of authentication \\
cgi configuration \\
performance: cgiD configuration: don't (yet) [but will be supported once all legacy code is gone] \\
security: suEXEC environment \\

\subsection{Differences between Apache 1.3 and 2+}

Explain how to use lsmb with 1.3 instead of 2+.

\section{PostgreSQL}

pg\_hba.conf: authentication \\
security: local vs IP connections \\




\section{LedgerSMB}

\subsection{ledgersmb.conf}

to be discussed:

Individual config keys; full discussion of possible values in reference appendix?

\subsubsection{General section}

\begin{description}
\item[auth]
\item[logging]
\item[tempdir]
\item[language]
\item[log\_level]
\item[DBI\_TRACE]
\item[pathsep]
\item[latex]
\item[check\_max\_invoices]
\item[max\_post\_size]
\item[decimal\_places]
\item[cookie\_name]
\item[no\_db\_str]

\end{description}

\subsubsection{'environment' section}

\begin{description}
\item[PATH]
\end{description}


\subsubsection{'paths' section}

\begin{description}
\item [spool]
\item [userspath]
\item [templates]
\item [images]
\item [memberfile]
\item [localepath]
\end{description}

\subsubsection{'programs' section}

\begin{description}
\item [gzip]
\end{description}


\subsubsection{'mail' section}

\subsubsection{'printers' section}

\subsubsection{'database' section}



\subsection{pos.pl.conf}

@@@ no idea what should go in here. to be investigated.

\subsection{templates}

temlates are global, however, set is selectable per company.

\subsection{ledgersmb.css}

\chapter{Per company configuration}

\section{Administrative user}
\section{Chart of accounts}
\subsection{Special accounts}
\begin{itemize}
\item AR/AP summary accounts
\item 5 other special purpose accounts, see ``Defaults'' screen discussion
\item sales tax accounts
\end{itemize}


\section{System menu settings}

This section enumerates the ``System'' menu's immediate children. In some cases the
functionality is too complex and is referred to a chapter of its own.

\subsection{Audit control}

\subsubsection{Enforce transaction reversal for all dates}

This is a Yes/No value which affects the actions which can be performed on posted financial transactions.
\begin{itemize}
\item No means transactions can be altered or deleted, even after posting them. Note that
if a transaction has been posted before the latest closing date, it can never be altered,
not even when this value is in effect.
\item Yes means transactions can't be altered after posting. This setting is highly preferred and considered the only correct approach to accounting as it assures visible
audit trails and thereby supports fraud detection.
\end{itemize}

\subsubsection{Close books up to}

@@@ This item isn't a system setting; shouldn't it move to ``Transaction approval''?? That way system settings (config) and processes are separated.

It's advisable to regularly close the books after review. This prevents user error changing
reviewed numbers: after closing the books, it's no longer possible to post in the closed
period.

There are also performance benefits to closing the books, because LedgerSMB uses the
fact that the figures are known-stable as a performance optimization when calculating
account balances.

\subsubsection{Activate audit trail}

This is a Yes/No value which - when Yes - causes the system to install triggers to register
user actions (creation/adjustments/reversals/etc...) executed on financial transactions.


@@@ Once activated, where can we see it the audit trail??


\subsection{Taxes}

This page lists all accounts which have the ``Tax'' account option enabled as discussed in section \ref{sec:AccountOptions}.

Each account is listed at least once, but can be listed many times, if it has had different
settings applied over different time periods. E.g. if one of the current VAT rates is 19\%,
today but it used to be 17.5\% until last month, there will be 2 rows for the applicable
VAT account. See chapter \ref{cha:Taxes} for further discussion of how taxes work in
LedgerSMB and the choices involved when being required to handle changes in tax rates.

Each row lists the following fields:

\begin{description}
\item [Rate (\%)] The tax rate to be applied when calculating VAT to be posted on this account.
\item [Number] @@@ No idea? VAT number??
\item [Valid To] The ending date of the settings in this row. This can apply to the rate as well as the ordering or the tax rules (but usually applies to the rate).
\item [Ordering] @@@ Has to do with multiple tax rows being applicable, but what exactly is the effect of this setting?
\item [Tax rules] LedgerSMB features an extendable structure to facilitate complex tax
calculations (see section \ref{sec:TaxRulePlugins}). By default the ``Simple'' module
is the only one installed.
\end{description}

\subsection{Defaults}

\subsubsection{Business number}
   @@@ No idea what this should do...
   
\subsubsection{Weight unit}
   The unit of measurement for weights. @@@ why don't we have a unit of measurement for distance as well??? And maybe a unit of measurement for content?
   
\subsubsection{Separation of duties}
   Set this field to 1 if you want to activate separation of duties or to 0 if you don't
   want to enforce it. Separation of duties is explained in section \ref{sec:SeparationOfDuties}. @@@ Should this config option not be in the ``Audit Control'' section???

\subsubsection{Default accounts}

 This setting will be used to pre-select an account in
the account listings of the three categories listed below:
\begin{itemize}
\item Inventory
\item Income
\item Expense
\end{itemize}


\subsubsection{Foreign exchange gain and loss accounts}

When working with foreign currencies,
the system needs two special purpose accounts. One to post the gains onto which are
caused by foreign currencies increasing in value; the other to post the losses onto
which are caused by foreign currencies decreasing in value.


\subsubsection{Default country}

This setting indicates which country needs to be pre-selected
   in country selection lists.


\subsubsection{Default language}

The language to be used when no other language has been selected. Several parts of the
application require language selection, such as customer, vendor and employee entry screens.

\subsubsection{Templates directory}

This setting indicates which set of templates - stored in the
   \texttt{templates/} directory - should be used. In a standard installation, the drop down
   lists two items:
   \begin{itemize}
   \item [demo] which contains templates based on \LaTeX, which is more commonly installed but has issues dealing with accented characters
   \item [xedemo] which contains the same templates based on TeTeX, which handles UTF-8 input (and thus accented characters) much better than \LaTeX and is broadly available, but not usually pre-installed.
   \end{itemize}


\subsubsection{List of currencies \& default currency}

Enter a list of all currencies you want
to use in your company, identified by their 3-letter codes separated by a colon; i.e.
``USD:EUR:CHF''. To ensure correct operation of the application, at least one currency
(the company default currency) must be listed. In case of multiple currencies the first
is used as the company default currency.
\subsubsection{Company data (name /address)}

The fields ``Company Name'', ``Company Address'',
``Company Phone'' and ``Company Fax'' will be used on printed/e-mailed invoices.

\subsubsection{Password duration} This field indicates the duration passwords will be considered valid. Freshly set passwords will expire after this period.\footnote{If this value isn't set, passwords will expire after 1 year by default.} Note that passwords for freshly
created users expire within 24 hours after creation. @@@ What's the input here? A number? Would that number be days, weeks, years?


\subsubsection{Default Email addresses}

These addresses will be used to send e-mails from the system.
Note that the ``Default Email From'' address should be configured in order to make sure
e-mail doesn't look like it's coming from your webserver. The format to be used is \texttt{``Name'' <e-mail address>} where the e-mail address should be inserted between the
``$\langle$'' and ``$\rangle$''.

\subsubsection{Max per dropdown}

Some elements in the screens may present a drop down. However, drop downs are
relatively unwieldy to work with when used to present a large number of values
to choose from.

This configuration option sets an upper limit on the number of records to be
presented as drop down.  When the number is exceeded, no drop down is used.  Instead,
a multi-step selection procedure will be used.

\subsubsection{Item numbering}

Many items in the system have sequence numbers: invoices, transactions, parts, etc. These sequence numbers can be just a number (i.e. 1 or 37),
but they can also be prefixed numbers, e.g. INV0001 for invoices and EMP001 for employees.
The numbers shown in the input boxes will be used to generate the next number in the
numbering sequence.
\begin{itemize}
\item [GL Reference number] @@@ where is this number being used??
\item [Sales invoice/ AR Transaction number] This number is used to generate an invoice
number when none is being filled out by the user.
\item [Sales order number] Same as Sales invoice number, except that it's used for sales orders @@@ layout issue: the label is too big to fit on the page
\item [Vendor invoice/ AP Transaction number] Same as Sales invoice, except that the number
is used for accounts payable transactions. @@@ layout issue: the label is too big to fit on the page
\item [Sales quotation number] Same as sales order number, except that it's used for quotations.
\item [RFQ number] Request for quotation number is like the sales quotation number, except
that it is used to track which vendors have been asked for quotes.
\item [Part number] All parts, services and assemblies are identified by a unique number.
When an item is created and no number is entered by the user, a number is generated
from this sequence.
\item [Job/project number] Used when creating new projects.
\item [Employee number] Same as the sales invoice number, used by new employee entry.
\item [Customer number] @@@ is this the control code number? or is this meta\_number??
\item [Vendor number] @@@ same question as customer number
\end{itemize}

\subsubsection{Check prefix} The prefix to use when printing checks. There's no check sequence number. That sequence number is requested from the check printing interface, because
checks can be created outside the application as well, meaning the numbers can
get out of sync.

\subsection{Year end}

@@@ Rename ``Yearend'' in menu interface to ``Year end''.


@@@ IMO this section doesn't belong here, because it's a process, not config, but does it belong in this menu then? IMO it doesn't...


\subsection{Admin users}

@@@ Same as Year end; doesn't belong here...

\subsection{Chart of accounts}

@@@ Chart of accounts isn't exactly a ``process'', but it doesn't feel like being pure
config either. At any rate it's a fact that the CoA discussion is a full chapter in and
of itself - so discussion here isn't necessary anymore.

\subsection{Warehouses}

Warehouses are stocking locations. They don't have any properties (in the system)
other than that they have a name. Warehouses can be added, modified and deleted from
the System $\rightarrow$ Warehouses menu item.

\subsection{Departments}

Departments can be used to divide a company in smaller pieces. LedgerSMB distinguishes two
types of departments:

\begin{itemize}
\item [Profit centers] which can be associated with any type of transaction, including AR transactions.
\item [Cost centers] which can be associated with all types of transactions, except AR transactions.
\end{itemize}

Departments can be created (added), modified or deleted through the System $\rightarrow$ Departments menu item.

\subsection{Type of business}

Types of business are used in sales operations where customers can be assigned a type
of business. Based on the type of business assignment, quotations, sales orders and
invoices will automatically apply discount rates. For each type of business you enter a description and a discount rate to be applied.

\subsection{Languages}

The language table is the table users can select languages from, both to present
the UI of the application as well as the setting for customers to be used to generate
documents.

This listing should correspond to the actual translations of the application being
available in the program installation directory.

Languages can be added, modified or deleted through the System $\rightarrow$ Language menu item.

\subsection{Standard Industry Code (SIC)}

SI codes feature these three fields:

\begin{description}
\item [Code]
\item [Heading]
\item [Description]
\end{description}

When creating a company (irrespective of whether that company will be assigned a customer
or vendor role) you can assign that company an SI code.

@@@ These SI codes may affect Tax and regulatory reports?? What do they ``do''?

\subsection{HTML templates}

\subsection{LaTeX templates}

\subsection{Text templates}



\part{Administration}
\label{part:Administration}

%\section{Introduction}
%This section of the book describes which tasks and processes might need to be carried out
%on a regular basis in order to keep the application in good health and in line with
%end-users requirements.  As a result this part is more related to the functional rather
%than technical parts of the application.
%
%The fact that the tasks described in this part of the book may be recurring during the
%lifetime of the application does not exclude them from being part of the setup phase
%of LedgerSMB.  In other words: you're likely to have to dig into this section when
%creating a company as well as when maintaining it.

\chapter{Overview}


(Within-application tasks)

(DBA tasks from setup.pl)

(Outside-application tasks)

\chapter{User management}

\section{User creation}

\section{User authorization}

\section{Same user in multiple companies}

\chapter{Definition of products and services}

\section{Definition of products}
Structure of products in the system.

\subsection{Definition of parts}
\label{sec:DefinitionOfParts}

\subsection{Definition of partgroups}
\subsection{Definition of assemblies}
\subsection{Definition of overhead}

\section{Definition of services}

\chapter{Chart of accounts}

\section{Accounts and headers}

The system allows ordering accounts into groups by assigning accounts to headers. Headers
can themselves be assigned to other headers resulting in trees of account groups.\footnote{Although the database structure supports this type of account hierarchy
doesn't the 1.3 user take advantage of it yet: in 1.3 accounts can be assigned a header,
but headers can't be assigned to headers themselves.}



\section{Account configuration}

Headers don't have any configuration, other than their number and description. Accounts also
have a number and description, but require additional configuration for the application to work
correctly. The settings are described in the sections that follow.

\subsection{Account options}
\label{sec:AccountOptions}
\begin{itemize}
\item Contra This checkmark identifies the account as a contra account, which means
   that the account is going to hold the opposite of an account it's associated with.
   A good example of this kind would be the depreciation account associated with a fixed
   asset account where the depreciation account contains the credit amount to be added to
   the original asset (debit) value to get the current asset value.
\item Recon This checkmark identifies the account as one which needs reconciliation as
   described in section \ref{sec:Reconciliation}.
\item Tax This checkmark identifies the account as a Tax (VAT) account. Tax accounts need
   to be further configured. See chapter \ref{cha:Taxes} for further discussion of the
   subject.
\end{itemize}

\subsection{Summary accounts}

There are currently three types of summary accounts:

\begin{enumerate}
\item AR Marking an account as a summary account for AR means that all outstanding
   receivable amounts will be posted to this account. The Accounts Receivable administration
   will contain the details of which amount is owed by which customer.
\item AP Same as the AR account, except for amounts owed to vendors.
\item Inventory This account holds the monetary value equal to the items on stock.
\end{enumerate}

\subsection{Account inclusion in drop down lists}

@@@ Add summary

\subsubsection{Receivables \& payables UI}

\begin{itemize}
\item[Income (AR\_amount)] ...
\item[Payment (AR\_paid)] This check mark adds the account to the list of accounts
   to choose from in the Receipts (AR) and Payments (AP) screens. Additionally, it
   adds the account to the part entry screen as described in section \ref{sec:DefinitionOfParts}.
\item[Tax (AR\_tax)] This check mark makes the account show up as a check mark on the
   customer (AR) or vendor (AP) entry screen. See chapter \ref{cha:Taxes} for further discussion.
\item[Overpayment (AR\_overpayment)] Adds the account to the receipts screen as discussed
   in section \ref{sec:UsingOverpayments}.
\item[Discount (AR\_discount)] Adds the account to the customer entry screen's selection
   list for accounts to post 
\end{itemize}

The payables UI works the same way as does the receivables UI. The difference is
that the technical names of the configuration identifiers are prefixed by AP\_ instead
of AR\_.

\subsubsection{Tracking Items}

The items on this line relate to stocked items, i.e. those tracked for inventory: parts and
assemblies.

\begin{enumerate}
\item[Income (IC\_sale)] Adds the account to the selection list of income accounts on the
   part and assembly definition screens.
\item[COGS (IC\_cogs)] Adds the account to the selection list of COGS @@@ accounts on the
   part, assembly and overhead definition screen.
\item[Tax (IC\_taxpart)] Adds a check mark to the part and assembly definition screen
   for the applicable account. See \ref{cha:Taxes} for more details on how taxes
   work in LedgerSMB.
\end{enumerate}

@@@ Question: Labor/Overhead accounts == inventory accounts??

\subsubsection{Non-tracking items}

The items on this line relate to untracked (non stocked) items, i.e. services.

\begin{enumerate}
\item[Income (IC\_income)] Adds the account to the income account selection list in
   the service definition screen.
\item[Expense (IC\_expense)] Adds the account to the expense account selection list in
   the service definition screen.
\item[Tax (IC\_taxservice)] Adds a check mark to the service definition screen for the
   applicable account. See \ref{cha:Taxes} for more details on how taxes work in LedgerSMB.
\end{enumerate}

\subsubsection{Fixed assets}

\begin{enumerate}
\item[Fixed asset (Fixed\_Asset)] Marks the account as holding the original asset value for the fixed
   assets module, for some classes of fixed assets.
\item[Depreciation (Asset\_Dep)] Marks the account as holding the cumulative depreciation amount
   for the fixed assets module, for some classes of fixed assets.
\item[Expense (asset\_expense)] Adds the expense account to the selection list of the fixed assets
   accounting module. See section \ref{sec:FixedAssetAccounting} for more details.
\item[Gain (asset\_gain)] Account to hold book value gain upon disposal of a fixed asset.
\item[Loss (asset\_loss)] Account to hold book value loss upon disposal of a fixed asset.
\end{enumerate}


\section{Alternative accounts (GIFI)}

Next to the regular account numbering scheme, LedgerSMB supports a second numbering scheme: GIFI numbering. The GIFI accounts are a kind of secondary numbering scheme to support legal requirements.

Some jurisdictions require a specific numbering scheme, which can be supported using GIFI. If you
use GIFI account numbers, each account is associated with a GIFI account. Multiple accounts may map
to a single GIFI account.

Many General Ledger reports exist in two variants: a variant using the normal G/L accounts and
one with the GIFI numbering scheme. In the GIFI variant, when a single GIFI has multiple accounts,
the total reported under GIFI is the sum of the mapped accounts.


\subsection{Maintaining GIFI}

GIFI accounts should be created before being assigned to a standard G/L account. GIFI accounts
can be maintained through the System $\rightarrow$ Chart of accounts $\rightarrow$ Add GIFI and List GIFI menu items. Existing accounts can be edited by selecting them from the List GIFI menu, which opens a page where individual GIFI items can have their number or
description adjusted.


\chapter{Taxes}
\label{cha:Taxes}

\section{Overview}



\section{Tax calculation plug-ins}
\label{sec:TaxRulePlugins}

\chapter{Pricing}

\section{Definition of types of business}

\section{Definition of price groups}

\chapter{contingency planning}

\section{Backup and restore}

\section{Advanced PostgreSQL: replication}

\chapter{Software updates}

\section{LedgerSMB patch release roll out}





\part{Workflows}
\label{part:Workflows}

\chapter{Customers and vendors}

\section{Creating customers and vendors}

\section{Multiple customers within one company}

\section{Creating vendors from customers}

\section{}

\section{Maintaining contact information}

\section{Credit management}


\chapter{Quotations from Vendors and for Customers}

\section{Creating Quotations and RFQs}

\chapter{Sales and vendor orders}

\section{Creating new orders}

\section{Creating orders from quotations}

\section{Creating orders from projects}

\section{Creating purchase orders from sales orders}

\section{Combining orders}

\section{Recurring orders}


\chapter{Sales and vendor invoices}

\section{Creating new invoices}

--> this section is about Sales and Vendor Invoices

\section{Creating invoices from orders}

\chapter{Shop sales}

\section{Opening and closing the cash register}

\section{Shop sales invoices}

\chapter{Production management}

\section{Producing sales orders}

\subsection{Work orders}


\chapter{Inventory management}

\section{Shipping}

\subsection{Pick lists}
\subsection{Packing list}

\section{Receiving}

\section{Partial shipments}

\section{Transferring between warehouses}

\section{Inventory reporting}



\chapter{Accounts receivable and payable}

\section{Creating generic AR/AP items}

\section{Handling refunds, overpayments and advances}

- this bit is about credit notes and debit notes

\section{Handling returns}

--> this bit is about credit (sales) and debit (vendor) invoices

\section{Managing arrears}

\section{Interest on arrears}

\section{Allowance for doubtful accounts}

\section{Writing off bad debt}

\subsection{Direct write-off}

\subsection{Allowed-for write-off}


\chapter{Receipts and payment processing}

\section{Check payments}

\section{Single receipts and payments}

\section{Batch receipts and payments}

\section{Using overpayments}
\label{sec:UsingOverpayments}

\section{Receipt and payment reversal}

\section{Receipts and payments in foreign currencies}

\section{Reconciliation}
\label{sec:Reconciliation}

\chapter{Accounting}

\section{Separation of duties: Transaction approval}
\label{sec:SeparationOfDuties}

\section{Period closing}

\section{Year-end processing}

\section{Tax (VAT) reporting}

\begin{itemize}
\item 1099
\item EU VAT
\end{itemize}

\section{Entering general accounting documents}

\section{Fixed asset accounting}
\label{sec:FixedAssetAccounting}

\section{Reporting}
\subsection{Income statement}
\subsection{Balance sheet}
\subsection{Trial balance}



\part{Customization}
\label{part:Customization}

\chapter{Batch data import methods}

\section{Custom bank statement import}



\part{Appendices}
\appendix

\chapter{Listing of application roles}

\chapter{Copyright and license}

Copyright (c) 2011 Erik H\"ulsmann.


This work is licensed under the Creative Commons Attribution License.
To view a copy of this license, visit http://creativecommons.org/licenses/by/3.0/
or send a letter to Creative Commons, 559 Nathan Abbott Way,
Stanford, California 94305, USA.

A summary of the license is given below, followed by the full legal text.

\section{License summary}

\begin{verbatim}

You are free:
  * to share -- to copy, distribute and transmit the work
  * to remix -- to adapt the work


Under the following condition:
  You must attribute he work in the manner specified by the author or licensor
  (but in a way that suggests that they endorse you or your use of the work).


With the understanding that:
  Waiver -- Any of the above conditions can be waived if you get permission
            from the copyright holder.
  
  Other rights -- In no way are any of the following rights affected by the license:
     * Your fair dealing or fair use rights, or other applicable 
         copyright exceptions and limitations;
     * The author's moral rights;
     * Rights other persons may have either in the work itself or
         in how the work is used, such as publicity or privacy rights.



\end{verbatim}



\section{Legal full text}

\begin{verbatim}
License

THE WORK (AS DEFINED BELOW) IS PROVIDED UNDER THE TERMS OF THIS CREATIVE COMMONS PUBLIC LICENSE ("CCPL" OR "LICENSE"). THE WORK IS PROTECTED BY COPYRIGHT AND/OR OTHER APPLICABLE LAW. ANY USE OF THE WORK OTHER THAN AS AUTHORIZED UNDER THIS LICENSE OR COPYRIGHT LAW IS PROHIBITED.

BY EXERCISING ANY RIGHTS TO THE WORK PROVIDED HERE, YOU ACCEPT AND AGREE TO BE BOUND BY THE TERMS OF THIS LICENSE. TO THE EXTENT THIS LICENSE MAY BE CONSIDERED TO BE A CONTRACT, THE LICENSOR GRANTS YOU THE RIGHTS CONTAINED HERE IN CONSIDERATION OF YOUR ACCEPTANCE OF SUCH TERMS AND CONDITIONS.

1. Definitions

"Adaptation" means a work based upon the Work, or upon the Work and other pre-existing works, such as a translation, adaptation, derivative work, arrangement of music or other alterations of a literary or artistic work, or phonogram or performance and includes cinematographic adaptations or any other form in which the Work may be recast, transformed, or adapted including in any form recognizably derived from the original, except that a work that constitutes a Collection will not be considered an Adaptation for the purpose of this License. For the avoidance of doubt, where the Work is a musical work, performance or phonogram, the synchronization of the Work in timed-relation with a moving image ("synching") will be considered an Adaptation for the purpose of this License.
"Collection" means a collection of literary or artistic works, such as encyclopedias and anthologies, or performances, phonograms or broadcasts, or other works or subject matter other than works listed in Section 1(f) below, which, by reason of the selection and arrangement of their contents, constitute intellectual creations, in which the Work is included in its entirety in unmodified form along with one or more other contributions, each constituting separate and independent works in themselves, which together are assembled into a collective whole. A work that constitutes a Collection will not be considered an Adaptation (as defined above) for the purposes of this License.
"Distribute" means to make available to the public the original and copies of the Work or Adaptation, as appropriate, through sale or other transfer of ownership.
"Licensor" means the individual, individuals, entity or entities that offer(s) the Work under the terms of this License.
"Original Author" means, in the case of a literary or artistic work, the individual, individuals, entity or entities who created the Work or if no individual or entity can be identified, the publisher; and in addition (i) in the case of a performance the actors, singers, musicians, dancers, and other persons who act, sing, deliver, declaim, play in, interpret or otherwise perform literary or artistic works or expressions of folklore; (ii) in the case of a phonogram the producer being the person or legal entity who first fixes the sounds of a performance or other sounds; and, (iii) in the case of broadcasts, the organization that transmits the broadcast.
"Work" means the literary and/or artistic work offered under the terms of this License including without limitation any production in the literary, scientific and artistic domain, whatever may be the mode or form of its expression including digital form, such as a book, pamphlet and other writing; a lecture, address, sermon or other work of the same nature; a dramatic or dramatico-musical work; a choreographic work or entertainment in dumb show; a musical composition with or without words; a cinematographic work to which are assimilated works expressed by a process analogous to cinematography; a work of drawing, painting, architecture, sculpture, engraving or lithography; a photographic work to which are assimilated works expressed by a process analogous to photography; a work of applied art; an illustration, map, plan, sketch or three-dimensional work relative to geography, topography, architecture or science; a performance; a broadcast; a phonogram; a compilation of data to the extent it is protected as a copyrightable work; or a work performed by a variety or circus performer to the extent it is not otherwise considered a literary or artistic work.
"You" means an individual or entity exercising rights under this License who has not previously violated the terms of this License with respect to the Work, or who has received express permission from the Licensor to exercise rights under this License despite a previous violation.
"Publicly Perform" means to perform public recitations of the Work and to communicate to the public those public recitations, by any means or process, including by wire or wireless means or public digital performances; to make available to the public Works in such a way that members of the public may access these Works from a place and at a place individually chosen by them; to perform the Work to the public by any means or process and the communication to the public of the performances of the Work, including by public digital performance; to broadcast and rebroadcast the Work by any means including signs, sounds or images.
"Reproduce" means to make copies of the Work by any means including without limitation by sound or visual recordings and the right of fixation and reproducing fixations of the Work, including storage of a protected performance or phonogram in digital form or other electronic medium.
2. Fair Dealing Rights. Nothing in this License is intended to reduce, limit, or restrict any uses free from copyright or rights arising from limitations or exceptions that are provided for in connection with the copyright protection under copyright law or other applicable laws.

3. License Grant. Subject to the terms and conditions of this License, Licensor hereby grants You a worldwide, royalty-free, non-exclusive, perpetual (for the duration of the applicable copyright) license to exercise the rights in the Work as stated below:

to Reproduce the Work, to incorporate the Work into one or more Collections, and to Reproduce the Work as incorporated in the Collections;
to create and Reproduce Adaptations provided that any such Adaptation, including any translation in any medium, takes reasonable steps to clearly label, demarcate or otherwise identify that changes were made to the original Work. For example, a translation could be marked "The original work was translated from English to Spanish," or a modification could indicate "The original work has been modified.";
to Distribute and Publicly Perform the Work including as incorporated in Collections; and,
to Distribute and Publicly Perform Adaptations.
For the avoidance of doubt:

Non-waivable Compulsory License Schemes. In those jurisdictions in which the right to collect royalties through any statutory or compulsory licensing scheme cannot be waived, the Licensor reserves the exclusive right to collect such royalties for any exercise by You of the rights granted under this License;
Waivable Compulsory License Schemes. In those jurisdictions in which the right to collect royalties through any statutory or compulsory licensing scheme can be waived, the Licensor waives the exclusive right to collect such royalties for any exercise by You of the rights granted under this License; and,
Voluntary License Schemes. The Licensor waives the right to collect royalties, whether individually or, in the event that the Licensor is a member of a collecting society that administers voluntary licensing schemes, via that society, from any exercise by You of the rights granted under this License.
The above rights may be exercised in all media and formats whether now known or hereafter devised. The above rights include the right to make such modifications as are technically necessary to exercise the rights in other media and formats. Subject to Section 8(f), all rights not expressly granted by Licensor are hereby reserved.

4. Restrictions. The license granted in Section 3 above is expressly made subject to and limited by the following restrictions:

You may Distribute or Publicly Perform the Work only under the terms of this License. You must include a copy of, or the Uniform Resource Identifier (URI) for, this License with every copy of the Work You Distribute or Publicly Perform. You may not offer or impose any terms on the Work that restrict the terms of this License or the ability of the recipient of the Work to exercise the rights granted to that recipient under the terms of the License. You may not sublicense the Work. You must keep intact all notices that refer to this License and to the disclaimer of warranties with every copy of the Work You Distribute or Publicly Perform. When You Distribute or Publicly Perform the Work, You may not impose any effective technological measures on the Work that restrict the ability of a recipient of the Work from You to exercise the rights granted to that recipient under the terms of the License. This Section 4(a) applies to the Work as incorporated in a Collection, but this does not require the Collection apart from the Work itself to be made subject to the terms of this License. If You create a Collection, upon notice from any Licensor You must, to the extent practicable, remove from the Collection any credit as required by Section 4(b), as requested. If You create an Adaptation, upon notice from any Licensor You must, to the extent practicable, remove from the Adaptation any credit as required by Section 4(b), as requested.
If You Distribute, or Publicly Perform the Work or any Adaptations or Collections, You must, unless a request has been made pursuant to Section 4(a), keep intact all copyright notices for the Work and provide, reasonable to the medium or means You are utilizing: (i) the name of the Original Author (or pseudonym, if applicable) if supplied, and/or if the Original Author and/or Licensor designate another party or parties (e.g., a sponsor institute, publishing entity, journal) for attribution ("Attribution Parties") in Licensor's copyright notice, terms of service or by other reasonable means, the name of such party or parties; (ii) the title of the Work if supplied; (iii) to the extent reasonably practicable, the URI, if any, that Licensor specifies to be associated with the Work, unless such URI does not refer to the copyright notice or licensing information for the Work; and (iv) , consistent with Section 3(b), in the case of an Adaptation, a credit identifying the use of the Work in the Adaptation (e.g., "French translation of the Work by Original Author," or "Screenplay based on original Work by Original Author"). The credit required by this Section 4 (b) may be implemented in any reasonable manner; provided, however, that in the case of a Adaptation or Collection, at a minimum such credit will appear, if a credit for all contributing authors of the Adaptation or Collection appears, then as part of these credits and in a manner at least as prominent as the credits for the other contributing authors. For the avoidance of doubt, You may only use the credit required by this Section for the purpose of attribution in the manner set out above and, by exercising Your rights under this License, You may not implicitly or explicitly assert or imply any connection with, sponsorship or endorsement by the Original Author, Licensor and/or Attribution Parties, as appropriate, of You or Your use of the Work, without the separate, express prior written permission of the Original Author, Licensor and/or Attribution Parties.
Except as otherwise agreed in writing by the Licensor or as may be otherwise permitted by applicable law, if You Reproduce, Distribute or Publicly Perform the Work either by itself or as part of any Adaptations or Collections, You must not distort, mutilate, modify or take other derogatory action in relation to the Work which would be prejudicial to the Original Author's honor or reputation. Licensor agrees that in those jurisdictions (e.g. Japan), in which any exercise of the right granted in Section 3(b) of this License (the right to make Adaptations) would be deemed to be a distortion, mutilation, modification or other derogatory action prejudicial to the Original Author's honor and reputation, the Licensor will waive or not assert, as appropriate, this Section, to the fullest extent permitted by the applicable national law, to enable You to reasonably exercise Your right under Section 3(b) of this License (right to make Adaptations) but not otherwise.
5. Representations, Warranties and Disclaimer

UNLESS OTHERWISE MUTUALLY AGREED TO BY THE PARTIES IN WRITING, LICENSOR OFFERS THE WORK AS-IS AND MAKES NO REPRESENTATIONS OR WARRANTIES OF ANY KIND CONCERNING THE WORK, EXPRESS, IMPLIED, STATUTORY OR OTHERWISE, INCLUDING, WITHOUT LIMITATION, WARRANTIES OF TITLE, MERCHANTIBILITY, FITNESS FOR A PARTICULAR PURPOSE, NONINFRINGEMENT, OR THE ABSENCE OF LATENT OR OTHER DEFECTS, ACCURACY, OR THE PRESENCE OF ABSENCE OF ERRORS, WHETHER OR NOT DISCOVERABLE. SOME JURISDICTIONS DO NOT ALLOW THE EXCLUSION OF IMPLIED WARRANTIES, SO SUCH EXCLUSION MAY NOT APPLY TO YOU.

6. Limitation on Liability. EXCEPT TO THE EXTENT REQUIRED BY APPLICABLE LAW, IN NO EVENT WILL LICENSOR BE LIABLE TO YOU ON ANY LEGAL THEORY FOR ANY SPECIAL, INCIDENTAL, CONSEQUENTIAL, PUNITIVE OR EXEMPLARY DAMAGES ARISING OUT OF THIS LICENSE OR THE USE OF THE WORK, EVEN IF LICENSOR HAS BEEN ADVISED OF THE POSSIBILITY OF SUCH DAMAGES.

7. Termination

This License and the rights granted hereunder will terminate automatically upon any breach by You of the terms of this License. Individuals or entities who have received Adaptations or Collections from You under this License, however, will not have their licenses terminated provided such individuals or entities remain in full compliance with those licenses. Sections 1, 2, 5, 6, 7, and 8 will survive any termination of this License.
Subject to the above terms and conditions, the license granted here is perpetual (for the duration of the applicable copyright in the Work). Notwithstanding the above, Licensor reserves the right to release the Work under different license terms or to stop distributing the Work at any time; provided, however that any such election will not serve to withdraw this License (or any other license that has been, or is required to be, granted under the terms of this License), and this License will continue in full force and effect unless terminated as stated above.
8. Miscellaneous

Each time You Distribute or Publicly Perform the Work or a Collection, the Licensor offers to the recipient a license to the Work on the same terms and conditions as the license granted to You under this License.
Each time You Distribute or Publicly Perform an Adaptation, Licensor offers to the recipient a license to the original Work on the same terms and conditions as the license granted to You under this License.
If any provision of this License is invalid or unenforceable under applicable law, it shall not affect the validity or enforceability of the remainder of the terms of this License, and without further action by the parties to this agreement, such provision shall be reformed to the minimum extent necessary to make such provision valid and enforceable.
No term or provision of this License shall be deemed waived and no breach consented to unless such waiver or consent shall be in writing and signed by the party to be charged with such waiver or consent.
This License constitutes the entire agreement between the parties with respect to the Work licensed here. There are no understandings, agreements or representations with respect to the Work not specified here. Licensor shall not be bound by any additional provisions that may appear in any communication from You. This License may not be modified without the mutual written agreement of the Licensor and You.
The rights granted under, and the subject matter referenced, in this License were drafted utilizing the terminology of the Berne Convention for the Protection of Literary and Artistic Works (as amended on September 28, 1979), the Rome Convention of 1961, the WIPO Copyright Treaty of 1996, the WIPO Performances and Phonograms Treaty of 1996 and the Universal Copyright Convention (as revised on July 24, 1971). These rights and subject matter take effect in the relevant jurisdiction in which the License terms are sought to be enforced according to the corresponding provisions of the implementation of those treaty provisions in the applicable national law. If the standard suite of rights granted under applicable copyright law includes additional rights not granted under this License, such additional rights are deemed to be included in the License; this License is not intended to restrict the license of any rights under applicable law.
\end{verbatim}



\end{document}

