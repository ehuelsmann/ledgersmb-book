

\part{Getting started}
\label{part:GettingStarted}

\chapter{Overview}

\section{Introduction}

This part of the book will run the reader through the LedgerSMB using an example
startup company run by Jack: JaCo Inc, which starts its life as a computer parts
store for the business to business market.

Jack just completed incorporation of JaCo and is ready to start doing business.
Before starting his operation Jack decides to look for tooling to run his operation
efficiently.

The other chapters in this part of the book show you what steps Jack has to go through
to get LedgerSMB up and running for JaCo, but also which steps he has to take to
keep LedgerSMB in good health.

Due to its success JaCo will grow, posing new challenges to LedgerSMB and we'll show
you how Jack can change the configuration to allow LedgerSMB to grow with his business.

It's outside the scope of this part to explain how to install the application. See
Part \ref{part:Configuration} for information on installation. \refcha{cha:CompanyCreation}
starts when the software has been installed yet nothing more has been done.

\section{LedgerSMB, because...}

Jack finds several tools which suit his requirements to some extent or another.
After evaluation of his options he decides to use LedgerSMB for the following reasons:

\begin{itemize}
\item Centralized data storage
\item Actively developed
\item Development team with security focus
\item Access to the application requires only a web browser
\item Integrated sales, shipping, invoicing, purchasing and accounting
\item Open source solution, so no vendor lock in
\item @@@ others?
\end{itemize}


\chapter{Creating a company administration}
\label{cha:CompanyCreation}



\chapter{The first login}

\chapter{Building up stock}

\chapter{Ramping up to the first sale}

% sending out a quote followed by a sales order

\chapter{Shipping sales}

\chapter{Invoicing}

\chapter{Collecting sales invoice payments}

\section{Customer payments}

\section{Customer payment mismatch}

% choosing between pardonning and registering underpayment

% large ones, as in partial payments or largish under/over payments

% pardonning small mismatches


\chapter{Paying vendor invoices}

% handling vendors who match amounts to exact invoices

% handling vendors with running balances

% handling bounced checks: voiding checks to undo payments of vendor invoices
%   relating to bounced checks

\chapter{Monitoring arrears}

% handling interest on arrears

\chapter{Handling sales taxes}

% invoices with taxes included

% invoices with explicit tax amounts

% 

\chapter{Branching out: services}

% including creation / assignment to different accounts


\chapter{Recording service hours}

\chapter{Customer approval on service hours}

\chapter{Invoicing services}

\chapter{Branching out II: service subscriptions}


