% !TeX encoding = UTF-8
% !TeX spellcheck = en_US
% !TeX root = ledgersmb-book.tex

% Do not put index entries in the glossary (this file).
% This causes indexes to refer to the title page in both the PDF and HTML.
%
% The proper way to handle having a glossary entry in the both the
% glossary and index is to annotate the term with both gls and index.
%
% For example:
%		.... accounting year\gls{accounting year}\index{accounting year} ....
% In this case, the user can click on the index entry to get to the 'accounting year' text,
% then click again on this text to get to the glossary entry.

% There does not seem to be an easy way to put the glossary entry directly into the index.
 
\newglossaryentry{accounting year}{
	name={accounting year},
	description={An accounting year, which may also be called fiscal year, financial year or budget year
		is a twelve month period used for annual financial reporting.
		An accounting year may be the same as a calendar year, but most jurisdictions allow the accounting year to be
		defined differently from the calendar year}}

\newglossaryentry{accrual basis}{
    name={accrual basis},
    description={Income is deemed earned when the customer invoice is prepared and     sent, and the expenses 
        are deemed incurred when the purchase order is prepared and sent}}

\newglossaryentry{add-on}{
    name=add-on,
    description={Extension to the base LedgerSMB system to add new functionality},
    plural=add-ons}

\newglossaryentry{cash basis}{
    name={cash basis},
    description={
        Income is deemed earned when the customer pays it, and the expenses 
        are deemed incurred when the business pays them}}

\newglossaryentry{company}{
    name=company,
    description={A company is a contact initially setup as a company and usually has further configuration as a vendor or customer}}

\newglossaryentry{contact}{
    name={contact},
    description={A contact is primarily a company or person.
        The contact can be further idenified as employee, vendor, customer, contact, lead, referral, hot lead, cold lead, sub-contractor, etc. 
        The contact's setup determines how it is used in LedgerSMB}}

\newglossaryentry{contra}{
    name={contra},
    description={
        A contra account is used in a general ledger to reduce the value of a related account when the two are netted together. 
        A contra account's natural balance is the opposite of the associated account. 
        If a debit is the natural balance recorded in the related account, the contra account records a credit}}

\newglossaryentry{counterparty}{
    name={counterparty},
    description={A legal entity, unincorporated entity, or collection of entities to which an exposure of financial risk may exist.
        In LSMB, a counterparty, is the same as an entity, and can be either a company entity (suppliers, vendors, customers), 
        or a person entity (including an employee)},
    plural={counterparties}}

\newglossaryentry{credit account}{
    name={credit account},
    description={The link between contacts and the accounting system. 
        Credit accounts can be setup to identify contacts as vendors, customers, or both. 
        When setup as both there may be different identifying numbers for the customer and vendor. 
        The term ``credit account'' is a generalization of the customer and vendor accounts},
    plural={credit accounts}}

\newglossaryentry{credit limit}{
    name={credit limit},
    description={Maximum amount of open invoices and orders to be allowed a credit
        entity account. Used to limit credit risk}}

 \newglossaryentry{customer}{
    name=customer,
    description={Is a contact initially setup as a company or person and further configured with at least one customer credit account},
    plural=customers}

\newglossaryentry{diacritic character}{
    name={diacritic character},
    description={The various little dots and squiggles which, in many languages, are written above, below or on top of 
        certain letters of the alphabet to indicate something about their pronunciation},
    plural={diacritic characters}}

\newglossaryentry{entity}{
    name={entity},
    description={A legal entity, unincorporated entity, or collection of entities to which an exposure of financial risk may exist.
        In LSMB, an entity, is the same as a counterparty, and can be either a company entity (suppliers, vendors, customers) or 
        a person entity (including an employee)},
    plural={counterparties}}

\newglossaryentry{expense}{
    name={expense},
    plural={expenses},
    description={
        A cost incurred by the company. Expenses are not payments. The most prominent
        example is where products are sold and the cost of inventory is recognized as
        expense (Cost Of Goods Sold) at the time of the sale while the products could
        have been bought months before}}

\newglossaryentry{gifi}{
    name={GIFI},
    description={Stands for Generalized Index for Financial Information,
        a system used by the Canada Revenue Agency for filing corporate
        tax returns (see \url{http://www.cra-arc.gc.ca/tx/bsnss/tpcs/crprtns/rtrn/wht/gifi-ogrf/menu-eng.html})}}

\newglossaryentry{ISO20022}{
    name={ISO-20022},
    description={ISO 20022 is a multi part International Standard prepared by ISO Technical Committee 
        TC68 Financial Services. It describes a common platform for the development of messages relating to the financial 
        services industry\footnote{From \url{https://iso20022.org/}}}}

\newglossaryentry{OFX}{
    name={OFX},
    description={Open Financial Exchange (OFX) is a data-stream format for exchanging financial information
         that evolved from Microsoft's Open Financial Connectivity (OFC) and Intuit's Open Exchange file 
         formats\footnote{From \url{https://en.wikipedia.org/wiki/Open_Financial_Exchange}}}}

\newglossaryentry{oldcode}{
    name={old code},
    description={Stands for the code inherited from SQL Ledger at the time
        of the fork.  This code stands to be replaced by ``Modern Perl''}}
 
\newglossaryentry{payment}{
    name={payment},
    description={A transfer of cash or cash equivalents out of the company. 
        Payments are not expenses: e.g. payments to pay off a loan}}
 
\newglossaryentry{permalink}{
    name={permalink},
    description={A permalink is a link which you can share with your colleagues and when inserted into the browser, 
        returns exactly the same report (or at least a search with exactly the same parameters as you used). 
        To access the link the user can use the browsers "Copy link" function to copy the link and pass it to a colleague},
    plural=permalinks}

\newglossaryentry{PL/SQL}{
    name={PL/SQL},
    description={Procedural Language for SQL, a programming language from Oracle for building database functionality. 
        LedgerSMB uses a variant supported by PostgreSQL called PL/pgSQL}}

\newglossaryentry{SPA}{
    name={Single Page Application},
    description={Single Page (web) Applications are a type of web applications; they employ JavaScript and HTML5
        techniques to retrieve data, code and layout fragments from the server as the UI needs them. 
        They are called ``single page'' because they use JavaScript to update the current page instead of loading a completely new page}}

\newglossaryentry{vendor}{
    name=vendor,
    description={Is a contact initially setup as a company or person and further configured with at least one vendor credit account},
    plural=vendors}
