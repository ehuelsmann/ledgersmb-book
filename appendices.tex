% !TeX spellcheck = en_US
% !TeX encoding = UTF-8
% !TeX root = ledgersmb-book.tex



\part{Appendices}
\label{part-appendices}

\appendix

\chapter{Differences between version 1.2 and 1.3}
\label{app-differences-12-13}

\section{Users}
\label{sec-diff12-13-users}

The way users are defined and used differs greatly between LedgerSMB 1.3 and
older versions. In version 1.3 user access to the database is enforced by the
database itself. This means that users logging in to the LedgerSMB web application
are in reality logging into the PostgreSQL database. In older versions, the web
app would verify the user's credentials (using a common database connection used
for all users).

The difference between these approaches is that security is no longer (solely)
maintained by the web application - with all inherent risks. Instead, the database
now plays an important role as well. The effect is that the LedgerSMB team now
leverages the experience of the PostgreSQL community - a highly respected community
well known for its security focus - to make sure your data stays secure.

This structure also enables LedgerSMB 1.3 to offer separation of duties and
authorizations throughout the application without being required to do a full
rewrite of the application.

It's this shift in paradigm that makes it impossible to meaningfully migrate
users from older LedgerSMB and SQL-Ledger versions to LedgerSMB 1.3.


\chapter{Why deleting invoices is not supported}
\label{app-no-invoice-deletion}

\section{Invoices are debt claims}
\label{sec-no-invoices-deletion-claims}

\section{Cost Of Goods Sold become ambiguous}
\label{sec-no-invoices-deletion-cogs-ambiguity}

From a mail by Chris Travers on the LedgerSMB mailing list:

\begin{quote}
Suppose I buy 10 widgets for \$1 each and then 1 more widget for \$10 each.  My FIFO cost queue looks like this:

\$1 \$1 \$1 \$1 \$1 \$1 \$1 \$1 \$1 \$1 \$10

My inventory account shows \$20 in debits and I have credited my AP account to compensate.

I then sell the 11 parts to 11 different people.

The first 10 invoice show a \$1 credit against inventory and a \$1 debit against COGS

The 11th invoice shows a \$10 credit against inventory and a \$10 credit against COGS 

and now inventory is down to 0.

Now the 5th customer invoice turns out to be in error. We never shipped this one. The customer never ordered it.  it was a data entry error.  Translation, we now have one in stock.

If we void the invoice properly, we will reverse the last sale, and put \$10 back in inventory. 

If we delete the invoice, we will just remove the \$1 removed.  But we don't really know which one was sold and so we de-allocate the \$10 sale.

So now our books are \$9 off of what they should be. They show a balance of \$1 in inventory, and \$19 in cogs. They should show \$10 in each.  The worse is still to come however.

Now we sell the item we had in stock.  This brings our (empty!) inventory value to -\$9 and our COGS value to \$29.  Our books are still \$9 off and we now have impossible, nonsensical values.  Delete and re-enter a few more invoices and you can inflate the COGS as far as you'd like.   This doesn't work.

Worse, this can't be fixed.  You can't make a deletion behave just like a reversal and still keep your books transparent auditability-wise.  Even if it could be fixed mathematically (which it can't), there isn't any agreement as to what the proper behavior should be (except 'don't do that').  So it isn't possible to support the workflow "properly" because "properly" can't be defined.

So unless someone can show that the above issues are incorrect, I don't see how we can support deleting invoices after they are posted to the books.

The alternative is the draft/voucher system which is supported in 1.3 for all non-inventory transactions and will be supported for inventory transactions in 1.4.  In this system, in the paper world, the clerk fills out a piece of paper with the information that will be entered as an invoice, and this is eventually gets entered and checked by someone else.  Both papers are kept in paper systems for reconciliation purposes but typically we tend to assume they are the same (this may be changing and we may be keeping both copies if they are changed in the future).  In this model, the voucher is not an invoice.  It is simply a piece of paper that represents what will be on the invoice.  It is subject to review, and may ultimately be denied.

So in this system we do *not* calculate extrinsic financial movements for documents until they post.  This includes COGS.  If a draft invoice or invoice voucher is deleted before it is approved it has none of the problems above.  Once approved though, it is a part of the permanent record.  This guards against data entry errors because a second person can review the data  before it is posted (either in bulk or individually).   Additionally this guards against theft by ensuring that a single individual cannot individually enter everything necessary to cover for theft, etc.
\end{quote}

\chapter{Migration}
\label{app-migration}

\section{Introduction}
\label{sec-migration-introduction}

There are plenty of reasons to want to migrate to LedgerSMB 1.3:

\begin{enumerate}
\item Separation of duties
\item Security better integrated into the application
\item Better, more strict data model
\label{item:StricterDataModel}
\item Some important sources of user error eliminated
\item Better workflows for cash reconciliation
\item @@@ others?
\end{enumerate}

Yet, while item \ref{item:StricterDataModel} is a good reason to want to switch, it's
also a reason why migration from older versions to 1.3 can be harder than earlier
migrations: when the data
in your older version is not consistent, it won't fit into the new data model and
will need to be fixed first.

Especially if your database has a very long line of history, being migrated trough
lots of SQL-Ledger and LedgerSMB versions, you may want to consider asking for help
from a professional party. It could save you a lot of time.

However, don't be discouraged and have a go yourself first. Just be sure to run
your upgrade on a backup database: the migration process is non-destructive, but
in case accounting data is involved: better safe than sorry!

Also it is worth noting that a number of automatic checks are performed on your
data prior to migration, and to the extent possible, you are given an
opportunity to fix those issues identified.  Because these checks are
pre-migration checks, they are written to your old data and will persist after
backing out of a migration to 1.3.

\section{From older LedgerSMB versions}
\label{sec-migration-from-older-ledgersmb}

\section{From SQL-Ledger 2.8}
\label{sec-migration-from-sql-ledger-28}

\section{From older SQL-Ledger versions}
\label{sec-migration-from-older-sql-ledger}

\section{From other accounting packages}
\label{sec-migration-from-other-accounting-systems}

\subsection{Overview}





While accounting and ERP solution have wildly differing structures to record their
data, this sections uses data with a relatively simple structure as a show case of
how this problem may be dealt with.

\begin{quote}
Note that the encoding you use to transfer to the database depends on the settings
used to create the PostgreSQL database with.  A migration is a good moment to think
about encodings an solve older encoding issues.  Now would be a good moment to
anticipate the requirement for accented characters and non-western alphabets: set up
a UTF-8 encoded database and recode your data accordingly.
\end{quote}

\subsection{Migrating customers}
\label{subsec-migration-others-customers}

The source system for this section uses a structure where every company has one
contact person, one address, one phone number and e-mail.

In order to understand how to migrate this data structure to LedgerSMB, it's
important to understand that:

\begin{enumerate}
\item The company from the source maps to the \emph{Company} and \emph{Entity} entities
\item The contact person maps to the \emph{Entity Credit Account} entity
\item The address maps to the \emph{Location} entity - and requires a location class: Sales, Billing or Shipping
\item The phone number, fax number and e-mail map to \emph{Contact items}
\end{enumerate}

The reason behind the separation between the \emph{Company} and \emph{Entity} entities
is that every customer is an \emph{Entity}, but not all entities are companies, since
some entities are \emph{Person}s - natural persons.

@@@ How to

\subsection{Migrating parts, services, ...}
\label{subsec-migration-others-products}

\subsection{Migrating the balance}

The strategy that I used when migrating to LedgerSMB is the following:

\begin{itemize}
\item create customers and vendors as needed
\item create at least the parts and services for which there are open AR or AP items in the closing balance
\item create the open AR/AP items in the closing balance by posting them on the original opening date (this allows you to do your aging management in LSMB from day 1)
\item create in addition the parts for which you have stock in the closing balance
\item import your stock by posting invoices against the inventory entity
\item make sure you "pay" the invoices - e.g. by paying them from equity
\item draw up the incomplete balance you have so far
\item calculate the delta of the partially imported balance against the closing balance
\item import the delta balance on top of what you already had to make a full balance
\end{itemize}


What I did is post the transaction from the last step on the day before I wanted to start my books; say you want to start your books on 2014-01-01, then you'd post it on 2013-12-31. That way, the opening balance of your books on the starting date is exactly what you had on the closing balance of the books you left behind.

\subsubsection{Migrating open AR and AP items}
\label{subsec-migration-others-open-items}

\subsubsection{Migrating your ledger}
\label{subsec-migration-others-ledger}

\section{Migrating between PostgreSQL versions}
\label{sec-migration-postrgesql}

If you're migrating between PostgreSQL versions, there are a few things to take
into account.

% pg_dump: use the NEW pg_dump utility against the old database
% pg_upgrade: <TODO>
% pg_<migratecluster>: <TODO>; in case of unclean extension upgrade,
%              possible need to issue ``CREATE EXTENSION tablefunc FROM UNPACKAGED''

\subsubsection{Notes on migration from 8.3 (or earlier) to 8.4 (or later)}
\label{subsec-migration-pre84-to84plus}

% Performance benefit due to built in support for recursive queries only available
% after the next setup.pl run
% Also, after next setup.pl run, one should uninstall the tablefunc extension

% 9.1+ DROP EXTENSION tablefunc;
% 8.4, 9.0: use uninstall_tablefunc.sql from the contrib directory

LedgerSMB 1.3 uses some extension modules for versions 8.3 and before for functionality
that has been built into 8.4 and later. To make use of the (faster) built in version
of that functionality, the following restore procedure should be used.

\begin{enumerate}
\item Migrate the database to the new function as described in section @@@ TODO
\item If you're using 9.1 and up, issue the command ``CREATE EXTENSION tablefunc FROM UNPACKAGED''
   from a psql prompt when connected to the company database
\item Run 'setup.pl' from your browser to upgrade the database's routines; this command will
   install routines optimized for your version of PostgreSQL
\item Run the command
        \begin{description}
        \item [8.4 or 9.0] \$ psql ... -f uninstall\_tablefunc.sql
        \item [9.1 and up] ``DROP EXTENSION tablefunc;'' from a psql session connected
                to the company database
        \end{description}
        to clean up functions and procedures in the database which are no longer used
\end{enumerate}


\chapter{Listing of application roles}
\label{app-role-listing}

Application roles \index{roles} specify the right (privilege) to execute one or more tasks in the application.
LedgerSMB enforces these roles by allowing a user to select (list, read) data from or to
insert (create), update (edit) or delete (delete) data in the tables holding the data
related to the execution of these tasks.

%This list contains workaround for HTML lists that looks so bad.  That is all of the '\htmlspacing' stuff.
% This change at least haromonizes the look between PDF and HTML except that
% PDF does not have a line space between the role and the description and the HTML does.
% I am looking for a better solution before we go about modifying all of the lists in the book. Neil

% For large groups of roles both the first and last items are indexed in case the list appears in separate pages.

\ifpdf
      \newcommand{\htmlspacing}{}
\else
      \newcommand{\htmlspacing}{ \hfill \\}
\fi

These roles definitions are generated directly from a running database using  \texttt{gather-db-info.pl}.


Auto generated using LedgerSMB version 1.12.0-dev on November 21, 16:06:05 2023 CST.

\begin{description}[style=nextline]
\item [account\_all] \htmlspacing 
                         This role combines all \gls{GL} \index{GL}\index{General Ledger} account and GIFI code rights.
                          \index{account all}
\item [account\_create] \htmlspacing 
                         This role allows creation of new \gls{GL} \index{GL}\index{General Ledger} accounts.
                          \index{account create}
\item [account\_delete] \htmlspacing 
                         This role allows deletion of \gls{GL} \index{GL}\index{General Ledger} accounts.
                         Please note that there are only very few circumstances
                         where a \gls{GL} \index{GL}\index{General Ledger} account can be deleted.  Instead of deleting
                         the account, the user is advised to mark the account
                         as 'obsolete'.
                          \index{account delete}
\item [account\_edit] \htmlspacing 
                         This role allows modification of \gls{GL} \index{GL}\index{General Ledger} accounts.
                          \index{account edit}
\item [account\_link\_description\_create] \htmlspacing 
                         This role allows creating new use-cases for \gls{GL} \index{GL}\index{General Ledger}
                         accounts (so called "account link descriptions").
                          \index{account link description create}
\item [ap\_all] \htmlspacing 
                         This role combines all (batches of) purchase transaction and invoices permissions.
                          \index{ap all}
\item [ap\_invoice\_create] \htmlspacing 
                         This role allows creation of new purchase invoices (not purchase transactions).
                          \index{ap invoice create}
\item [ap\_invoice\_create\_voucher] \htmlspacing 
                         This role allows creation of batches of new purchase invoices (not purchase transactions).
                          \index{ap invoice create voucher}
\item [ap\_transaction\_all] \htmlspacing 
                         This role allows creating and viewing purchase transactions and accounts as well as creating attachments.
                          \index{ap transaction all}
\item [ap\_transaction\_create] \htmlspacing 
                         This role allows creation of purchase transactions (not invoices).
                          \index{ap transaction create}
\item [ap\_transaction\_create\_voucher] \htmlspacing 
                         This role allows creation of batches of new purchase transactions (not invoices).
                          \index{ap transaction create voucher}
\item [ap\_transaction\_list] \htmlspacing 
                         This role allows viewing of purchase transactions and invoices.
                          \index{ap transaction list}
\item [ap\_voucher\_all] \htmlspacing 
                         This role allows creation of batches of both purchase transactions and invoices.
                          \index{ap voucher all}
\item [ar\_all] \htmlspacing 
                         This role combines all (batches of) sales transaction and invoices permissions.
                          \index{ar all}
\item [ar\_invoice\_create] \htmlspacing 
                         This role allows creation of new sales invoices (not sales transactions).
                          \index{ar invoice create}
\item [ar\_invoice\_create\_voucher] \htmlspacing 
                         This role allows creation of batches of new sales invoices (not sales transactions).
                          \index{ar invoice create voucher}
\item [ar\_transaction\_all] \htmlspacing 
                         This role allows creating and viewing sales transactions and accounts as well as creating attachments.
                          \index{ar transaction all}
\item [ar\_transaction\_create] \htmlspacing 
                         This role allows creation of new sales transactions (not nvoices).
                          \index{ar transaction create}
\item [ar\_transaction\_create\_voucher] \htmlspacing 
                         This role allows creation of batches of new sales transactions (not invoices).
                          \index{ar transaction create voucher}
\item [ar\_transaction\_list] \htmlspacing 
                         This role allows viewing of sales transactions and invoices.
                          \index{ar transaction list}
\item [ar\_voucher\_all] \htmlspacing 
                         This role allows creation of batches of both sales transactions and invoices.
                          \index{ar voucher all}
\item [assembly\_stock] \htmlspacing 
                         This role allows triggering a stocking action on assemblies.

                         Stocking assemblies means converting labor and parts to stocked assemblies.
                          \index{assembly stock}
\item [assets\_administer] \htmlspacing 
                         This role combines all assets rights.
                          \index{assets administer}
\item [assets\_approve] \htmlspacing 
                         This role allows approving the output of the
                         depreciation procedure.
                          \index{assets approve}
\item [assets\_depreciate] \htmlspacing 
                         This role allows running the asset depreciation
                         procedure.
                          \index{assets depreciate}
\item [assets\_enter] \htmlspacing 
                         This role allows creation of new assets.
                          \index{assets enter}
\item [audit\_trail\_maintenance] \htmlspacing 
                         This role grants delete access to the audit trail table.
                          \index{audit trail maintenance}
\item [auditor] \htmlspacing 
                         This role grants read access to the audit trail table.
                          \index{auditor}
\item [base\_user] \htmlspacing 
  Users need to be given this role in order to be granted access to the database schema which holds all LedgerSMB objects.

  This role only allows access to menu items Preferences, Logout, and New Window. The user basically cannot do anything without added additional roles.
   \index{base user}
\item [batch\_create] \htmlspacing 
                         This role allows creation of new batches and vouchers.
                          \index{batch create}
\item [batch\_list] \htmlspacing 
                         This role allows listing existing batches.
                          \index{batch list}
\item [batch\_post] \htmlspacing 
                         This role allows posting batches of e.g. transactions, payments and invoices.
                          \index{batch post}
\item [budget\_approve] \htmlspacing 
                         This role allows searching, viewing and approving of budgets.
                          \index{budget approve}
\item [budget\_enter] \htmlspacing 
                         This role allows creation and updating of budgets.
                          \index{budget enter}
\item [budget\_obsolete] \htmlspacing 
                         This role allows searching and viewing budgets as well as marking them  obsolete (=no longer applicable).
                          \index{budget obsolete}
\item [budget\_view] \htmlspacing 
                         This role allows searching and viewing of budgets.
                          \index{budget view}
\item [business\_type\_all] \htmlspacing 
                         This role combines the create and edit righs for
                         'type of business' classes.
                          \index{business type all}
\item [business\_type\_create] \htmlspacing 
                         This role allows creation of new 'type of business'
                         classes.
                          \index{business type create}
\item [business\_type\_edit] \htmlspacing 
                         This role allows modification of 'type of business'
                         classes.
                          \index{business type edit}
\item [business\_units\_manage] \htmlspacing 
                         This role allows searching, viewing, creation and editing of business (reporting) classes and their members.
                          \index{business units manage}
\item [cash\_all] \htmlspacing 
                         This role combines the all reconciliation rights with the rights to enter payments and receipts.
                          \index{cash all}
\item [contact\_all\_rights] \htmlspacing 
                         This role combines all 'contact\_class\_$\ast$' and 'contact\_$\ast$' roles and grants all access rights to all contact classes.
                          \index{contact all rights}
\item [contact\_class\_cold\_lead] \htmlspacing 
                         This role allows access to cold sales lead contact data.
                         Combine with 'contact\_read', 'contact\_create', 'contact\_edit' and/or 'contact\_delete' to determine
                         the type of access granted.
                          \index{contact class cold lead}
\item [contact\_class\_contact] \htmlspacing 
                         This role allows access to contact data (e-mail, phone, etc) of all kinds of contacts (customer/vendor/...).
                         Combine with 'contact\_read', 'contact\_create', 'contact\_edit' and/or 'contact\_delete' to determine
                         the type of access granted.
                          \index{contact class contact}
\item [contact\_class\_customer] \htmlspacing 
                         This role allows access to customer contact data.
                         Combine with 'contact\_read', 'contact\_create', 'contact\_edit' and/or 'contact\_delete' to determine
                         the type of access granted.
                          \index{contact class customer}
\item [contact\_class\_employee] \htmlspacing 
                         This role allows access to employee contact data.
                         Combine with 'contact\_read', 'contact\_create', 'contact\_edit' and/or 'contact\_delete' to determine
                         the type of access granted.
                          \index{contact class employee}
\item [contact\_class\_hot\_lead] \htmlspacing 
                         This role allows access to hot sales lead contact data.
                         Combine with 'contact\_read', 'contact\_create', 'contact\_edit' and/or 'contact\_delete' to determine
                         the type of access granted.
                          \index{contact class hot lead}
\item [contact\_class\_lead] \htmlspacing 
                         This role allows access to sales lead contact data.
                         Combine with 'contact\_read', 'contact\_create', 'contact\_edit' and/or 'contact\_delete' to determine
                         the type of access granted.
                          \index{contact class lead}
\item [contact\_class\_referral] \htmlspacing 
                         This role allows access to referral contact data.
                         Combine with 'contact\_read', 'contact\_create', 'contact\_edit' and/or 'contact\_delete' to determine
                         the type of access granted.
                          \index{contact class referral}
\item [contact\_class\_robot] \htmlspacing 
                         This role allows access to robot (automated process, acting on behalf of...) contact data.
                         Combine with 'contact\_read', 'contact\_create', 'contact\_edit' and/or 'contact\_delete' to determine
                         the type of access granted.
                          \index{contact class robot}
\item [contact\_class\_sub\_contractor] \htmlspacing 
                         This role allows access to subcontractor contact data.
                         Combine with 'contact\_read', 'contact\_create', 'contact\_edit' and/or 'contact\_delete' to determine
                         the type of access granted.
                          \index{contact class sub contractor}
\item [contact\_class\_vendor] \htmlspacing 
                         This role allows access to vendor contact data.
                         Combine with 'contact\_read', 'contact\_create', 'contact\_edit' and/or 'contact\_delete' to determine
                         the type of access granted.
                          \index{contact class vendor}
\item [contact\_create] \htmlspacing 
                         When paired with one or more 'contact\_class\_$\ast$' role/-s, this role allows
                         creation of new entities, persons and companies (contacts).

                         Each contact\_class\_\textless{}resource\textgreater{} role, when paired with contact\_read, enables
                         this access for the specific \textless{}resource\textgreater{}. On it's own, the contact\_create-role
                         does not provide any rights.
                          \index{contact create}
\item [contact\_delete] \htmlspacing 
                         When paired with one or more 'contact\_class\_$\ast$' role/-s, this role allows
                         removal of existing entities, persons and companies (contacts).

                         Note that in order to be able to search for contacts to be deleted, the user
                         needs to be assigned the 'contact\_read' role.

                         Each contact\_class\_\textless{}resource\textgreater{} role, when paired with contact\_read, enables
                         this access for the specific \textless{}resource\textgreater{}. On it's own, the contact\_delete-role
                         does not provide any rights.
                          \index{contact delete}
\item [contact\_edit] \htmlspacing 
                         When paired with one or more 'contact\_class\_$\ast$' role/-s, this role allows
                         editing of existing entities, persons and companies (contacts).

                         Each contact\_class\_\textless{}resource\textgreater{} role, when paired with contact\_read, enables
                         this access for the specific \textless{}resource\textgreater{}. On it's own, the contact\_edit-role
                         does not provide any rights.
                          \index{contact edit}
\item [contact\_read] \htmlspacing 
                         When paired with one or more 'contact\_class\_$\ast$' role/-s, this role allows
                         searching and viewing entities, persons and companies (contacts).

                         Each contact\_class\_\textless{}resource\textgreater{} role, when paired with contact\_read, enables
                         this access for the specific \textless{}resource\textgreater{}. On it's own, the contact\_read-role
                         does not provide any rights.
                          \index{contact read}
\item [country\_all] \htmlspacing 
                         This role combines all rights for countries.
                          \index{country all}
\item [country\_create] \htmlspacing 
                         This role allows creation of new countries.
                          \index{country create}
\item [country\_edit] \htmlspacing 
                         This role allows modification of countries.
                          \index{country edit}
\item [draft\_modify] \htmlspacing 
                         This role allows modification of existing draft (= saved) transactions.
                          \index{draft modify}
\item [draft\_post] \htmlspacing 
                         This role allows posting of saved transactions to the ledger.
                          \index{draft post}
\item [employees\_manage] \htmlspacing 
                         This role allows creation, updating and searching of employees.
                          \index{employees manage}
\item [exchangerate\_edit] \htmlspacing 
                         This role allows searching, viewing and editing of currencies, exchange rates and exchange rate types.
                          \index{exchangerate edit}
\item [file\_attach\_eca] \htmlspacing 
                         This role allows attaching files to entity credit accounts (customers/vendors).
                          \index{file attach eca}
\item [file\_attach\_entity] \htmlspacing 
                         This role allows attaching files to entities (contacts).
                          \index{file attach entity}
\item [file\_attach\_order] \htmlspacing 
                         This role allows attaching files to orders and quotes.
                          \index{file attach order}
\item [file\_attach\_part] \htmlspacing 
                         This role allows attaching files to goods and services.
                          \index{file attach part}
\item [file\_attach\_tx] \htmlspacing 
                         This role allows attaching files to transactions and invoices.
                          \index{file attach tx}
\item [file\_read] \htmlspacing 
                         This role allows reading of files attachments and files uploaded through the system menu.
                          \index{file read}
\item [file\_upload] \htmlspacing 
                         This role allows uploading of files through the system menu.
                          \index{file upload}
\item [financial\_reports] \htmlspacing 
                         This role allows running of financial reports:
                         Income Statement, Balance Sheet, Trial Balance and
                         Inventory \& COGS.
                          \index{financial reports}
\item [gifi\_create] \htmlspacing 
                         This role allows creation of new GIFI codes.
                          \index{gifi create}
\item [gifi\_edit] \htmlspacing 
                         This role allows modification of GIFI codes.
                          \index{gifi edit}
\item [gl\_all] \htmlspacing 
                         This role combines \gls{GL} \index{GL}\index{General Ledger} transaction and batch creation
                         with \gls{GL} \index{GL}\index{General Ledger} reporting and year-end processing.
                          \index{gl all}
\item [gl\_reports] \htmlspacing 
                         This role allows searching transactions in the general ledger.
                          \index{gl reports}
\item [gl\_transaction\_create] \htmlspacing 
                         This role allows creation of new and updating of saved \gls{GL} \index{GL}\index{General Ledger} transactions.
                          \index{gl transaction create}
\item [gl\_voucher\_create] \htmlspacing 
                         This role allows creation of batches of \gls{GL} \index{GL}\index{General Ledger} transactions.
                          \index{gl voucher create}
\item [inventory\_adjust] \htmlspacing 
                         This role allows adjusting inventory by creating inventory adjustment reports.
                          \index{inventory adjust}
\item [inventory\_all] \htmlspacing 
                         This role grants all rights to manage warehouse configuration, stock receipt, shipping and transfer.
                          \index{inventory all}
\item [inventory\_approve] \htmlspacing 
                         This role allows confirmation of inventory adjustments by approval of inventory adjustment reports.
                          \index{inventory approve}
\item [inventory\_receive] \htmlspacing 
                         This role allows receiving of parts into stock.
                          \index{inventory receive}
\item [inventory\_reports] \htmlspacing 
                         This role allows searching for and reading existing inventory adjustment reports.
                          \index{inventory reports}
\item [inventory\_ship] \htmlspacing 
                         This role allows shipping of stocked parts.
                          \index{inventory ship}
\item [inventory\_transfer] \htmlspacing 
                         This role allows moving stock between warehouses.
                          \index{inventory transfer}
\item [language\_create] \htmlspacing 
                         This role allows creation of new languages.
                          \index{language create}
\item [language\_edit] \htmlspacing 
                         This role allows modification of languages.
                          \index{language edit}
\item [orders\_generate] \htmlspacing 
                         This role combines the rights to generate orders from
                         time cards, purchase orders from sales orders
                         and consolidate (purchase and sales) orders.
                          \index{orders generate}
\item [orders\_manage] \htmlspacing 
                         This role combines all order generation and
                         consolidation rights.
                          \index{orders manage}
\item [orders\_purchase\_consolidate] \htmlspacing 
                         This role allows generating consolidated purchase
                         orders from multiple outstanding purchase orders.
                          \index{orders purchase consolidate}
\item [orders\_sales\_consolidate] \htmlspacing 
                         This role allows generating consolidated sales
                         orders from multiple outstanding sales orders.
                          \index{orders sales consolidate}
\item [orders\_sales\_to\_purchase] \htmlspacing 
                         This role allows generating purchase orders
                         from sales orders.
                          \index{orders sales to purchase}
\item [part\_create] \htmlspacing 
                         This role allows creation of new parts.

                         So as to let the user of this role see/manage pricing per customer, this role includes
                         the ability to read contacts.
                          \index{part create}
\item [part\_delete] \htmlspacing 
                         This role allows deletion of existing parts.
                          \index{part delete}
\item [part\_edit] \htmlspacing 
                         This role allows changing existing parts.
                          \index{part edit}
\item [payment\_process] \htmlspacing 
                         This role allows entry of payments to vendors.
                          \index{payment process}
\item [pricegroup\_create] \htmlspacing 
                         This role allows creation of new price groups.
                          \index{pricegroup create}
\item [pricegroup\_edit] \htmlspacing 
                         This role allows changing existing price groups.
                          \index{pricegroup edit}
\item [purchase\_order\_create] \htmlspacing 
                         This role allows creating purchase orders.
                          \index{purchase order create}
\item [purchase\_order\_delete] \htmlspacing 
                         This role allows (searching for and) deleting existing purchase orders.
                          \index{purchase order delete}
\item [purchase\_order\_edit] \htmlspacing 
                         This role allows (searching for and) modifying existing purchase orders.
                          \index{purchase order edit}
\item [purchase\_order\_list] \htmlspacing 
                         This role allows searching and viewing sales orders.
                          \index{purchase order list}
\item [receipt\_process] \htmlspacing 
                         This role allows entry of receipts from customers.
                          \index{receipt process}
\item [reconciliation\_all] \htmlspacing 
                         This role combines creation, updating and approval rights for reconciliation reports.
                          \index{reconciliation all}
\item [reconciliation\_approve] \htmlspacing 
                         This role allows approval of reconciliation reports.
                          \index{reconciliation approve}
\item [reconciliation\_enter] \htmlspacing 
                         This role allows creation and updating of reconciliation reports.
                          \index{reconciliation enter}
\item [recurring] \htmlspacing 
                         This role allows access to the Recurring Transactions
                         menu; it does not grant rights to list or create
                         transactions.
                          \index{recurring}
\item [rfq\_create] \htmlspacing 
                         This role allows creating (purchase) requests for quotation.
                          \index{rfq create}
\item [rfq\_delete] \htmlspacing 
                         This role allows (searching for and) deleting existing requests for quotation.
                          \index{rfq delete}
\item [rfq\_list] \htmlspacing 
                         This role allows searching and viewing (purchase) requests for quotation.
                          \index{rfq list}
\item [sales\_order\_create] \htmlspacing 
                         This role allows creating sales orders.
                          \index{sales order create}
\item [sales\_order\_delete] \htmlspacing 
                         This role allows (searching for and) deleting existing sales orders.
                          \index{sales order delete}
\item [sales\_order\_edit] \htmlspacing 
                         This role allows (searching for and) modifying existing sales orders.
                          \index{sales order edit}
\item [sales\_order\_list] \htmlspacing 
                         This role allows searching and viewing sales orders.
                          \index{sales order list}
\item [sales\_quotation\_create] \htmlspacing 
                         This role allows creating sales quotations.
                          \index{sales quotation create}
\item [sales\_quotation\_delete] \htmlspacing 
                         This role allows (searching for and) deleting existing sales quotations.
                          \index{sales quotation delete}
\item [sales\_quotation\_list] \htmlspacing 
                         This role allows searching and viewing sales quotations.
                          \index{sales quotation list}
\item [sic\_all] \htmlspacing 
                         This role combines all rights for Standardized Industry
                         Codes (SIC).
                          \index{sic all}
\item [sic\_create] \htmlspacing 
                         This role allows creation of new Standardized Industry
                         Codes (SIC).
                          \index{sic create}
\item [sic\_edit] \htmlspacing 
                         This role allows modification of Standardized Industry
                         Codes (SIC).
                          \index{sic edit}
\item [system\_admin] \htmlspacing 
                         This role combines the rights to manage settings,
                         \gls{GL} \index{GL}\index{General Ledger} accounts, types of business, SIC, users and tax
                         forms.
                          \index{system admin}
\item [system\_settings\_change] \htmlspacing 
                         This role allow changing items in the
                         System \textgreater{} Defaults menu.
                          \index{system settings change}
\item [system\_settings\_list] \htmlspacing 
                         This role allows viewing items in the System \textgreater{} Defaults
                         menu.
                          \index{system settings list}
\item [tax\_form\_save] \htmlspacing 
                         This role allows modification of tax forms.
                          \index{tax form save}
\item [taxes\_set] \htmlspacing 
                         This role allows changing tax rates on tax accounts.
                          \index{taxes set}
\item [template\_edit] \htmlspacing 
                         This role allows modification of document
                         (e.g. invoice) templates.
                          \index{template edit}
\item [timecard\_add] \htmlspacing 
                         This role allows adding time cards for which it needs
                         read access to customers.
                          \index{timecard add}
\item [timecard\_list] \htmlspacing 
                         This role allows viewing the list of time cards;
                         for which it needs read access to customers.
                          \index{timecard list}
\item [timecard\_order\_generate] \htmlspacing 
                         This role allows generating orders from time cards.
                          \index{timecard order generate}
\item [transaction\_template\_delete] \htmlspacing 
                         This role allows deletion of template (i.e. unposted)
                         transactions.
                          \index{transaction template delete}
\item [translation\_create] \htmlspacing 
                         This role allows creation of translations for parts,
                         parts groups and reporting units.
                          \index{translation create}
\item [users\_manage] \htmlspacing 
                         This role allows addition and removal of users to
                         the current company.
                          \index{users manage}
\item [voucher\_delete] \htmlspacing 
                         This role allows deletion of vouchers (i.e. groups of e.g. payments).
                          \index{voucher delete}
\item [warehouse\_create] \htmlspacing 
                         This role allows creation of (configuration of) new warehouses.
                          \index{warehouse create}
\item [warehouse\_edit] \htmlspacing 
                         This role allows updating of (configuration of) existing warehouses.
                          \index{warehouse edit}
\item [yearend\_reopen] \htmlspacing 
                         This role allows undoing a prior year-end run by reversing the year-end transaction.
                          \index{yearend reopen}
\item [yearend\_run] \htmlspacing 
                         This role allows running the year-end process, i.e. clearing the P\&L.
                          \index{yearend run}
\end{description}

\chapter{Deprecated Content}
\label{app-deprecated-content}

\section{ledgersmb.conf}
\label{app-deprecated-ledgersmb-conf}

This ledgersmb.conf is deprecated as of 1 Jan 2023 and only remains here for historical use.

The environment variable for controlling the configuration is \texttt{LSMB\_CONFIG\_FILE}.

By default it pointed to \texttt{/etc/ledgersmb/ledgersmb.conf} and can be set to an alternate location.

\subsubsection{[main] Section}
\label{subsubsec-global-config-ledgersmb-conf-general}

\begin{description}[style=nextline]
        \item[language] Set language for login and admin pages. 
        \item[date\_format] Set the date format for the admin pages.  The default is \texttt{yyyy-mm-dd}. Unset means detected from the browser)
        \item[log\_level] Specifies the logging level to be used. This setting is ignored when the \texttt{log\_config} setting 
        is supplied and indicates an existing file. Available values are OFF, FATAL, ERROR, WARN, INFO, DEBUG, TRACE.
        \item [log\_config] Set the name of the extended logging configuration file. This file uses the \texttt{log4j/Log::Log4perl} syntax described at \url{https://metacpan.org/pod/Log::Log4perl#Configuration-files}
        \item [dojo\_theme] This is the Dojo theme that is used by default when no other theme
        has been selected.  The default if unset is \texttt{claro}. Other values can be \texttt{nihilo}, \texttt{soria}, and \texttt{tundra}.
        \item[auth] Indicates which authentication module is to be used; default is DB, referring to LedgerSMB::Auth::DB.
        \item[cache\_templates] Determines if HTML templates used for the LedgerSMB UI will be stored
        in precompiled form for improved performance. The value 0 (zero) means no caching; the value
        1 means caching in the directory \$(tempdir)/lsmb\_templates; systems running multiple
        LedgerSMB instances with customized UIs should use separate values for {\tt tempdir} for that
        reason

% From the old book, commented out for now.
%       \item[check\_max\_invoices] @@@TODO
%       \item[max\_post\_size] States the maximum size in bytes of the body in case of a POST request; this
%       setting reduces the impact of DoS requests, if the system is under attack. The setting also limits the
%       maximum size of attachments (see \secref{sec-workflows-quotations-file-attachments}) that can be uploaded
%       \item[decimal\_places] @@@TODO
%       \item[cookie\_name] @@@TODO
%       \item[no\_db\_str] @@@TODO
        
\end{description}

\subsubsection{[environment] Section}
\label{subsubsec-global-config-ledgersmb-conf-environment}

        Sets shell environment variables when starting up.

\begin{description}[style=nextline]
        \item[PATH] If the server can't find applications, append to the path. For example:
        \texttt{PATH=/bin:/usr/bin:/usr/local/bin:/usr/local/pgsql/bin}
        \item [PERL5LIB] On some operating systems like macOS the perl path may need to be set. For example, on macOS using Fink's Perl libs it might be set like:
        \texttt{PERL5LIB=/sw/lib/perl5}
\end{description}


\subsubsection{[paths] Section}
\label{subsubsec-global-config-ledgersmb-conf-paths}

\begin{description}[style=nextline]
        \item [templates] Templates base directory. Typically set to \texttt{templates}.
        \item [images] Images base directory. Typically set to \texttt{UI/images}.
        \item [localepath] Path to locale files. Defaults to \texttt{locale/po}.
        \item [templates\_cache] Location where compiled templates are stored. When relative, appended to the directory specified by \texttt{File::Spec->tmpdir()}. 
        
        Defaults to \texttt{lsmb\_templates}.
        \item [workflows] Directory where workflow files are stored. Defaults to \texttt{workflows}
        \item [custom\_workflows] Directory where custom workflow files are stored. Custom workflows are used to override behavior of the default workflows by providing actions, conditions, etc. with the same name and type or by providing workflows of the same type with additional states and actions. Defaults to \texttt{custom\_workflows}.
        
% Items from the old book
%       \item [spool] @@@TODO
\end{description}

\subsubsection{[programs] Section}
\label{subsubsec-global-config-ledgersmb-conf-programs}

Setup the various helper programs that can be used with LedgerSMB.

\subsubsection{[mail] Section}
\label{subsubsec-global-config-ledgersmb-conf-mail}

\begin{description}[style=nextline]
        \item [smtphost] The host name of the smtp server. If provided, then do not use local sendmail. Note that one of sendmail or smtphost has to be provided for email to work.
        \item [sendmail] Path the local sendmail executable. If provided then do not use an smtp server. Note that one of sendmail or smtphost has to be provided for email to work.
        \item [smtpport] The port number of the smtp server. Note this might vary depending on whether TLS or SSL is used.
        \item [smtpuser] The smpt server account name
        \item [smtpauthmech] The smtp server authentication method
        \item [smtppass] The smtp server password
        \item [backup\_email\_from] Set to enable e-mail delivery of backups. For example:
        \texttt{backup\_email\_from = backups@lsmb\_hosting.com}
        \item [smtptimeout] The timeout value in seconds.

        % Not sure this is correct.  Why use tls to get SSL? But this appears to be the way the code currently works.
        \item [smtptls] If unset or 'no' then do not use either TLS or SSL. If 'yes' then use TLS and if 'tls' use SSL for the smtp connection.

        \item [smtpsender\_hostname] Override the default host name for \texttt{helo}.
\end{description}

\subsubsection{[printers] Section}
\label{subsubsec-global-config-ledgersmb-conf-printers}

This section contains a list of printers and the commands
to be executed in order for the output to be sent to the given printer with
the document to be printed fed to the program through standard input.

The example below shows how to send standard input data to a printer called ``laser''
when selecting the item ``Laser'' in the LedgerSMB printer selection drop down.

\texttt{
        Laser    = lpr -Plaser
}

Or to an Epson Printer.

\texttt{
        Epson   = lpr -PEpson
}


\subsubsection{[database] Section}
\label{subsubsec-global-config-ledgersmb-conf-database}

\begin{description}[style=nextline]
        \item [host] Name of the host to connect to. See the documentation of the {\tt -h} command line option at 
        \url{https://www.postgresql.org/docs/current/static/app-psql.html}
        for more information (documentation unchanged since before 8.3, so applies to older versions as well)
        \item [port] Port number to be used to connect to. See the documentation of the {\tt -p} option at the
        same URL as the previous item
        \item [default\_db] Database to connect to when the ``Company'' entry field in the login screen is left blank
        \item [db\_namespace] The name space the company resides in; expert setting -- should not be used (default is
        ``public'')
        \item [sslmode] Selects whether to use \gls{SSL} over TCP connections or not; can be ``require'', ``allow'',
        ``prefer'' or ``disable''.  Default is ``prefer''.
\end{description}

\chapter{Open source explained}
\label{app-open-source-explained}

\section{An open source application}
\label{sec-open-source-application}

\section{An open source book}
\label{sec-open-source-book}

\chapter{Project history}

\label{cha-ledgersmb-history}

%% @@@TODO This section should be entirely rewritten from the perspective of 1.10

The project started out as a fork of \gls{SL} - the open source \gls{erp}
developed by Dieter Simader - somewhere between SQL-Ledger versions 2.6
and 2.8.  A fork happens when a group of developers can't - for whatever
reason - continue to work as one group on a project.  At that time, the
project splits into two or more projects and the fork is in effect.

LedgerSMB split off from the SQL-Ledger project (i.e. forked) because
there was disagreement between developers about how to go forward both with
respect to handling of security vulnerability reports as well as the general
state of the code base.

After the fork, between versions 1.0 and 1.2 most energy was spent on
making LedgerSMB more secure (i.e. less vulnerable to security attacks).  In
technical terms, measures were taken to fend off (amongst other things):

\begin{itemize}
	\item Cross site scripting attacks
	\item Replay attacks
	\item SQL injection attacks
\end{itemize}

Come version 1.3 the development directed toward improvement of the overall
quality of the code base as the old \acrshort{SL} code was in very poor state:
looking very much like web programs as they were written in 1998, the code had
grown largely outdated in style and was no longer maintainable by 2007.

The 1.3 effort focused on bringing relief with a new application structure.
Modern and important features were realized: separation of duties (for the
accounting part of the application) and authorizations to allow distinguishing
different roles in a company.

Unfortunately, by the beginning of 2011 the project looked mostly dead from
an outside perspective: the team had not brought forward any releases since
2007, there were no signs of development and the mailing lists (a measure
of community activity) were
completely silent.  \gls{svn} commits were continuing, but were being made by
ever fewer committers and contributors.

Fortunately development activity was increasing in the first half year of 2011,
leading to the release of version 1.3 by September.  Between September and the
year end in total 10 small bug fixes were released, showing active commitment
of the developers to maintain the application.

New committers showed up, indicating revived community interest. Other signs of
increased interest are the higher number of bug reports and the creation of the
Linux package for Debian 7, which has been included in Ubuntu 12.04 as of
October 2012.



\chapter{Copyright and license}
\label{app-copyright-license}

Copyright (c) 2011, 2012 Erik H\"ulsmann.


This work is licensed under the Creative Commons Attribution License.
To view a copy of this license, visit \url{https://creativecommons.org/licenses/by/3.0/}
or send a letter to Creative Commons, 559 Nathan Abbott Way,
Stanford, California 94305, USA.

A summary of the license is given below, followed by the full legal text.

\section{License summary}
\label{sec-license-summary}

\begin{verbatim}

You are free:
  * to share -- to copy, distribute and transmit the work
  * to remix -- to adapt the work


Under the following condition:
  You must attribute the work in the manner specified by
  the author or licensor (but in a way that suggests that
  they endorse you or your use of the work).


With the understanding that:
  Waiver -- Any of the above conditions can be waived if
            you get permission from the copyright holder.
  
  Other rights -- In no way are any of the following
                  rights affected by the license:
     * Your fair dealing or fair use rights, or other
         applicable copyright exceptions and limitations;
     * The author's moral rights;
     * Rights other persons may have either in the work
         itself or in how the work is used, such as
         publicity or privacy rights.

\end{verbatim}



\section{Legal full text}
\label{sec-license-fulltext}

\begin{verbatim}
License

THE WORK (AS DEFINED BELOW) IS PROVIDED UNDER THE TERMS
OF THIS CREATIVE COMMONS PUBLIC LICENSE ("CCPL" OR
"LICENSE"). THE WORK IS PROTECTED BY COPYRIGHT AND/OR
OTHER APPLICABLE LAW. ANY USE OF THE WORK OTHER THAN AS
AUTHORIZED UNDER THIS LICENSE OR COPYRIGHT LAW IS
PROHIBITED.

BY EXERCISING ANY RIGHTS TO THE WORK PROVIDED HERE, YOU
ACCEPT AND AGREE TO BE BOUND BY THE TERMS OF THIS
LICENSE. TO THE EXTENT THIS LICENSE MAY BE CONSIDERED TO
BE A CONTRACT, THE LICENSOR GRANTS YOU THE RIGHTS
CONTAINED HERE IN CONSIDERATION OF YOUR ACCEPTANCE OF
SUCH TERMS AND CONDITIONS.

1. Definitions

"Adaptation" means a work based upon the Work, or upon
the Work and other pre-existing works, such as a
translation, adaptation, derivative work, arrangement of
music or other alterations of a literary or artistic
work, or phonogram or performance and includes cinemato-
graphic adaptations or any other form in which the Work
may be recast, transformed, or adapted including in any
form recognizably derived from the original, except that
a work that constitutes a Collection will not be
considered an Adaptation for the purpose of this License.
For the avoidance of doubt, where the Work is a musical
work, performance or phonogram, the synchronization of
the Work in timed-relation with a moving image
("synching") will be considered an Adaptation for the
purpose of this License.

"Collection" means a collection of literary or artistic
works, such as encyclopedias and anthologies, or
performances, phonograms or broadcasts, or other works or
subject matter other than works listed in Section 1(f)
below, which, by reason of the selection and arrangement
of their contents, constitute intellectual creations, in
which the Work is included in its entirety in unmodified
form along with one or more other contributions, each
constituting separate and independent works in
themselves, which together are assembled into a
collective whole. A work that constitutes a Collection
will not be considered an Adaptation (as defined above)
for the purposes of this License.

"Distribute" means to make available to the public the
original and copies of the Work or Adaptation, as
appropriate, through sale or other transfer of ownership.

"Licensor" means the individual, individuals, entity or
entities that offer(s) the Work under the terms of this
License.

"Original Author" means, in the case of a literary or
artistic work, the individual, individuals, entity or
entities who created the Work or if no individual or
entity can be identified, the publisher; and in addition
(i) in the case of a performance the actors, singers,
musicians, dancers, and other persons who act, sing,
deliver, declaim, play in, interpret or otherwise perform
literary or artistic works or expressions of folklore;
(ii) in the case of a phonogram the producer being the
person or legal entity who first fixes the sounds of a
performance or other sounds; and,
(iii) in the case of broadcasts, the organization that
transmits the broadcast.

"Work" means the literary and/or artistic work offered
under the terms of this License including without
limitation any production in the literary, scientific
and artistic domain, whatever may be the mode or form
of its expression including digital form, such as a
book, pamphlet and other writing; a lecture, address,
sermon or other work of the same nature; a dramatic or
dramatico-musical work; a choreographic work or
entertainment in dumb show; a musical composition with
or without words; a cinematographic work to which are
assimilated works expressed by a process analogous to
cinematography; a work of drawing, painting,
architecture, sculpture, engraving or lithography; a
photographic work to which are assimilated works
expressed by a process analogous to photography; a work
of applied art; an illustration, map, plan, sketch or
three-dimensional work relative to geography, topography,
architecture or science; a performance; a broadcast; a
phonogram; a compilation of data to the extent it is
protected as a copyrightable work; or a work performed
by a variety or circus performer to the extent it is not
otherwise considered a literary or artistic work.

"You" means an individual or entity exercising rights
under this License who has not previously violated the
terms of this License with respect to the Work, or who
has received express permission from the Licensor to
exercise rights under this License despite a previous
violation.

"Publicly Perform" means to perform public recitations
of the Work and to communicate to the public those public
recitations, by any means or process, including by wire
or wireless means or public digital performances; to make
available to the public Works in such a way that members
of the public may access these Works from a place and at
a place individually chosen by them; to perform the Work
to the public by any means or process and the
communication to the public of the performances of the
Work, including by public digital performance; to
broadcast and rebroadcast the Work by any means including
signs, sounds or images.

"Reproduce" means to make copies of the Work by any means
including without limitation by sound or visual
recordings and the right of fixation and reproducing
fixations of the Work, including storage of a protected
performance or phonogram in digital form or other
electronic medium.

2. Fair Dealing Rights. Nothing in this License is
intended to reduce, limit, or restrict any uses free
from copyright or rights arising from limitations or
exceptions that are provided for in connection with the
copyright protection under copyright law or other
applicable laws.

3. License Grant. Subject to the terms and conditions of
this License, Licensor hereby grants You a worldwide,
royalty-free, non-exclusive, perpetual (for the duration
of the applicable copyright) license to exercise the
rights in the Work as stated below:

to Reproduce the Work, to incorporate the Work into one
or more Collections, and to Reproduce the Work as
incorporated in the Collections;
to create and Reproduce Adaptations provided that any
such Adaptation, including any translation in any medium,
takes reasonable steps to clearly label, demarcate or
otherwise identify that changes were made to the original
Work. For example, a translation could be marked "The
original work was translated from English to Spanish,"
or a modification could indicate "The original work has
been modified.";
to Distribute and Publicly Perform the Work including as
incorporated in Collections; and, to Distribute and
Publicly Perform Adaptations.

For the avoidance of doubt:

Non-waivable Compulsory License Schemes. In those
jurisdictions in which the right to collect royalties
through any statutory or compulsory licensing scheme
cannot be waived, the Licensor reserves the exclusive
right to collect such royalties for any exercise by You
of the rights granted under this License; Waivable
Compulsory License Schemes. In those jurisdictions in
which the right to collect royalties through any
statutory or compulsory licensing scheme can be waived,
the Licensor waives the exclusive right to collect such
royalties for any exercise by You of the rights granted
under this License; and, Voluntary License Schemes.
The Licensor waives the right to collect royalties,
whether individually or, in the event that the Licensor
is a member of a collecting society that administers
voluntary licensing schemes, via that society, from any
exercise by You of the rights granted under this License.
The above rights may be exercised in all media and
formats whether now known or hereafter devised. The
above rights include the right to make such modifications
as are technically necessary to exercise the rights in
other media and formats. Subject to Section 8(f), all
rights not expressly granted by Licensor are hereby
reserved.

4. Restrictions. The license granted in Section 3 above
is expressly made subject to and limited by the
following restrictions:

You may Distribute or Publicly Perform the Work only
under the terms of this License. You must include a copy
of, or the Uniform Resource Identifier (URI) for, this
License with every copy of the Work You Distribute or
Publicly Perform. You may not offer or impose any terms
on the Work that restrict the terms of this License or
the ability of the recipient of the Work to exercise the
rights granted to that recipient under the terms of the
License. You may not sublicense the Work. You must keep
intact all notices that refer to this License and to the
disclaimer of warranties with every copy of the Work You
Distribute or Publicly Perform. When You Distribute or
Publicly Perform the Work, You may not impose any
effective technological measures on the Work that
restrict the ability of a recipient of the Work from You
to exercise the rights granted to that recipient under
the terms of the License. This Section 4(a) applies to
the Work as incorporated in a Collection, but this does
not require the Collection apart from the Work itself to
be made subject to the terms of this License. If You
create a Collection, upon notice from any Licensor You
must, to the extent practicable, remove from the
Collection any credit as required by Section 4(b), as
requested. If You create an Adaptation, upon notice from
any Licensor You must, to the extent practicable, remove
from the Adaptation any credit as required by
Section 4(b), as requested.
If You Distribute, or Publicly Perform the Work or any
Adaptations or Collections, You must, unless a request
has been made pursuant to Section 4(a), keep intact all
copyright notices for the Work and provide, reasonable
to the medium or means You are utilizing: (i) the name
of the Original Author (or pseudonym, if applicable) if
supplied, and/or if the Original Author and/or Licensor
designate another party or parties (e.g., a sponsor
institute, publishing entity, journal) for attribution
("Attribution Parties") in Licensor's copyright notice,
terms of service or by other reasonable means, the name
of such party or parties; (ii) the title of the Work if
supplied; (iii) to the extent reasonably practicable,
the URI, if any, that Licensor specifies to be associated
with the Work, unless such URI does not refer to the
copyright notice or licensing information for the Work;
and (iv), consistent with Section 3(b), in the case of an
Adaptation, a credit identifying the use of the Work in
the Adaptation (e.g., "French translation of the Work by
Original Author," or "Screenplay based on original Work
by Original Author"). The credit required by this
Section 4 (b) may be implemented in any reasonable
manner; provided, however, that in the case of a
Adaptation or Collection, at a minimum such credit will
appear, if a credit for all contributing authors of the
Adaptation or Collection appears, then as part of these
credits and in a manner at least as prominent as the
credits for the other contributing authors. For the
avoidance of doubt, You may only use the credit required
by this Section for the purpose of attribution in the
manner set out above and, by exercising Your rights under
this License, You may not implicitly or explicitly assert
or imply any connection with, sponsorship or endorsement
by the Original Author, Licensor and/or Attribution
Parties, as appropriate, of You or Your use of the Work,
without the separate, express prior written permission
of the Original Author, Licensor and/or Attribution
Parties.
Except as otherwise agreed in writing by the Licensor
or as may be otherwise permitted by applicable law,
if You Reproduce, Distribute or Publicly Perform the
Work either by itself or as part of any Adaptations or
Collections, You must not distort, mutilate, modify or
take other derogatory action in relation to the Work
which would be prejudicial to the Original Author's
honor or reputation. Licensor agrees that in those
jurisdictions (e.g. Japan), in which any exercise of the
right granted in Section 3(b) of this License (the right
to make Adaptations) would be deemed to be a distortion,
mutilation, modification or other derogatory action
prejudicial to the Original Author's honor and
reputation, the Licensor will waive or not assert, as
appropriate, this Section, to the fullest extent
permitted by the applicable national law, to enable You
to reasonably exercise Your right under Section 3(b) of
this License (right to make Adaptations) but not
otherwise.

5. Representations, Warranties and Disclaimer

UNLESS OTHERWISE MUTUALLY AGREED TO BY THE PARTIES IN
WRITING, LICENSOR OFFERS THE WORK AS-IS AND MAKES NO
REPRESENTATIONS OR WARRANTIES OF ANY KIND CONCERNING
THE WORK, EXPRESS, IMPLIED, STATUTORY OR OTHERWISE,
INCLUDING, WITHOUT LIMITATION, WARRANTIES OF TITLE,
MERCHANTIBILITY, FITNESS FOR A PARTICULAR PURPOSE,
NONINFRINGEMENT, OR THE ABSENCE OF LATENT OR OTHER
DEFECTS, ACCURACY, OR THE PRESENCE OF ABSENCE OF ERRORS,
WHETHER OR NOT DISCOVERABLE. SOME JURISDICTIONS DO NOT
ALLOW THE EXCLUSION OF IMPLIED WARRANTIES, SO SUCH
EXCLUSION MAY NOT APPLY TO YOU.

6. Limitation on Liability.

EXCEPT TO THE EXTENT REQUIRED BY APPLICABLE LAW, IN NO
EVENT WILL LICENSOR BE LIABLE TO YOU ON ANY LEGAL THEORY
FOR ANY SPECIAL, INCIDENTAL, CONSEQUENTIAL, PUNITIVE OR
EXEMPLARY DAMAGES ARISING OUT OF THIS LICENSE OR THE USE
OF THE WORK, EVEN IF LICENSOR HAS BEEN ADVISED OF THE
POSSIBILITY OF SUCH DAMAGES.

7. Termination

This License and the rights granted hereunder will
terminate automatically upon any breach by You of the
terms of this License. Individuals or entities who have
received Adaptations or Collections from You under this
License, however, will not have their licenses terminated
provided such individuals or entities remain in full
compliance with those licenses.
Sections 1, 2, 5, 6, 7, and 8 will survive any
termination of this License. Subject to the above terms
and conditions, the license granted here is perpetual
(for the duration of the applicable copyright in the
Work). Notwithstanding the above, Licensor reserves the
right to release the Work under different license terms
or to stop distributing the Work at any time; provided,
however that any such election will not serve to
withdraw this License (or any other license that has
been, or is required to be, granted under the terms of
this License), and this License will continue in full
force and effect unless terminated as stated above.

8. Miscellaneous

Each time You Distribute or Publicly Perform the
Work or a Collection, the Licensor offers
to the recipient a license to the Work on the same terms
and conditions as the license granted to You under this
License. Each time You Distribute or Publicly Perform an
Adaptation, Licensor offers to the recipient a license
to the original Work on the same terms and conditions
as the license granted to You under this License.
If any provision of this License is invalid
or unenforceable under applicable law, it shall not
affect the validity or enforceability of the remainder
of the terms of this License, and without further
action by the parties to this agreement, such provision
shall be reformed to the minimum extent necessary
to make such provision valid and enforceable. No term or
provision of this License shall be deemed waived and no
breach consented to unless such waiver or consent shall
be in writing and signed by the party to be charged with
such waiver or consent.
This License constitutes the entire agreement between the
parties with respect to the Work licensed here. There
are no understandings, agreements or representations
with respect to the Work not specified here. Licensor
shall not be bound by any additional provisions that may
appear in any communication from You. This License may
not be modified without the mutual written agreement of
the Licensor and You. The rights granted under, and the
subject matter referenced, in this License were drafted 
utilizing the terminology of the Berne Convention for the
Protection of Literary and Artistic Works (as amended
on September 28, 1979), the Rome Convention of 1961,
the WIPO Copyright Treaty of 1996, the WIPO Performances
and Phonograms Treaty of 1996 and the Universal Copyright
Convention (as revised on July 24, 1971). These
rights and subject matter take effect in the relevant
jurisdiction in which the License terms are sought to be
enforced according to the corresponding provisions
of the implementation of those treaty provisions
in the applicable national law. If the standard
suite of rights granted under applicable copyright
law includes additional rights not granted under 
this License, such additional rights are deemed to be 
included in the License; this License is not intended to
restrict the license of any rights under applicable law.

\end{verbatim}

